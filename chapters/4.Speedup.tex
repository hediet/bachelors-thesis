\chapter{Speedup-Konstruktionen}
\label{chap:SpeedupKonstr}

Bei der Fragestellung, wie Linearzeit-Zellularautomaten zu Echtzeit-Zellularautomaten verhalten,
ist es essentiell, zu untersuchen, inwiefern sich Zellularautomaten beschleunigen lassen.
Sogenannte Speedup-Sätze für Zellularautomaten sind schon einige bekannt (siehe \cite{Kutrib2009, MAZOYER199259}).
Um dem Formalismus dieser Arbeit gerecht zu werden, werden einige dieser Sätze in diesem Kapitel neu bewiesen \acs{bzw.} verallgemeinert.
In \cref{chap:AdvAuto} und \cref{chap:EingeschrAuto} werden die hier vorgestellten Speedup-Konstruktionen wieder aufgegriffen.

\section{Speedup um konstant viele Schritte}

Ein zentraler Speedup-Satz zeigt, dass Zellularautomaten um einen Schritt beschleunigt werden können,
wenn Annahmen über den Rand getroffen werden können. \cref{satzRauteTot} zeigt, dass wir diese Annahmen treffen dürfen.
Eine Besonderheit der hier vorgestellten Konstruktion ist, dass sie in einem gewissen Rahmen passive Zustände erhält.
Das wird in \cref{chap:EingeschrAuto} wieder aufgegriffen.

\begin{definition}[1-Schritt-Speedup-Konstruktion]
    Sei $(Q, \delta) \in \CA$ ein Zellularautomat.
    Definiere den Zellularautomaten $Sp(C) := (Q', \delta')$ mit $Q' := Q \times (Q^Q \cup \finset{ \perp })$.
    Identifiziere $Q$ mit $Q \times \finset{ \perp }$.
    Für ein Element $q \in Q'$ bezeichnen $q_q \in Q$ und $q_f \in Q^Q \cup \finset{ \perp }$ die erste \acs{bzw.} zweite Komponente, sodass gilt: $q = (q_q, q_f)$.
    Die Zustandsübergangsfunktion $\delta'$ sei wie folgt definiert:
    \[
        \delta'((a_q, a_f), (b_q, b_f), (c_q, c_f)) :=
        \begin{cases}
            (\delta(a_q, b_q, c_q), \perp) & \text{falls } c_f = \perp \\
            \phi_{c_f(b_q)}(\delta(a_q, b_q, c_q)) & \text{sonst} 
        \end{cases}
    \]
    
    Definiere weiter $\phi_c: Q \to Q'$ für ein $c \in Q$:
    \[
        \phi_c(b) := (b, \; Q \ni a \mapsto \delta(a, b, c) \in Q)
    \]
    
    \begin{figure}[h!]
        \begin{center}
        \includesvg[width=190pt]{images/KonstanterSpeedup}
        \end{center}
        \caption{1-Schritt Speedup}
        
        Für einen Zustand $q$ beschreiben die eingezeichneten Pfeile die Abbildung $q_f$:
        Der jeweils linke Zustand wird auf den unteren abgebildet, sofern der Zustand rechts
        zum Zeitpunkt $0$ bekannt ist.
        \label{fig:KonstanterSpeedup}
    \end{figure}
\end{definition}

\begin{satz}[Korrektheit der 1-Schritt-Speedup-Konstruktion]
    \label{satzSpeedupConstruction}
    Sei $C = (Q, \delta) \in \CA$ ein Zellularautomat, $\# \in Q$ ein ausgezeichneter Zustand
    und $c \in Q^{\Z}$ eine Konfiguration.
    Setze $C' := Sp(C)$.
    Sei $c' \in {Q_{C'}}^{\Z}$ eine Konfiguration mit $c'_j \in \finset{c_j, \phi_{\#}(c_j)}$ für alle $j \in \Z$. Sei $t \in \Nz$ und $i \in \Z$.
    
    \begin{enumerate}
        \item
            Falls $c'_{t+i} \in Q \times Q^Q$, dann $\Delta^t_{C'}(c')_i \in Q \times Q^Q$.
        \item
            Es gilt: $(\Delta^t_{C'}(c')_i)_q = \Delta^t_{C}(c)_i$.
        \item
            Falls $\Delta^t_{C'}(c')_i \in Q \times Q^Q$ und $c_{i+t+1} = \#$, dann gilt, wie in \cref{fig:KonstanterSpeedup} veranschaulicht:
            \[(\Delta^t_{C'}(c')_i)_f((\Delta^t_{C'}(c')_{i-1})_q) = \Delta^{t+1}_C(c)_i\]
    \end{enumerate}
\end{satz}
\begin{proof}
    Führe für diesen Beweis folgende Notation ein: $c'^{t}_i := \Delta^{t}_{C'}(c')_i$ und $c^t_i := \Delta^{t}_{C}(c)_i$.

    \begin{enumerate}
        \item
            Die Aussage gilt offensichtlich für $t = 0$. Angenommen, die Aussage gilt nun für ein $t \in \Nz$ und alle $i \in \Z$.
            Sei $c'_{t+1+i} \in Q \times Q^Q$. Dann gilt nach Definition von $\delta_{C'}$ und wegen $c'^{t}_{i+1} \in Q \times Q^Q$:
            \[
                c'^{t+1}_i = \delta_{C'}(c'^{t}_{i-1}, c'^{t}_{i}, c'^{t}_{i+1}) \in Q \times Q^Q
            \]
            Damit gilt die Aussage auch für $t + 1$.
            
        \item Gilt offensichtlich.
        
        \item
            Beweis durch Induktion über $t$.
            Für $t = 0$ gilt mit $c'^0_i \in Q \times Q^Q$ und $c_{i+1} = \#$:
            \begin{align*}
                  && (c'^0_i)_f((c'^0_{i-1})_q) \\
                = && (c'_i)_f((c'_{i-1})_q) \\
                = && (\phi_s(c_i))_f(c_{i-1}) \\
                = && \delta(c_{i-1}, c_i, c_{i+1}) = c^{1}_i
            \end{align*}
            
            Es gelte nun die Behauptung für ein $t \in \Nz$ und alle $i \in \Z$.
            Sei $c'^{t+1}_i \in Q \times Q^Q$ und $c_{i+(t+1)+1} = \#$.
            
            Nach Definition von $\delta_{C'}$ gilt $(c'^{t}_{i+1})_f \in Q^Q$.
            Wegen $c_{(i+1)+t+1} = \#$ gilt nach Induktionsvoraussetzung:
            \[
                (c'^t_{i+1})_f((c'^t_{i})_q) = c^{t+1}_{i+1}
            \]
            
            Es gilt:
            \begin{align*}
                  && (c'^{t+1}_i)_f = \delta_{C'}(c'^{t}_{i-1}, c'^{t}_{i}, c'^{t}_{i+1})_f \\
                = && \phi_{(c'^{t}_{i+1})_f((c'^{t}_{i})_q)}(\delta((c'^{t}_{i-1})_q, (c'^{t}_{i})_q, (c'^{t}_{i+1})_q))_f \\
                = && \phi_{c^{t+1}_{i+1}}(c^{t+1}_{i})_f \\
                = && Q \ni a \mapsto \delta(a, c^{t+1}_{i}, c^{t+1}_{i+1}) \in Q
            \end{align*}
            
            Und damit:
            \begin{align*}
                  && (c'^{t+1}_i)_f((c'^{t+1}_{i-1})_q) \\
                = && \delta(c^{t+1}_{i-1}, c^{t+1}_{i}, c^{t+1}_{i+1}) \\
                = && c^{t+2}_i
            \end{align*}
            
            
    \end{enumerate}
\end{proof}

Rekursiv lässt sich mithilfe des gezeigten Satzes ein Zellularautomat um konstant viele Schritte beschleunigen.
Wird \cref{satzRauteTot} ausgenutzt, lässt sich die folgende Aussage zeigen.

\begin{satz}[$k$-Schritt Speedup]
    \label{satzEchtzeitSpeedup}
    Es gilt:
    \[
        \CAL{\CART, \CALeft} =
            \CAL{\CAT, \CALeft, \mathrm{time}(\set{ n \mapsto \max \finset{0,  n + k - 1} }{ k \in \Nz })}
    \]
\end{satz}
\begin{proof}
    Es reicht zu zeigen, dass für alle $k \in \Nz$ gilt:
    \[
        \CAL{\CAT, \CALeft, \mathrm{time}(
                    n \mapsto \max \finset{0,  n + k - 1}
                )
            }
        \supseteq
        \CAL{\CAT, \CALeft, \mathrm{time}(
                    n \mapsto n + k
                )
            }
    \]
    Induktiv gilt dann die Behauptung.

    Sei $L \in \CAL{\CAT, \CALeft, \mathrm{time}(
                    n \mapsto n + k  )}$.
    Dann existiert ein $t$-haltender Zellularautomat $C$,
    sodass $\Delta_C^{|w|+k}([w])_1 \in F^+_C \Leftrightarrow w \in L$.
    Wegen \cref{satzRauteTot} kann angenommen werden, dass $\#_C$ ein toter Zustand ist.
    
    Setze $C' := Sp(C)$. Identifiziere $c \in \Sigma_C$ mit $\phi_{\#_C}(c)$ in $\Sigma_{C'}$ und setze $\#_{C'} := \phi_{\#_C}(\#_C)$.
    Definiere $F_{C'}^+$ wie folgt: \[
        F_{C'}^+ := \set{ q \in Q_C \times Q_C^{Q_C} }{ q_f({\#_C}) \in F_C^+ }
    \]
    
    Es kann angenommen werden, dass $|w| + k - 1 \geq 0$, da mit \cref{endlichVieleAusnahmen} das leere Wort ignoriert werden kann.
    Da $[w]_{|w|+k} \in Q_C \times Q_C^{Q_C}$, $[w]_{|w|+k+1} = \#_C$ und $\Delta^{|w|+k-1}_C([w])_0 = \#_C$, folgt mit \cref{satzSpeedupConstruction}
    $\Delta_{C'}^{|w|+k-1}([w])_1 \in F'^+_C
    \Leftrightarrow \Delta_C^{|w|+k}([w])_1 \in F^+_C
    \Leftrightarrow w \in L$.
    Also $L \in \CAL{\CAT, \CALeft, \mathrm{time}(
                    n \mapsto \max \finset{0,  n + k - 1}
                )}$.
\end{proof}


Mithilfe der Speedup-Sätze kann beispielsweise nun einfach gezeigt werden,
dass Wortlängen einer \CAM{\CART, \CALeft}-Sprache \acs{OBdA.} Vielfache einer beliebigen Zahl $k \in \N$ sind:
\begin{satz}
    \label{wrtRealtimeLengthIdeal}
    Sei $L \subseteq \Sigma^*$ eine Sprache, $\chr{x} \not\in \Sigma$, $k \in \N$.
    Definiere zu $L$:
    \[
        L(k) := \set{ w \chr{x}^{\min \set{i \in \Nz }{ (|w| + i) \in k\Nz }} } { w \in L }
    \]
    
    Es gilt: $L(k) \in \CAL{\CART, \CALeft}$ genau dann, wenn $L \in \CAL{\CART, \CALeft}$.
\end{satz}
\begin{proof}
    Angenommen $L(k) \in \CAL{\CART, \CALeft}$.
    Dann existiert ein Echtzeit-Automat $C'$
    mit $L_{C'} = L(k)$.
    Da bei links-erkennenden Echtzeit-Zellularautomaten
    das Akzeptanzverhalten unabhängig vom Verhalten des rechten
    Rands ist, kann der rechte Rand mit \chr{x} identifiziert werden (\acs{OBdA.} kann der rechte Rand vom linken unterschieden werden).
    Dieser Automat erkennt nun $L$
    in $t$ Schritten mit $t \in \finset{n-1, ..., n-1+k-1}$.
    Setze $C'_i$ auf den $i$-Schritt-Speedup-Automaten von $C'$
    und konstruiere den Automaten $C$, der die endlich vielen Automaten
    $C'_0$, $C'_1$ bis $C'_{k-1}$ parallel ausführt und genau
    dann akzeptiert, wenn einer von ihnen akzeptiert.
    Damit erkennt $C$ die Sprache $L$ in Echtzeit.
    
    Umgekehrt sei nun $L \in \CAL{\CART, \CALeft}$.
    Dann existiert ein Echtzeit-Automat $C$
    mit $L_C = L$.
    Konstruiere den Automat $C'$ wie folgt:
    $C'$ löscht in genau $k-1$ Schritten die
    Zellen am rechten Rand im Zustand \chr{x}.
    Sind es mehr als $k-1$ solcher Zellen, lehne das Wort ab, bei weniger
    wird gewartet, bis die $k-1$ Schritte vorbei sind.
    Anschließend wird auf dieser neuen Konfiguration
    nun der Automat $C$ ausgeführt.
    Parallel wird durch einen Automaten $X$ getestet,
    ob die Länge des Wortes ein Vielfaches von $m$ ist.
    Das Eingabewort soll nun genau dann von $C'$ akzeptiert werden,
    falls $C$ und $X$ akzeptieren.
    Dieser Automat $C'$ akzeptiert nun $L(k)$ in $n-1+k-1$
    vielen Schritten.
    Da $k$ fest ist, garantiert \cref{satzEchtzeitSpeedup}
    nun die Existenz eines $L(k)$-erkennenden Echtzeitautomaten.
\end{proof}


\section{Speedup von Echtzeit-Zellularautomaten}

Im vorherigen Abschnitt wurde gezeigt, dass Zellularautomaten,
die nur eine konstante Anzahl Schritte von Echtzeit entfernt sind,
auf Echtzeit beschleunigt werden können.
In diesem Kapitel wird gezeigt, dass sich Echtzeit-Zellularautomaten in gewisser Hinsicht auch beschleunigen lassen.
Offensichtlich ist die Zeiteinschränkung von Echtzeit-Zellularautomaten bereits minimal:
Wenn ein Wort $w$ in weniger als $|w|-1$ Schritten erkannt wird, kann das letzte Zeichen des Eingabewortes nicht zur Akzeptanz beitragen.
Allerdings lässt sich die Anzahl benötigter Schritte tatsächlich weiter verringern,
wenn dafür die Position der Zelle, die Akzeptanz oder Ablehnung signalisiert, in Richtung Mitte wandert.

Zunächst wird aber das folgende Lemma benötigt.

\begin{lemma}
    \label{linksunabhaengigSpeedup}
    Sei $C \in \CA$ ein links-unabhängiger Zellularautomat (wie in \cref{fig:LinUnabhSpeedup1}) und $2 \leq k \in \N$.
    Sei ferner $\# \in Q_C$ ein passiver und initialer Zustand.
    Dann gibt es einen links-unabhängigen Zellularautomaten
    $C' \in \CA$, sodass für $i \in \Z$ mit $i \leq 0$ und $t \in \Nz$
    und alle Konfigurationen $c$ mit $c_p = \#$ für $p \leq 0$ oder $p \geq p_0$ gilt:
    \[
        \Delta^{t}_{C'}(c)_i =
            w \in Q^k
            \text{ mit } w_j := \Delta^{ t + i - ki+k-j }_C(c)_{ ki-k+j }
    \]
    
    Für $k = 2$ erhält man, wie in \cref{fig:LinUnabhSpeedup2} gezeigt:
    \[
        \Delta^{t}_{C'}(c)_i = ( \Delta^{ t - i+1 }_C(c)_{ 2i-1 },
            \Delta^{ t - i }_C(c)_{ 2i })
    \]

    Für $k = 3$ ergibt sich:
    \[
        \Delta^{t}_{C'}(c)_i = ( \Delta^{t-2i+2}_C(c)_{3i-2},
        \Delta^{t-2i+1}_C(c)_{3i-1},
        \Delta^{t-2i}_C(c)_{3i}
        )
    \]
    
    \begin{figure}[!ht]
        \centering
        \includesvg[width=190pt]{images/LinUnabhSpeedup}
        \caption{Teilweise Ausführung des Automaten $C$}
        \label{fig:LinUnabhSpeedup1}
    \end{figure}
    \begin{figure}[!ht]
        \centering
        \includesvg[width=190pt]{images/LinUnabhSpeedup2}
        \caption{Teilweise Ausführung des Automaten $C'$ für $k = 2$}
        \label{fig:LinUnabhSpeedup2}
    \end{figure}
    
\end{lemma}
\begin{proof}
    Definiere $\delta^{w} \in Q^{|w-1|}$ für $w \in Q^*$ mit $|w| \geq 2$ wie folgt:
    \begin{align*}
        \delta^{w}_i := \begin{cases}
            \delta(\#, w_{|w|-1}, w_{|w|}) & \text{falls } i = |w| - 1 \\
            \delta(\#, w_{i}, \delta^{w}_{i+1}) & \text{sonst}
        \end{cases}
    \end{align*}

    Beispielsweise gilt $\delta^{q_1q_2q_3} = \delta(\#, q_1, \delta(\#, q_2, q_3))\delta(\#, q_2, q_3)$.
    
    Setze $Q' := Q_C \cup Q_C^k$. Identifiziere $\#$ mit $\#^k$. Definiere $\delta'$ wie folgt:
    \begin{alignat*}{2}
        & \delta'(\cdot, q_b \in Q_C \setminus \finset{\#}, & q_c \in Q_C) & := \delta(\#, q_b, q_c) \\
        & \delta'(\cdot, w_b \in Q_C^k, & q_c \in Q_C) & := \delta^{w_bq_c} \\
        & \delta'(\cdot, x_b \in Q', & w_c \in Q^k_C) & := \delta'(\#, x_b, (w_c)_1)
    \end{alignat*}
    
    Da $\#$ ein initialer Zustand in $C$ ist und auch in $\delta'$ durch die Identifikation mit $\#^k$ passiv bleibt, verhält sich $\delta'$ für Zellen mit positiver Position genauso wie $\delta$.
    
    Definiere nun die Hilfsfunktionen $\psi$ und $\phi$:
    \begin{alignat*}{2}
        & \psi(i, j) && := ki+j-k \\
        & \phi(t, i, j) && := t + i - \psi(i, j) \\
        & c^t_i && := \Delta^t_i(c)
    \end{alignat*}

    Für $\phi$ und $\psi$ gelten für $j \in \finset{1, ..., k}$ folgende triviale Fakten, die im Folgenden verwendet werden:
    \begin{alignat*}{2}
        & \psi(i, j) && \leq 0 \\
        & \phi(0, i, j) && \leq -\psi(i, j) \\
        & \psi(1, 1) && = 1 \\
        & \psi(i + 1, 1) && = \psi(i, k) + 1 \\
        & \psi(i, j) && = \psi(i, j - 1) + 1 \\
        & \phi(t, 1, 1) && = t \\
        & \phi(t, i+1, 1) && = \phi(t, i, k) \\
        & \phi(t + 1, i, j) && = \phi(t, i, j-1) \\
    \end{alignat*}
    Da im Beweis nur diese erwähnten Fakten über $\psi$ und $\phi$ verwendet werden, ist es nicht verwunderlich, dass diese Fakten $\psi$ und $\phi$ für $i \leq 0$ und $t \geq 0$ eindeutig bestimmen.

    Für die Wahl $C' := (Q', \delta')$ ist nun zu für alle $i \leq 0$ und $t \in \Nz$ zeigen: \[
        (\Delta^{t}_{C'}(c)_i)_j = c^{\phi(t, i, j)}_{\psi(i, j)}
    \]
    
    Beweis der Aussage über Induktion nach $t \in \Nz$.
    
    Sei $t = 0$. Da $\#$ passiv ist, gilt $c^{t'}_{i'} = \#$ für $0 \leq t' \leq -i'$ und $i \leq 0$.
    Wegen $\phi(0, i, j) \leq -\psi(i, j)$ und $\psi(i, j) \leq 0$ folgt
    $(\Delta^{0}_{C'}(c)_i)_j = \# = c^{\phi(0, i, j)}_{\psi(i, j)}$ für alle $j \in \finset{1, ..., k}$.
    
    Die Aussage gelte nun für ein $t \in \Nz$.
    
    Sei $x_a := \Delta^{t}_{C'}(c)_i$ und $x_b := \Delta^{t}_{C'}(c)_{i+1}$.
    Wegen $i \leq 0$ folgt $x_a \in Q^k_C$.
    
    
    Nach Induktionsvermutung gilt $(x_a)_j = c^{\phi(t, i, j)}_{\psi(i, j)}$.
    
    Wähle $q :=
    \begin{cases}
        x_b & \text{falls } x_b \in Q_C \\
        (x_b)_1 & \text{falls } x_b \in Q^k_C \\
    \end{cases}$.
    
    Für diese Wahl von $q$ gilt nun $q = c^{\phi(t, i + 1, 1)}_{\psi(i + 1, 1)}$:
    
    Angenommen, $x_b \in Q_C$. Nach Definition von $\delta'$ folgt $i = 0$
    und $q = x_b = c^t_1 = c^{\phi(t, 1, 1)}_{\psi(1, 1)} = c^{\phi(t, i + 1, 1)}_{\psi(i + 1, 1)}$.
    
    Andernfalls gilt $x_b \in Q^k_C$ und damit $i < 0$.
    Nach Induktionsvermutung gilt dann
    $q = (x_b)_1 = c^{\phi(t, i + 1, 1)}_{\psi(i + 1, 1)}$.
    
    Zeige $(\Delta^{t+1}_{C'}(c)_i)_j =  c^{\phi(t + 1, i, j)}_{\psi(i, j)}$ durch endliche, absteigende Induktion über $j$.
    
    Für $j = k$ gilt:
    \begin{align*}
        & (\Delta^{t+1}_{C'}(c)_i)_k \\
        & = \delta'(\#, x_a, x_b)_k \\
        & = \delta^{x_a q}_k \\
        & = \delta(\#, (x_a)_k, q) \\
        & = \delta(\#, c^{\phi(t, i, k)}_{\psi(i, k)}, c^{\phi(t, i + 1, 1)}_{\psi(i + 1, 1)}) \\
        & = \delta(\#, c^{\phi(t, i, k)}_{\psi(i, k)}, c^{\phi(t, i, k)}_{\psi(i, k) + 1}) \\
        & = c^{\phi(t + 1, i, k)}_{\psi(i, k)} \\
    \end{align*}
    
    Es gelte die Induktionsvermutung nun für $1 < j \leq k$. Dann gilt:
    \[
    \delta^{x_a q}_j = (\Delta^{t+1}_{C'}(c)_i)_j = c^{\phi(t + 1, i, j)}_{\psi(i, j)}
    \]
    
    Zeige nun die Vermutung für $j-1$:
    \begin{align*}
        & (\Delta^{t+1}_{C'}(c)_i)_{j-1} \\
        & = \delta'(\#, x_a, x_b)_{j-1} \\
        & = \delta^{x_a q}_{j-1} \\
        & = \delta(\#, (x_a)_{j-1}, \delta^{(x_a q)}_j) \\
        & = \delta(\#, c^{\phi(t, i, j-1)}_{\psi(i, j-1)}, c^{\phi(t + 1, i, j)}_{\psi(i, j)}) \\
        & = \delta(\#, c^{\phi(t, i, j-1)}_{\psi(i, j-1)}, c^{\phi(t, i, j-1)}_{\psi(i, j-1)+1}) \\
        & = c^{\phi(t + 1, i, j-1)}_{\psi(i, j - 1)} \\
    \end{align*}
    
    Damit ist auch der Induktionsschritt gezeigt und der Beweis vollendet.
\end{proof}

Damit lässt sich dann der folgende Satz zeigen.

\begin{satz}
    \label{CAgfSpeedup}
    Sei $C \in \CA$ ein Zellularautomat.
    Dann gibt es einen Zellularautomaten $C''' \in \CA$ und Funktionen $g_1$ und $g_2$, sodass für alle $p \in \N$ gilt:
    \begin{align*}
        g_1(\Delta_{C'''}^{2p-1}(c)_p) & = \Delta_C^{3p-2}(c)_1 \\
        g_2(\Delta_{C'''}^{2p}(c)_{p+1}) & = (\Delta_C^{3p-1}(c)_1, \Delta_C^{3p}(c)_1)
    \end{align*}
    
    Ferner existiert dann ein Automat $C_1$ und eine Funktion $f$, sodass für $i \geq 1$ gilt:
    \[
        f(\Delta_{C_1}^{2i+1}(c)_i) = (\Delta_C^{3i-3}(c)_1, \Delta_C^{3i-2}(c)_1, \Delta_C^{3i-1}(c)_1)
    \]
    
    \cref{fig:EchtzeitSpeedup} zeigt die Zustands-Abhängigkeiten der Automaten zueinander.
    Die Kreise stellen die Zustände der Ausführung des Automaten $C$ dar, die Sechsecke zeigen das Bild von $g_1$ bzw. $g_2$
    der Zustände von $C'''$ und die Fünfecke das Bild von $f$ der Zustände von $C_1$. Die eingezeichneten Pfeile
    zeigen, wodurch sich die Zustände ergeben.
    
    \begin{figure}[H]
        \centering
        \includesvg[width=175pt]{images/EchtzeitSpeedup}
        \caption{Visualisierung der Ausführung der Automaten $C$ (Zustände rund), $C'''$ (Zustände sechseckig) und $C_1$ (Zustände fünfeckig)}
        \label{fig:EchtzeitSpeedup}
    \end{figure}
\end{satz}
\begin{proof}
    Wähle $C'$ wie in \cref{zellautoZuLinksunabhaengig}, sodass
    für $\forall c \in Q^{\Z}, \forall t \in \Nz, \forall i \in \Z$ gilt:
    \[
        \Delta_{C'}^{t}(c)_i =
        \begin{cases}
            \Delta_C^{\frac{t}{2}}(c)_{i+\frac{t}{2}} & \text{ falls } $t$ \text{ gerade, } \\
            (\Delta_C^{\frac{t - 1}{2}}(c)_{i+\frac{t - 1}{2}}, \Delta_C^{\frac{t-1}{2}}(c)_{i+\frac{t+1}{2}})  & \text{ sonst.}
        \end{cases}
    \]
    
    Wähle $C''$ wie in \cref{linksunabhaengigSpeedup}, sodass für $i \in \Z$ mit $i \leq 0$ und $t \in \Nz$
    und alle Konfigurationen $c$ mit $c_p = \#$ für $p \leq 0$ oder $p \geq p_0$ gilt:
    \[
        \Delta^{t}_{C''}(c)_i = ( \Delta^{t-2i+2}_{C'}(c)_{3i-2},
        \Delta^{t-2i+1}_{C'}(c)_{3i-1},
        \Delta^{t-2i}_{C'}(c)_{3i}
        )
    \]
    
    Wähle $C'''$ wie in \cref{linksunabhaengigZuZellauto}, sodass für $\forall c \in Q^{\Z}$,
    $\forall t \in \Nz$, $\forall i \in \Z$ gilt:
    \[
        \Delta^t_{C'''}(c)_{i} = \Delta^{2t}_{C''}(c)_{i-t}
    \]
    
    Wähle $g_1(q \in Q''') := q_3$ und $g_2(q \in Q''') := ((q_2)_1, q_1)$.
    
    Es gilt dann:
    \begin{align*}
          & g_1(\Delta_{C'''}^{2p-1}(c)_p) \\
        = & (\Delta_{C''}^{4p-2}(c)_{1-p})_3 \\
        = & \Delta_{C'}^{6p-4}(c)_{3-3p} \\
        = & \Delta_{C}^{3p-2}(c)_1
    \end{align*}
    
    Außerdem gilt:
    \begin{align*}
          & g_2(\Delta_{C'''}^{2p}(c)_{p+1}) \\
        = & g_2(\Delta_{C''}^{4p}(c)_{1-p}) \\
        = & ((\Delta_{C'}^{6p-1}(c)_{2-3p})_1, \Delta_{C'}^{6p}(c)_{1-3p}) \\
        = & (\Delta_C^{3p-1}(c)_1, \Delta_C^{3p}(c)_1)
    \end{align*}

    Anhand von \cref{fig:EchtzeitSpeedup} ist nun ersichtlich, wie $C_1$ und $f$ konstruiert werden müssen.
\end{proof}

\section{Speedup um einen Faktor durch Komprimierung}

Wenn das Eingabewort komprimiert werden darf, lässt sich ein noch viel stärkerer Speedup-Satz zeigen.

\begin{definition}[Speedup-Konstruktion $S_k(C)$, $S_k(c)$, $\gamma_q$]
    \label{factorSpeedupConstruction}
    Es sei ein Zellularautomat $C := (Q, \delta) \in \CA$ und ein $k \in \N$ gegeben.
    Definiere $S_k(C) := (Q', \delta')$ mit $Q' := Q^k$.
    
    Definiere für $q \in Q$ und $l, r \in Q^*$ die Konfiguration $c(q, l, r) \in Q^{\Z}$:
    \[
        c(q, l, r)_i := \begin{cases}
            (lr)_{i + |l|} & \text{ falls } 1-|l| \leq i \leq |r| \\
            q & \text{ sonst}
        \end{cases}
    \]
    
    Wähle $q \in Q$ beliebig.
    Definiere $\delta' : {Q'}^3 \to Q'$ wie folgt:
    \[
        \delta'(a, b, c) := \Delta_C^k(c(q, a, bc))[1..k]
    \]
    
    Definiere die Konfigurationskomprimierung $S_k(c) \in Q'^\Z$ für eine Konfiguration $c \in Q^\Z$:
    \[
        S_k(c)_i := c_{k(i - 1) + 1}c_{k(i - 1) + 2}...c_{k(i - 1) + k} \in Q^k
    \]
    
    Definiere außerdem noch $\gamma_q : Q' \to Q$ mit
    $\gamma_q(a) := \Delta^{k-1}_C(c(q, \epsilon, a))_1$.
\end{definition}

\begin{satz}[Korrektheit der Speedup-Konstruktion]
    \label{factorSpeedupConstructionCorrectness}
    Es sei ein Zellularautomat $C := (Q, \delta) \in \CA$, ein $k \in \N$ und eine Konfiguration $c: \Z \to Q$ gegeben.
    Für beliebige $t \in \Nz, i \in \Z$ und $j \in \finset{1, ..., k}$ gilt dann:
    \[
        (\Delta^t_{S_k(C)}(S_k(c))_i)_j = \Delta^{kt}_C(c)_{k(i-1)+j}
    \]
    Ferner gilt für $q := (\Delta^t_{S_k(C)}(S_k(c))_{i-1})_k$:
    \[
        \gamma_{q}( \Delta^t_{S_k(C)}(S_k(c))_{i} )
        = \Delta^{kt+k-1}_C(c)_{k(i-1)+1}
    \]
\end{satz}
\begin{proof}
    Die Aussage kann durch einfaches Nachrechnen gezeigt werden, auf das an dieser Stelle verzichtet wird.
\end{proof}

Die Konstruktion aus \cref{factorSpeedupConstruction} kann beispielsweise verwendet werden, um folgenden Satz zu beweisen.

\begin{theorem}
    \label{linSpeedup}
    $\CAL{\CALT, \CALeft} = \CAL{\CAT, \CALeft, \mathrm{time}(\finset{ n \mapsto 2n }) }$
\end{theorem}
\begin{proof}
    Mithilfe der in \cref{chap:ErweiterteNakamuraKonstr} und \cref{factorSpeedupConstruction} vorgestellten Konstruktionen kann das Theorem leicht bewiesen werden.
    Weitere Beweise finden sich in \cite{MAZOYER199259} und \cite{IBARRA1988225}
\end{proof}

