\chapter{Advice-Zellularautomaten}
\label{chap:AdvAuto}

Dieses Kapitel beschäftigt sich damit, inwiefern die Mächtigkeit bestimmter Klassen von Zellularautomaten verändert wird,
wenn der Automat zusätzlich zum Eingabewort ein vom Eingabewort abhängiges Unterstützungswort (im Folgenden Advice genannt) erhält.
Insbesondere wird gezeigt, dass bestimmte solcher Klassen unter Zuhilfenahme ausgewählter Advices abgeschlossen sind.
Solche Advices heißen dann verträglich bezüglich dieser Klasse.
Diese Abschlusseigenschaft ist nützlich, um die Konstruktion einiger Echtzeit-Zellularautomaten zu vereinfachen und um besser verstehen zu können,
wie mächtig Echtzeit-Zellularautomaten sind.

\section{Definition}

Es wird zunächst definiert, was ein Advice ist.

\begin{definition}
    Ein \emph{$\Sigma$-$\Gamma$-Advice} (Hinweis) $\advA$ ist eine Abbildung
        $\advA: \Sigma^* \to \Pot(\Gamma^*)$ mit $\forall w \in \Sigma^*: \advA(w) \subseteq \Sigma^{|w|}$.
    
    Wenn $\forall w \in \Sigma^*: |\advA(w)| = 1$, kann $\advA$ auch
    als Abbildung $\advA : \Sigma^* \to \Gamma^*$ aufgefasst werden.
    $\advA$ heißt dann \emph{deterministisch}.

    Ein Hinweis kann eine ganze Menge von Unterstützungswörtern bereitstellen.
    In diesem Fall wird ein Wort akzeptiert,
    wenn der Automat das Eingabewort mit Unterstützung irgendeines der bereitgestellten Wörter akzeptiert.
\end{definition}

Um zu beschreiben, wann ein Automat mithilfe eines Advices ein Wort akzeptiert oder ablehnt,
müssen zunächst die Funktionen $\splitw$ und $\joinw$ definiert werden.

\begin{definition}
    Definiere injektive Funktion $\splitw: (\Sigma \times \Gamma)^* \to (\Sigma^* \times \Gamma^*)$
    um ein Wort über zwei Alphabete in zwei Wörter über je ein Alphabet zu zerlegen:
    \[
        \forall v \in \Sigma^*, w \in \Gamma^{|v|}: \splitw(\Spvek{v_1; w_1}...\Spvek{v_{|v|}; w_{|v|}}) := \Spvek{v; w}
    \]
    
    Definiere $\joinw: \splitw((\Sigma \times \Gamma)^*) \to (\Sigma \times \Gamma)^*$,
    um zwei Wörter über je ein Alphabet zu einem Wort über zwei Alphabete zusammenzufügen:
    \[
        \joinw(w) := \splitw^{-1}(w)
    \]
\end{definition}

\begin{exmp}
    \[
        \splitw(\Spvek{\chr{1}; \chr{a}}\Spvek{\chr{2}; \chr{b}}\Spvek{\chr{3}; \chr{c}}) = \Spvek{\chr{123}; \chr{abc}}
    \]
    
    \[
        \joinw(\Spvek{\chr{123}; \chr{abc}}) = \Spvek{\chr{1}; \chr{a}}\Spvek{\chr{2}; \chr{b}}\Spvek{\chr{3}; \chr{c}}
    \]        
\end{exmp}

Jetzt kann definiert werden, welche Auswirkung ein Advice auf einen Zellularautomaten hat.

\begin{definition}
    Es sei ein Advice $\advA$ und ein Zellularautomat $C \in \ECA$ gegeben.
    Mit Unterstützung des Advices $\advA$ kann $C$ nun folgende Sprache erkennen:
    \[
         L_{\advA, C} := \set{ w \in \Sigma^*}{\exists v \in \advA(w): \joinw(\Spvek{w; v}) \in L(C)}
    \]
    
    Für $M(\advA) := \set{ \joinw(\Spvek{ w; v}) }
        { w \in \Sigma^* \land v \in \advA(w) }$
    kann $L_{\advA, C}$ auch wie folgt aufgefasst werden:
    \[
        L_{\advA, C} = \splitw(
            L(C) \cap M(\advA)
        )_1
    \]
    
    Folgender $\ECA$-Funktor stellt einer Klasse von Zellularautomaten einen gegebenen Advice zur Verfügung:
    \[
        \orakel(\advA)(M \subseteq \ECA) :=
            \set{ (Q_C, \delta_C, \Sigma, \#_C, L_{\advA, C}, (S_C, C))} { C \in M, \; \Sigma_C = \Sigma \times \Gamma }
    \]
\end{definition}

Eine Fragestellung, die sich bei solchen Advices zwingend auftut, ist, ob ein gegebener Advice eine Klasse von Zellularautomaten echt mächtiger macht.
Die folgende Definition führt dafür einen entsprechenden Begriff ein.

\begin{definition}[$\advA$ ist $M$-verträglich]
    Sei $M \subseteq \ECA$ eine Menge von Zellularautomaten und $\advA$ ein Advice.
    $\advA$ heißt $M$-verträglich, falls gilt:
    \[
        \mathcal{L}(M) = \mathcal{L}(M^\advA)
    \]
\end{definition}

\section{Ergebnisse}

Offensichtlich verringert ein Advice die Mächtigkeit \enquote{wohlgeformter} Klassen von Zellularautomaten nicht, wie im folgenden Satz formalisiert. Der Satz gilt nicht für beliebige Klassen, da sich zum Beispiel das Alphabet ändert und in der Konstruktion entsprechend gewählt werden muss. Mithilfe von \cref{lemmaIgnoriereAdvice} muss zum Zeigen der Verträglichkeit eines Advices immer nur die Inklusion \enquote{$\supseteq$} beachtet werden.

\begin{lemma}
    \label{lemmaIgnoriereAdvice}
    Sei ein Advice $\advA$ und eine Menge
    $F \subseteq {\Nz}^{\Nz}$ von Funktionen gegeben.
    Dann gilt: $\mathcal{L}(\CAM{ \mathrm{time}(F), \CALeft }) \subseteq  \mathcal{L}(\CAM{ \mathrm{time}(F), \CALeft, \orakel(\advA)})$.
\end{lemma}
\begin{proof}
    Der Advice kann ignoriert werden, indem $\Sigma \times \Gamma$ auf $\Sigma$ reduziert wird.
\end{proof}

Der nächste Satz zeigt ein Advice, das vermutlich nicht Echtzeit-Zellularautomaten-verträglich ist.
Wäre es Echtzeit-Zellularautomaten-verträglich, wäre auch Echtzeit genauso mächtig wie Linearzeit.

\begin{satz}
    Sei $(\Sigma \cup \finset{\Box}) ^2 \subseteq \Gamma, \Box \not\in \Sigma$.
    Definiere den Komprimierungs-Advice $\advA_K$:
    \[
        \advA_K(w) := 
                  \Spvek{w_1; w_2} \Spvek{w_3; w_4} ... \Spvek{w_{|w-1|}; w_{|w|}}
                        \; \Spvek{\Box; \Box}^\frac{|w|}{2}
    \]
    Falls $|w|$ ungerade, setze $w := w\Box$.
    Unter dem Advice $\advA_K$ fällt Echtzeit mit Linearzeit zusammen:
    \[
        \CAL{\CART, \CALeft, \orakel({\advA})} = \CAL{\CALT, \CALeft}
    \]
\end{satz}
\begin{proof}
    Die Inklusion \enquote{$\subseteq$} ist klar: Offensichtlich ist $\advA_K$ $\CAL{\CALT, \CALeft}$-verträglich.
    Sei nun $L \in \CAL{\CALT, \CALeft}$.
    Nach \cref{linSpeedup} existiert ein Zellularautomat $C$, der $L$ in $2n$ Schritten erkennt.
    Der Automat $C' := S_2(C)$ aus \cref{factorSpeedupConstruction}
    erkennt $L$ dann in $n$ Schritten auf der komprimierten Konfiguration $S_2([w])$.
    Da die komprimierte Konfiguration durch den Advice bereitgestellt wird, kann $L$ mithilfe des Advices in Echtzeit erkannt werden.
\end{proof}

Nicht-deterministische Advices werden in diesem Kapitel nicht weiter untersucht.
Trotzdem soll der folgende Zusammenhang nicht vorenthalten werden.

\begin{satz}
    Seien $\advA_1$ und $\advA_2$ zwei Advices.
    Definiere $(\advA_1 \cup \advA_2)(w) := \advA_1(w) \cup \advA_2(w)$ und 
    $(\advA_1 \cap \advA_2)(w) := \advA_1(w) \cap \advA_2(w)$.
    
    Sind $\advA_1$ und $\advA_2$ $\CAM{\CART, \CALeft}$-verträglich, so sind
    auch $\advA_1 \cup \advA_2$ und $\advA_1 \cap \advA_2$
    $\CAM{\CART, \CALeft}$-verträglich.
\end{satz}
\begin{proof}
    Die Inklusion \enquote{$\subseteq$} der Verträglichkeit ist mit \cref{lemmaIgnoriereAdvice} klar.
    Wähle $\mathbin{\oplus} \in \finset{ \cup, \cap }$.
    Sei $L \in \CAL{\CART, \CALeft, \advA_1 \mathbin{\oplus} \advA_2}$.
    Es existieren nun die Echtzeit-Zellularautomaten $C$ mit $L_{\advA_1 \mathbin{\oplus} \advA_2, C} = L$, $C_1$ mit $L_{C_1} = L_{\advA_1, C}$ und $C_2$ mit $L_{C_2} = L_{\advA_2, C}$.
    
    Offensichtlich gilt für beide Wahlen von $\mathbin{\oplus}$ dann
    $L_{\advA_1 \mathbin{\oplus} \advA_2, C} = L_{\advA_1, C} \mathbin{\oplus} L_{\advA_2, C} = L_{C_1} \mathbin{\oplus} L_{C_2}$.
    Da Echtzeit-Zellularautomaten unter Vereinigung und Schnitt abgeschlossen sind, folgt die Behauptung.
\end{proof}

Der Satz lässt sich auf eine endliche Menge von Vereinigungen \acs{bzw.} Schnitten fortführen.

\begin{comment}
    \begin{satz}
        Sei $(\Sigma \cup \finset{\Box}) ^2 \subseteq \Gamma, \Box \not\in \Sigma$.
        
        \begin{enumerate}
            \item  $\advA_1(w) := \Gamma^{|w|}$
        \end{enumerate}
        
        Es gilt jeweils:
        \[
            \CAL{\CART, \orakel({\advA_i})} = \CAL{\CALT, \orakel({\advA_i})}
        \]
        
    \end{satz}

    \begin{definition}
        Seien $\advA_1$ und $\advA_2$ zwei Orakel, $M \subseteq \ECA$ (falls nicht angegeben, $M := \ECA^\CART$).
        \[
            \advA_1 \leq_M \advA_2 \; :\Leftrightarrow \; \mathcal{L}(M^{\orakel(\advA_1)}) \subseteq \mathcal{L}(M^{\orakel(\advA_2)})
        \]        
    \end{definition}
    
    \begin{lemma}
        Es gibt kein maximales Orakel.
    \end{lemma}
    
\end{comment}

Die folgende Überlegung zeigt, dass die Fragestellung, ob Linearzeit mit Echtzeit zusammenfällt,
auch für Advice-Zellularautomaten interessant ist.
\begin{satz}
    Sei $\advA$ ein Advice mit $\CAL{\CART, \CALeft, \orakel({\advA})} \neq \CAL{\CALT, \CALeft, \orakel({\advA})}$.
    Dann $\CAL{\CART, \CALeft} \neq \CAL{\CALT, \CALeft}$.
\end{satz}
\begin{proof}
    Angenommen $\mathcal{L}_1 := \CAL{\CART, \CALeft, \orakel({\advA})} \neq \CAL{\CALT, \CALeft, \orakel({\advA})} =: \mathcal{L}_2$.
    Wegen $\mathcal{L}_1 \subseteq \mathcal{L}_2$
    gibt es laut Annahme ein $L \in \mathcal{L}_2 \setminus \mathcal{L}_1$.
    Dann gibt es einen Automaten $C \in \CAM{\CALT, \CALeft}$, sodass gilt:
    
    \[
        L = L_{\advA, C} = \set{ w \in \Sigma^*}{\exists v \in \advA(w): \joinw(\Spvek{w; v}) \in L(C)}
    \]
    
    Angenommen, es gäbe einen Automaten $C' \in \CAM{\CART, \CALeft}$ mit $L_{C'} = L_C$.
    Dann $L_{\advA, C} = \set{ w \in \Sigma^*}{\exists v \in \advA(w): \joinw(\Spvek{w; v}) \in L(C')} = L_{\advA, C'}$ und damit
    $L \in \mathcal{L}_1$.
    Widerspruch zur Annahme! Also $L(C) \in \CAL{\CALT, \CALeft} \setminus \CAL{\CART, \CALeft}$.
\end{proof}


\section{\texorpdfstring{$\CAM{\CART, \CALeft}$}{CA-RT-L}-verträgliche Advices}

In diesem Kapitel werden eine Reihe von Advices vorgestellt, die sich als Echtzeit-verträglich herausgestellt haben.
Die meisten hier vorgestellten Advices verwenden $\B$ als Advice-Alphabet
und markieren damit ausgewählte Zeichen des Eingabeworts.
Einfache Markierungen werden durch die im folgenden definierte Funktion erreicht.
Diese Funktion wird auch in \cref{chap:EingeschrAuto} wieder aufgegriffen.

\begin{definition}
    Seien $k, n \in \Nz$ zwei Zahlen. Die Funktion $\mathrm{ones}$ gibt ein Wort der Länge $n$ zurück, sodass die $k$ ersten Zeichen Einsen sind.
    \[
        \mathrm{ones}(k, n) := (\chr{1}^k \chr{0}^n)[1..n]
    \]
\end{definition}

\begin{exmp}
    \[
        \mathrm{ones}(2, 5) = \chr{11000}
    \]
    \[
        \mathrm{ones}(6, 5) = \chr{11111}
    \]        
\end{exmp}

Eine Reihe einfacher Advices stützt sich auf Berechnungen in endlich vielen Schritten.
Dazu muss zunächst definiert werden, was es heißt, wenn ein Zellularautomat eine Funktion berechnet.
\begin{definition}
    Sei $\Gamma$ ein endliches Alphabet.
    Ein $F$-haltender Zellularautomat $C \in \ECA^\CAF$ berechnet eine Funktion $f: \Sigma^* \to \Gamma^*$, wenn
    $\Gamma \subseteq Q_C$, $\#_C \not\in \Gamma$ und 
    \[
        \forall w \in \Sigma^*: \Delta^{\ctime_C(w)}_{C}([w]) = [f(w)]
    \]
    gilt.
\end{definition}

Der folgende Satz zeigt gleich die Echtzeit-Verträglichkeit einer ganzen Klasse von \enquote{einfachen} Advices.
\begin{satz}
    \label{lemmaEinfachesOrakel}
    Wenn $\advA: \Sigma^* \to \Gamma^*$ eine Funktion ist mit $\forall w \in \Sigma^*: |f(w)| = |w|$,
    die von einem $F$-haltenden Zellularautomaten $B$ mit
    $\ctime_C(\cdot) = k$ und $k \in \Nz$ berechnet wird,
    dann ist $\advA$ $\CAM{\CART, \CALeft}$-verträglich.
\end{satz}
\begin{proof}
    Die Inklusion \enquote{$\subseteq$} der $\CAM{\CART, \CALeft}$-Verträglichkeit folgt durch ~\cref{lemmaIgnoriereAdvice}.
    
    Sei also $L \in \CAL{\CART, \CALeft, \orakel({\advA})}$ und $C$
    der Echtzeit-Zellularautomat mit $L_{C, \advA} = L$.
    
    Es gibt einen Zellularautomaten $I$, der in $k$ Schritten die Identität berechnet. Folglich gibt es einen Zellularautomaten $A$, der die Funktion $f: \Sigma^* \to (\Sigma \times \Gamma)^*$ mit
    $f(w) = \joinw(\Spvek{w; f(w)})$ in $k$ Schritten berechnet, in dem $I$ und $B$ für $k$ Schritte parallel ausgeführt werden.
    
    Konstruiere den Automaten $C'$ durch Hintereinanderausführung der Automaten $A$ und $C$ - dies ist möglich, da $k$ fest ist und nicht von der Eingabe abhängt.
    Für ein Wort $w \in \Sigma^*$ braucht der Automat $C'$ $k + |w|$ viele Schritte und es gilt $L_{C'} = L$.
    
    Nach \cref{satzEchtzeitSpeedup} gibt es nun einen Automaten $C'' \in \CAM{\CART, \CALeft}$
    mit $L_{C''} = L_{C'}$.
    Also $L \in \CAL{\CART, \CALeft}$.
\end{proof}

Mit \cref{lemmaEinfachesOrakel} kann nun leicht die Echtzeit-Verträglichkeit einiger konkreter Advices gezeigt werden.
\begin{corollary}
    \label{bestimmteEinfacheAdvices}
    Sei $\Gamma = \B$, $k \in \Nz$.
    Folgende Advices sind $\CAM{\CART, \CALeft}$-verträglich:
    \begin{enumerate}
        \item $\advA_1(w) := \mathrm{ones}(k, |w|)$ 
        \item $\advA_2(w) := \mathrm{ones}(k, |w|)^{\mathrm{Rev}}$
    \end{enumerate}
\end{corollary}
\begin{proof}
    Die Funktionen $\advA_i$ können offensichtlich
    von einem $F$-haltenden Zellularautomat in $k$ Schritten berechnet werden.
    Mit ~\cref{lemmaEinfachesOrakel} folgt die Behauptung.
\end{proof}

\begin{comment}
        $Q' := (Q_C \times \finset{ 0, \dots, c })$. Identifiziere $\Sigma' := \Sigma_C$ mit $(\finset{\chr{0}} \times \Sigma_C) \times \finset{0} $. Setze $\#' := (\#_C, 0)$.
        
        $\delta'(\Spvek{\#_C; k_a}, \Spvek{\Spvek{ \gamma_b; c_b}; k_b}, c) := \Spvek{ \Spvek{\chr{1}; c_b}; k_b + 1}$
        
        $\delta'(\Spvek{\Spvek{\gamma_a; c_a}; k_a}, \Spvek{\Spvek{\gamma_b; c_b}; k_b}, c)
            := \Spvek{ \Spvek{ \max \finset{\gamma_a, \gamma_b}; c_b}; k_b + 1}$
        
        $\delta'(a, \Spvek{q_b; k_b}, c) := \Spvek{q_b; k_b + 1}$
\end{comment}



\begin{comment}
    \begin{satz}
        Für $\advA(w) := a_1...a_{|w|} \in \B^*$ mit \[
           a_i :=
           \begin{cases}
             \chr{1} & \text{falls } \exists j \in \Nz: i = 2^j \\
             \chr{0} & \text{sonst}
           \end{cases}
        \]
        gilt $\CAL{\CART, \orakel({\advA})} = \CAL{\CART}$.
    \end{satz}
    
    
    
    \begin{definition}
        Setze $C_{exp} := (Q, \delta)$ mit $Q := \B \times (\B \cup \finset{ \perp })$.
        $Q$ setzt sich aus zwei Komponenten zusammen.
        Die erste Komponente eines Zustands zu einem Zeitpunkt $t$ an der Position $i$ signalisiert, ob $i+t$ eine Zweierpotenz sein kann.
        In der Abbildung wird dies durch die Farbe blau dargestellt.
        Die zweite Komponente bestimmt den Zustand des Filters. In der Abbildung kennzeichnet Grün einen geöffneten Filterzustand (1), Rot einen geschlossenen (0)
        und keine Markierung einen nicht initialisierten Filter ($\perp$).
        
        Die Zustandsübergangsfunktion $\delta$ ist dann wie folgt definiert:
        \[
            \delta(\cdot, (e, f), (e_R, f_R)) :=
            \begin{cases}
                (e_R, \perp) & \text{ falls } f = \perp \text{ und } f_R \neq 0 \\
                (e_R, 1) & \text{ falls } f = \perp \text{ und } f_R = 0 \\
                (0, f) & \text{ falls } f \neq \perp \text{ und } e_R = 0 \\
                (f, 1 - f) & \text{ falls } f \neq \perp \text{ und } e_R = 1
            \end{cases}
        \]
        
        Für eine Start-Konfiguration
        ${c_{exp}}_i = \begin{cases} (0, \perp) & \text{ falls } i < 1  \\ (1, 1) & \text{ falls } i = 1 \\ (1, \perp) & \text{ falls } i > 1 \end{cases}$
        ergibt sich folgendes Zustands-Zeit-Diagramm:
        
        \includegraphics[scale=.5]{exp1}
        
        Offensichtlich ist $\delta$ links-unabhängig.
        
        Nach Anwendung von \cref{linksunabhaengigSpeedup} ergibt sich folgendes Zustands-Zeit-Diagramm:
        \includegraphics[scale=.5]{exp2}
    \end{definition}
    
    
    \begin{lemma}
        \label{lemmaCExpDetail}
        Setze $c^t_i := \Delta^t_{C_{exp}}(c_{exp})_i$.
        Es bezeichne $(c^t_i)_e$ die erste Komponente 
        und $(c^t_i)_f$ die zweite Komponente des Tupels $c^t_i$.
        Es gilt:
        \begin{itemize}
            \item[1)] $(c^t_i)_e = 1$, falls $i \geq 2$.
            \item[2)] $(c^t_i)_e = 1$, falls $i+t \in 2^{2-i}\N$.
            \item[3)] $(c^t_i)_e = 1$, falls $i+t \in \set{2^k}{0 \leq k \leq 1-i}$.
            \item[4)] $(c^t_i)_e = 0$, falls $i < 2$ und $i+t \not\in 2^{2-i}\N \cup \set{2^k}{0 \leq k \leq 1-i}$.
            \item[5)] $(c^t_i)_f = \perp$, falls $i \geq 2$.
            \item[6)] $(c^t_i)_f = \perp$, falls $i + t < 2^{1-i}$.
            \item[7)] $(c^t_i)_f = \lfloor \frac{i+t}{2^{1-i}} \rfloor \; \mathrm{mod} \; 2$, falls $i < 2$ und $i + t \geq 2^{1-i}$.
        \end{itemize}
        
        
        \[
            (c^t_i)_e = \begin{cases}
                1 & \text{falls } 
            i \geq 2 \text{ oder } i+t \in (2^{2-i})\N \text{ oder } i+t \in \set{2^k}{0 \leq k \leq 1-i} \\
                0 & \text{sonst}
            \end{cases}
        \]
        und
        \[
            (c^t_i)_f = \begin{cases}
                \perp & \text{ falls } i \geq 2 \text{ oder } i + t < 2^{1-i} \\
                \lfloor \frac{i+t}{2^{1-i}} \rfloor \; \mathrm{mod} \; 2 & \text{ sonst }
            \end{cases}
        \]
    \end{lemma}
    \begin{proof}
        Beweis aller Aussagen durch simultane Induktion über $t$.
        Einsetzen zeigt, dass alle Aussagen für $t = 0$ gelten.
        
        Es gelten nun alle Aussagen für ein $t \geq 0$.
        
        \begin{itemize}
            \item[1), 5)]
                Sei $i \geq 2$.
                
                Wegen 1) und 5) gilt $(c^t_{i+1})_e = 1$,
                $(c^t_{i})_f = \perp$ und $(c^t_{i+1})_f = \perp \neq 0$.
                Damit $c^{t+1}_i = \delta_{C_{exp}}(?, (?, \perp), (1, \perp)) = (1, \perp)$.
            
            \item[2)]
                Sei $i+(t+1) = 2^{2-i}k$ für ein $k \in \N$. OBdA ist $i < 2$.
                
                Wegen $i+t \geq 2^{2-i} - 1 \geq 2^{1-i}$ folgt mit 7)
                \begin{align*}
                (c^t_i)_f
                & = \lfloor \frac{i+t}{2^{1-i}} \rfloor \; \mathrm{mod} \; 2 \\
                & = \lfloor \frac{2^{2-i}k - 1}{2^{1-i}} \rfloor \; \mathrm{mod} \; 2
                = \lfloor 2k - \frac{1}{2^{1-i}} \rfloor \; \mathrm{mod} \; 2 \\
                & = 2k-1 \; \mathrm{mod} \; 2 = 1
                \end{align*}
                
                Weiter gilt wegen 2)
                $(c^t_{i+1})_e = 1$, da $(i + 1) + t \in 2^{2-i}\N$.
                Also gilt $(c^{t+1}_i)_e = \delta_{C_{exp}}(?, (?, 1), (1, ?)) = 1$.
                
            \item[3)]
                Sei nun $i + (t + 1) \in \set{2^k}{0 \leq k \leq 1-i}$. OBdA ist $i < 2$.
                
                Wegen 6) gilt $(c^t_i)_f = \perp$, da $i + t < i + t + 1 \leq 2^{1 - i}$.
                Außerdem gilt wegen 3)
                $(c^t_{i+1})_e = 1$, da $i + t + 1 \in \set{2^k}{0 \leq k \leq 1-(i+1)} \; \cup \; \finset{2^{2-(i+1)}}$.
                Damit folgt $(c^{t+1}_i)_e = \delta_{C_{exp}}(?, (?, \perp), (1, ?))_e = 1$.
                
            \item[4)]
                Sei $i < 2$, $i + (t+1) \not\in 2^{2-i}\N \cup \set{2^k}{0 \leq k \leq 1-i}$.
                
                Angenommen, $(c^{t+1}_i)_e = 1$.
                Nach Definition von $\delta_{C_{exp}}$ folgt
                $(c^t_{i+1})_e = 1$ und $(c^t_i)_f \neq 0$.
                Wegen 4) gilt $i+1 \geq 2$ oder $(i+1)+t \in 2^{2-(i+1)}\N \; \cup \; \set{2^k}{0 \leq k \leq 1-(i+1)}$.
                Folglich $i = 1$ oder $i+t+1 \in (2^{2-i}\N) + 2^{1-i}$.
                Letzteres gilt auch im Fall $i = 1$.
                Damit gibt es ein $k \in \N$, sodass $i+t = 2^{2-i}k + 2^{1-i} - 1$.
                Da insbesondere $i + t \geq 2^{2-i} + 2^{1-i} - 1 > 2^{1-i}$ gilt, ergibt sich mit 7) folgendes:
                \begin{align*}
                    (c^t_i)_f
                    & = \lfloor \frac{i+t}{2^{1-i}} \rfloor \; \mathrm{mod} \; 2 \\
                    & = \lfloor \frac{2^{2-i}k + 2^{1-i} - 1}{2^{1-i}} \rfloor \; \mathrm{mod} \; 2
                    = \lfloor 2k + 1 - \frac{1}{2^{1-i}} \rfloor \; \mathrm{mod} \; 2 \\
                    & = 2k \; \mathrm{mod} \; 2
                    = 0
                \end{align*}
                Was der Annahme widerspricht. Damit gilt $(c^{t+1}_i)_e = 0$.
            
            \item[6)]
                Sei nun $i + (t+1) < 2^{1 - i}$.
                Mit 6) folgt wegen $i + t < 2^{1 - i}$, dass $(c^t_i)_f = \perp$.
                
                Angenommen, $(i+1)+t < 2^{1-(i+1)}$. Dann gilt wegen 6) $(c^t_{i+1})_f = \perp \neq 0$.
                Andernfalls, $(i+1)+t \geq 2^{1-(i+1)}$. Dann $2^{1-(i+1)} \leq (i+1)+t < 2^{1-i}$.
                Daraus folgt $\lfloor \frac{(i+1)+t}{2^{1-(i+1)}} \rfloor = 1$.
                Auch hier gilt mit 7) dann $(c^t_{i+1})_f = 1 \neq 0$.
                
                Es folgt $(c^{t+1}_i)_f = \delta_{C_{exp}}(?, (?, \perp), (?, ? \neq 0))_f = \perp$.
            
            \item[7)]
                Sei nun $i < 2$ und $i + (t + 1) \geq 2^{1-i}$.
                \begin{itemize}
                    \item[Fall 1)] $(c^t_i)_f = \perp$.
                        Wegen 7) gilt $i + t + 1 = 2^{1-i}$.
                        Dann $(c^t_{i+1})_f = \lfloor \frac{(i+1)+t}{2^{1-(i+1)}} \rfloor \; \mathrm{mod} \; 2 = 0$,
                        also $(c^{t+1}_i)_f = \delta_{C_{exp}}(?, (?, \perp), (?, 0))_f = 1 = \lfloor \frac{i+(t+1)}{2^{1-i}} \rfloor \; \mathrm{mod} \; 2$.
                    \item[Fall 2)] $(c^t_i)_f \neq \perp$.
                        \begin{itemize}
                            \item[Fall 2.1)] $(c^t_{i+1})_e = 0$.
                                Wegen 4) gilt dann $(i + 1) + t \not\in (2^{2-(i+1)})\N$.
                                Nach Definition von $\delta_{C_{exp}}$ gilt $(c^{t+1}_i)_f = (c^t_i)_f$.
                                
                                Angenommen $
                                (c^{t+1}_i)_f
                                = (c^t_i)_f
                                = \lfloor \frac{i+t}{2^{1-i}} \rfloor \; \mathrm{mod} \; 2
                                \neq \lfloor \frac{i+(t+1)}{2^{1-i}} \rfloor \; \mathrm{mod} \; 2
                                $.
                                Dann ergibt sich durch $i+(t+1) \in 2^{1-i}\N$ ein Widerspruch.
                            
                            \item[Fall 2.2)] $(c^t_{i+1})_e = 1$.
                                Wegen 4) ist dann $(i+1)+t \in 2^{2-(i+1)}\N$.
                                Nach Definition gilt $(c^{t+1}_i)_f = 1 - (c^t_i)_f$.
                                
                                Es bleibt also zu zeigen, dass
                                $\lfloor \frac{i+(t+1)}{2^{1-i}} \rfloor \; \mathrm{mod} \; 2
                                \neq \lfloor \frac{i+t}{2^{1-i}} \rfloor \; \mathrm{mod} \; 2$.
                                
                                Diese Ungleichung gilt wegen $i+1+t \in 2^{1-i}\N$.
                        
                        \end{itemize}
                \end{itemize}
                
                
                
            
        \end{itemize}
    \end{proof}
    \begin{lemma}
        \label{lemmaCExpProp2}
        Für $i < 2$ und $t < 3*2^{2-i} - i$ gilt: 
        \[
            (\Delta_{C_{exp}}^t(c_{exp})_i)_e = 1 \Leftrightarrow \exists k \in \Nz: i + t = 2^k
        \]
    \end{lemma}
    \begin{proof}
        Folgt direkt aus \cref{lemmaCExpDetail}.
    \end{proof}
    
    
    \begin{lemma}
        \label{lemmaNumberTheoryInequality}
        Sei $t, i \in \Nz$ mit $i \geq 1$ und $t \geq 2i - 2$.
        Dann $i - 2 < t < 3 * 2 ^ {2+t-i}-i$.
    \end{lemma}
    \begin{proof}
        Die erste Ungleichung ist trivial.
        Es gilt $1 \leq i \leq \frac{t}{2} + 1$.
        Wenn die zweite Ungleichung nicht gilt, gilt sie insbesondere für solche $i$ nicht, 
        die für ein festes $t$ maximal gewählt werden und damit für $i = \frac{t}{2} + 1$.
        Es bleibt zu zeigen, dass $\frac{3}{2}t + 1 < 6 * 2^{\frac{t}{2}}$.
        Diese Ungleichung gilt offensichtlich für $t = 0$ und $t = 1$ und kann
        leicht induktiv mit Schrittweite 2 auf alle natürlichen Zahlen fortgesetzt werden.
    \end{proof}
    
    \begin{satz}
        Es gibt einen Zellularautomaten $C$,
        sodass für $i \in \Z$ und $t \in \Nz$ mit $i - 2 < t < 3*2^{2-i+t}-i$ gilt:
        \[
            (\Delta^{t}_{C}(c_{exp})_i)_e = 1 \Leftrightarrow \exists k \in \Nz: i + t = 2^k
        \]
        Wegen \cref{lemmaNumberTheoryInequality} gilt die Aussage insbesondere,
        falls $i \geq 1$ und $t \geq 2i - 2$.
    \end{satz}
    \begin{proof}
        Nach \cref{linksunabhaengigSpeedup} gibt es einen Zellularautomaten $C$,
        sodass $\Delta^t_{C}(c_{exp})_i = \Delta^{2t}_{C_{exp}}(c_{exp})_{i-t}$.
        Aus den Voraussetzungen folgt, dass $i-t < 2$ und $2t < 3 * 2^{2-(i-t)}-(i-t)$.
        Es gilt damit mit \cref{lemmaCExpProp2}:
        \[
            i + t = (i-t) + 2t = 2^k
            \Leftrightarrow(\Delta^{2t}_{C_{exp}}(c_{exp})_{i-t})_e = 1
            \Leftrightarrow (\Delta^t_{C}(c_{exp})_{i})_e = 1
        \]        
        
    \end{proof}
\end{comment}


Es hat sich herausgestellt, dass die im folgenden definierte Eigenschaft auf viele interessante Advices zutrifft.
Wird sie als Annahme gefordert, können ohne Beschränkung der Allgemeinheit weitere Eigenschaften angenommen werden - 
dies zeigt \cref{wrtRealtimeLengthIdealAdvice}.

\begin{definition}[Präfixstabil]
    Ein Advice $\advA$ heißt präfixstabil, wenn für alle $w, s \in \Sigma^*$ gilt:
    \[
        \advA(w) =
        \set{ v[1, ..., |w|] } { v \in \advA(ws) }
    \]
\end{definition}

\begin{lemma}
    \label{wrtRealtimeLengthIdealAdvice}
    
    Sei $\advA$ ein präfixstabiler Advice, $k \in \N$ und $L \in \CAL{\CART, \CALeft, \orakel({\advA})}$.
    Dann gilt für $L(k)$ aus \cref{wrtRealtimeLengthIdeal}:
    \[
        L(k) \in \CAL{\CART, \CALeft, \orakel({\advA})}
    \]
    
    Sei $\mathcal{M}_k$ die Menge alle Sprachen, deren Wortlängen Vielfache von k sind.
    Es gilt dann folgende Implikation:
    \begin{align*}
        \CAL{\CART, \CALeft, \orakel({\advA})} \cap \mathcal{M}_k  & = \CAL{\CART, \CALeft} \cap \mathcal{M}_k \\
        \Rightarrow
        \CAL{\CART, \CALeft, \orakel({\advA})} & = \CAL{\CART, \CALeft}
    \end{align*}
\end{lemma}
\begin{proof}
    Analog zum Beweis von \cref{wrtRealtimeLengthIdeal}:
    Da $\advA$ präfixstabil ist, können Zeichen am Ende des Wortes gelöscht werden, ohne dass sich
    der Advice auf dem übrig gebliebenen Teil des Wortes ändert.
    
    Ist nun $L \in \CAL{\CART, \CALeft, \orakel({\advA})}$,
    gilt $L(k) \in \CAL{\CART, \CALeft, \orakel({\advA})} \cap \mathcal{M}_k$
    und damit auch
    $L(k) \in \CAL{\CART, \CALeft} \cap \mathcal{M}_k$.
    Mit \cref{wrtRealtimeLengthIdeal} folgt dann $L \in \CAL{\CART, \CALeft}$.
\end{proof}

Das folgende \cref{thmAdviceMain} ist eine der zentralen Erkenntnisse dieser Arbeit.
Es macht eine Aussage über die Echtzeit-Verträglichkeit von Advices,
die sich auf bestimmte Art durch beliebige Zellularautomaten ergeben.
\cref{folgerungThmMainAdv} zeigt, wie mit diesem Theorem Echtzeit-Zellularautomaten
in gewisser Hinsicht hintereinandergeschaltet werden können.

Zum Beweis werden Sätze aus allen vorangegangen Kapitel verwendet.

\begin{theorem}
    \label{thmAdviceMain}
    Sei $H = (Q, \delta) \in \CA$ ein Zellularautomat mit $\Sigma \subseteq Q$.
    Der folgende Advice $\advA_H$ ist $\CAM{\CART, \CALeft}$-verträglich:
    \[
        \advA_H(w) := v \in Q^{|w|} \text{ mit } v_i := \Delta_A^{i-1}([w])_1 \\
    \]
\end{theorem}
\begin{proof}
    Die Inklusion \enquote{$\subseteq$} der $\CAM{\CART, \CALeft}$-Verträglichkeit folgt durch ~\cref{lemmaIgnoriereAdvice}.
    Sei nun $L \in \CAL{\CART, \CALeft, \orakel({\advA_H})}$.
    Da $\advA_H$ präfixstabil ist, kann wegen \cref{wrtRealtimeLengthIdealAdvice} und \cref{endlichVieleAusnahmen}
    \acs{OBdA.} angenommen werden,
    dass die Wortlängen von $L$ alle Vielfache von $3$ sind und dass $L$ nicht das leere Wort enthält.

    Es existiert also ein Echtzeit-Zellularautomat $C$ mit $L_{\advA, C} = L$. Nach \cref{satzRauteTot} kann der Rand von $C$ tot gewählt werden.
    Sei $w \in \Sigma^*$ mit $|w| \in 3\N$, $c := [w]$, $v := \advA_H(w)$ und $c' := [\joinw(\Spvek{ w; v})]$.
    Nach \cref{CAgfSpeedup} gibt es einen Automaten $C_1$ und eine Funktion $f_1$, sodass für $i \in \N$ gilt:
    \[
        f_1(\Delta_{C_2}^{2i+1}(c)_i) = (\Delta_C^{3i-3}(c)_1, \Delta_C^{3i-2}(c)_1, \Delta_C^{3i-1}(c)_1)
    \]
    
    Ferner lässt sich leicht ein Automat $K$ konstruieren und eine Funktion $f_2$ angeben, sodass für $i \geq 1$ gilt:
    \[
        f_2(\Delta_{K}^{2i+1}(c)_i) = (c_{3i-2}, c_{3i-1}, c_{3i})
    \]
    
    Indem $C_1$ und $K$ gleichzeitig ausgeführt werden, lässt sich ein Automat $M$ konstruieren und eine Funktion $f$ angeben,
    sodass für $n := \frac{|w|}{3}$ gilt:
    \[
        f(\Delta_M^t(c)_i) =
        \begin{cases}
            \mathrm{reset} & \text{falls } t = 0 \text{ und } i \in \finset{1, .., n}, \\
            \mathrm{set}(\#\#\#) & \text{falls } t = 0 \text{ und } i \not\in \finset{1, .., n}, \\
            \mathrm{set}(s_i)
                & \text{falls } i \in \finset{1, ..., n} \text{ und } t = 2 * i + 1, \\
            \mathrm{step} & \text{sonst.}
        \end{cases}
    \]
    Wobei $s_i$ für $i \in \Z$ wie folgt definiert ist:
    \[
        s_i := \begin{cases}
                    \Spvek{ c_{3i-2} ; \Delta_C^{3i-3}(c)_1 }
                        \Spvek{ c_{3i-1} ; \Delta_C^{3i-2}(c)_1} 
                        \Spvek{ c_{3i} ; \Delta_C^{3i-1}(c)_1}
                        \in (\Sigma \times Q_H)^3
                        & \text{falls } i \in \finset{ 1, ..., n } \\
                    \#\#\# \in Q_C & \text{sonst} 
                \end{cases}
    \]
    
    Damit gilt aber gerade $s = S_3(c')$ wobei $S_3(c')$ die komprimierte $c'$-Konfiguration aus \cref{factorSpeedupConstruction} ist.
    Setze $S := S_3(C)$ auf die Speedup-Konstuktion, die drei Schritte in einem simuliert.
    Setze nun $A := A(M, f, S, \finset{\#})$ aus \cref{erweiterteNakamuraKonstruktion}.
    Dann gilt nach \cref{timeNakamuraConstruction}:
    \[
        (\Delta^{ |w| }_A(c)_1)_a = \Delta_S^{n - 1}(s)
    \]
    Da $\#$ in $C$ und damit auch $\#\#\#$ in $S$ tot ist, gilt damit nach \cref{factorSpeedupConstructionCorrectness}:
    \[
        \gamma_\#((\Delta^{ |w| }_A(c)_1)_a) = \gamma_\#(\Delta_S^{n-1}(s)) = \Delta_C^{|w| - 1}(c')
    \]
    
    Mit entsprechender Wahl der akzeptierenden Zustände und einem konstanten Speedup von einem Schritt
    durch \cref{satzEchtzeitSpeedup} folgt damit $L \in \CAL{\CART, \CALeft}$.
\end{proof}

\begin{corollary}
    \label{folgerungThmMainAdv}
    Sei $L \in \CAL{\CART, \CALeft}$.
    Der folgende Advice $\advA_L$ ist $\CAM{\CART, \CALeft}$-verträglich:
    \[
        \advA_L(w) := v \in \B^{|w|} \text{ mit } v_i :=
        \begin{cases}
            1 & \text{ falls } w_{[1...i]} \in L \\
            0 & \text{ sonst }
        \end{cases}
    \]
\end{corollary}
\begin{proof}
    Nach Wahl von $L$ gibt es einen Zellularautomaten $C \in \CAM{\CART, \CALeft}$
    mit $L_C = L$. Damit gilt für ein $w \in \Sigma_C^*$ und alle $t \in \finset{0, ..., |w|-1}$:
    \[
        \Delta_C^{t}([w])_1 \in
        \begin{cases}
            F_C^+ & \text{falls } w_{[1...t+1]} \in L \\
            Q \setminus F_C^+ & \text {sonst}
        \end{cases}
    \]
    Indem Zustände aus $F_C^+$ als $1$ und Zustände aus $Q \setminus F_C^+$ als $0$ aufgefasst werden, 
    folgt mit \cref{thmAdviceMain} die Behauptung.
\end{proof}

Wenn $L_1$ also eine Echtzeit-Sprache ist und mit $\advA_{L_1}$ gezeigt werden kann,
dass $L_2$ eine Echtzeit-Sprache ist, dann ist auch der Advice $\advA_{L_2}$ Echtzeit-verträglich
und kann beispielsweise dazu benutzt werden, um zu zeigen, dass eine Sprache $L_3$ von einem Echtzeit-Zellularautomaten erkannt werden kann!

Die nächste Frage, die sich auftut, ist, ob es interessante Echtzeit-verträgliche Advices gibt, die nicht präfixstabil sind.
Diese Advices können offensichtlich nicht durch \cref{thmAdviceMain} konstruiert werden, da $\advA_H$ für jede Wahl von $H$ präfixstabil ist.
$\advA_2$ aus \cref{bestimmteEinfacheAdvices} ist tatsächlich nicht präfixstabil und Echtzeit-verträglich, aber auch nicht besonders
spannend.
Der folgende Satz zeigt beschreibt ein \enquote{spannendes}, Echtzeit-verträgliches und nicht-präfixstabiles Advice.
Tatsächlich ist der im folgenden beschriebene Advice der Grund, warum die erweiterte Nakamura-Konstruktion aus \cref{chap:ErweiterteNakamuraKonstr} die Rücksetz-Funktionalität besitzt.

\begin{satz}
    \label{expMidAdv}
    Der folgende Advice $\advA$ ist $\CAM{\CART, \CALeft}$-verträglich:
    \[
        \advA(w) := \mathrm{ones}(2^{\lfloor \log_2{|w|} \rfloor - 1}, |w|) \in \B^*
    \]
\end{satz}
\begin{proof}
    Wegen \cref{folgerungThmMainAdv} kann angenommen werden,
    dass die Zellen des Eingabewortes, deren Position eine Zweierpotenz ist, markiert sind (in \cref{fig:ExpMidAdvice} durch rote Vierecke zu sehen).
    Die Idee ist nun, diese markierten Zellen als Signal für eine neue Mitte zu verwenden.
    Offensichtlich müssen frühere Berechnungen, die von anderen \enquote{Mitten} ausgegangen sind,
    zurückgesetzt und neu gestartet werden.
    Um für die Simulation genügend Zeit zu haben, wird dazu der Ausgangsautomat nun unabhängig zwei Mal mithilfe der in \cref{chap:ErweiterteNakamuraKonstr} vorgestellten Konstruktion simuliert.
    In der Abbildung \cref{fig:ExpMidAdvice} werden die Setz-Befehle
    für die eine Simulation durch grüne Ovale und für die andere durch blaue dargestellt.
    In allen anderen Zuständen wird jeweils versucht, einen Simulationsschritt durchzuführen.
    Die markierten Zellen setzen dann jeweils abwechselnd den einen oder den anderen simulierten Automaten
    zurück und markieren jeweils eine Mitte mit größerer Position.
    Die erste Zelle verwendet für das Akzeptanzverhalten dann jeweils die eine oder die andere Simulation.
    Da sich die Abstände jeweils verdoppeln, reicht die Zeit aus.
    
    \begin{figure}[h!]
        \begin{center}
            \newcommand{\setq}{\scriptsize $\mathrm{set}$}
            \includesvg[width=310pt]{images/ExpSimulation}
        \end{center}
        \caption{Visualisierung der Ausführung}
        \label{fig:ExpMidAdvice}
    \end{figure}
\end{proof}

Auf eine formale Beschreibung der Konstruktion \acs{bzw.} einen formalen Beweis der Korrektheit wurde
hier aus Verständlichkeits-, Umfangs- und Zeitgründen verzichtet. Da viele Überlegungen in \cref{expMidAdv} gesteckt wurden
und dies insbesondere die Rücksetz-Funktionalität der erweiterten Nakamura-Konstruktion begründet hat,
wurde der Satz trotzdem aufgeführt.

\section{Ausblick}

Folgende Advices wurden außerdem noch auf Echtzeit-Verträglichkeit untersucht:
\begin{itemize}
    \item $\advA_1(w) := v \in \B^{|w|}$ mit $v_i = \chr{1} \Leftrightarrow i = \lfloor \frac{|w|}{2} \rfloor$
    \item $\advA_2(w) := v \in \B^{|w|}$ mit $v \in \chr{0}^*\chr{1}\chr{0}^*$
    \item $\advA_3(w) := \B^{|w|}$
\end{itemize}
Wie diese Advices Echtzeit-Zellularautomaten erweitern, ist nicht klar geworden.
$\advA_3$ hilft Echtzeit-Zellularautomaten sogar \acs{ggf.} NP-schwere Probleme zu lösen.
Wenn $\advA_3$ Echtzeit-verträglich ist, so ist es auch $\advA_2$.
Wenn $\advA_2$ Echtzeit-verträglich ist, so ist es auch $\advA_1$.

Da bei $\advA_1$ Echtzeit mit Linearzeit für unäre Sprachen zusammenfällt und die Markierung
der Mitte ein äußerst schwieriges Unterfangen darstellt, vermutet der Autor, dass schon $\advA_1$ nicht
Echtzeit-verträglich ist.
Interessant ist auch die Frage, ob $\advA_2$ Echtzeit-verträglich ist, wenn angenommen wird,
dass $\advA_1$ Echtzeit-verträglich ist, \acs{bzw.} allgemeiner, ob unter dieser Annahme
Realzeit mit Linearzeit zusammenfällt.