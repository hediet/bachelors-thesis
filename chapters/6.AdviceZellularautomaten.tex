\chapter{Advice-Zellularautomaten}
\label{chap:AdvAuto}

Dieses Kapitel beschäftigt sich damit, inwiefern die Mächtigkeit bestimmter Klassen von Zellularautomaten verändert wird,
wenn der Automat zusätzlich zum Eingabewort ein vom Eingabewort abhängiges Unterstützungswort (im Folgenden Advice genannt) erhält.
Insbesondere wird gezeigt, dass bestimmte solcher Klassen unter Zuhilfenahme ausgewählter Advices abgeschlossen sind.
Solche Advices heißen dann verträglich bezüglich dieser Klasse.
Diese Abschlusseigenschaft ist nützlich, um die Konstruktion einiger Realzeit-Zellularautomaten zu vereinfachen.

\section{Definition}

\begin{definition}
    Ein $\Sigma$-$\Gamma$-Advice (Hinweis) $\mathcal{A}$ ist eine Abbildung
        $\mathcal{A}: \Sigma^* \to \Pot(\Gamma^*)$ mit $\forall w \in \Sigma^*: \mathcal{A}(w) \subseteq \Sigma^{|w|}$.
    Wenn $\forall w \in \Sigma^*: |\mathcal{A}(w)| = 1$, kann $\mathcal{A}$ auch
    als Abbildung $\mathcal{A} : \Sigma^* \to \Gamma^*$ aufgefasst werden.
\end{definition}

\begin{definition}
    Definiere injektive Funktion $\comp: (\Sigma \times \Gamma)^* \to (\Sigma^* \times \Gamma^*)$
    um ein Wort über zwei Alphabete in zwei Wörter über je ein Alphabet zu zerlegen:
    \[
        \forall v \in \Sigma^*, w \in \Gamma^{|v|}: \comp(\Spvek{v_1; w_1}...\Spvek{v_{|v|}; w_{|v|}}) := \Spvek{v; w}
    \]
    
    Definiere $\comb: \comp((\Sigma \times \Gamma)^*) \to (\Sigma \times \Gamma)^*$,
    um zwei Wörter über je ein Alphabet zu einem Wort über zwei Alphabete zusammenzufügen:
    \[
        \comb(w) := \comp^{-1}(w)
    \]
    
\end{definition}

\begin{exmp}
    \[
        \comp(\Spvek{\chr{1}; \chr{a}}\Spvek{\chr{2}; \chr{b}}\Spvek{\chr{3}; \chr{c}}) = \Spvek{\chr{123}; \chr{abc}}
    \]
    
    \[
        \comb(\Spvek{\chr{123}; \chr{abc}}) = \Spvek{\chr{1}; \chr{a}}\Spvek{\chr{2}; \chr{b}}\Spvek{\chr{3}; \chr{c}}
    \]        
\end{exmp}

\begin{definition}
    Es sei ein Advice $\mathcal{A}$ und ein Zellularautomat $C \in \ECA$ gegeben. Mit Unterstützung des Advices $\mathcal{A}$ kann $C$ nun folgende Sprache erkennen:
    \[
         L_{\mathcal{A}, C} := \set{ w \in \Sigma^*}{\exists v \in \mathcal{A}(w): \comb(\Spvek{w; v}) \in L(C)}
    \]
    
    Für $M(\mathcal{A}) := \set{ \comb(\Spvek{ w; v}) }
        { w \in \Sigma^* \land v \in \mathcal{A}(w) }$
    kann $L_{\mathcal{A}, C}$ auch wie folgt aufgefasst werden:
    \[
        L_{\mathcal{A}, C} = \comp(
            L(C) \cap M(\mathcal{A})
        )_1
    \]
    
    Folgender $\ECA$-Funktor stellt einer Klasse von Zellularautomaten ein gegebenes Advice zur Verfügung:
    \[
        \orakel(\mathcal{A})(M \subseteq \ECA) := \set{ (Q_C, \delta_C, \Sigma, \#_C, L_{\mathcal{A}, C}, (S_C, C))} { C \in M, \; \Sigma_C = \Sigma \times \Gamma }
    \]
\end{definition}

\section{Ergebnisse}

\begin{satz}
    Seien $\mathcal{A}_1$ und $\mathcal{A}_2$ zwei Advices.
    Definiere $(\mathcal{A}_1 \cup \mathcal{A}_2)(w) := \mathcal{A}_1(w) \cup \mathcal{A}_2(w)$ und 
    $(\mathcal{A}_1 \cap \mathcal{A}_2)(w) := \mathcal{A}_1(w) \cap \mathcal{A}_2(w)$.
    
    Sind $\mathcal{A}_1$ und $\mathcal{A}_2$ $\CAM{\CART, \CALeft}$-verträglich, so sind
    auch $\mathcal{A}_1 \cup \mathcal{A}_2$ und $\mathcal{A}_1 \cap \mathcal{A}_2$
    $\CAM{\CART, \CALeft}$-verträglich.
\end{satz}
\begin{proof}
    Es werden entsprechend zwei Ausführungen simuliert - je eine auf einem 
    
\end{proof}

\begin{satz}
    Sei $(\Sigma \cup \finset{\Box}) ^2 \subseteq \Gamma, \Box \not\in \Sigma$.
    Definiere den \enquote{Komprimierungs}-Advice $\mathcal{A}_K$:
    \[
        \mathcal{A}_K(w) := 
                  \Spvek{w_1; w_2} \Spvek{w_3; w_4} ... \Spvek{w_{|w-1|}; w_{|w|}}
                        \; \Spvek{\Box; \Box}^\frac{|w|}{2}
    \]
    Falls $|w|$ ungerade, setze $w := w\Box$.
    Unter dem Advice $\mathcal{A}_K$ fällt Realzeit mit Linearzeit zusammen:
    \[
        \CAL{\CART, \CALeft, \orakel({\mathcal{A}})} = \CAL{\CALT, \CALeft}
    \]
\end{satz}
\begin{proof}
    Die Inklusion \enquote{$\subseteq$} ist klar.
    Sei $L \in \CAL{\CALT, \CALeft}$.
    Nach \cref{linSpeedup} existiert ein Zellularautomat $C$, der $L$ in $2n$ Schritten erkennt.
    Der Automat $C' := S_2(C)$ aus \cref{factorSpeedupConstruction}
    erkennt $L$ dann in $n$ Schritten auf der komprimierten Konfiguration $S_2([w])$.
    Da die komprimierte Konfiguration durch das Advice bereitgestellt wird, kann $L$ mithilfe des Advices in Echtzeit erkannt werden.
\end{proof}


\begin{comment}
    \begin{satz}
        Sei $(\Sigma \cup \finset{\Box}) ^2 \subseteq \Gamma, \Box \not\in \Sigma$.
        
        \begin{enumerate}
            \item  $\mathcal{A}_1(w) := \Gamma^{|w|}$
        \end{enumerate}
        
        Es gilt jeweils:
        \[
            \CAL{\CART, \orakel({\mathcal{A}_i})} = \CAL{\CALT, \orakel({\mathcal{A}_i})}
        \]
        
    \end{satz}

    \begin{definition}
        Seien $\mathcal{A}_1$ und $\mathcal{A}_2$ zwei Orakel, $M \subseteq \ECA$ (falls nicht angegeben, $M := \ECA^\CART$).
        \[
            \mathcal{A}_1 \leq_M \mathcal{A}_2 \; :\Leftrightarrow \; \mathcal{L}(M^{\orakel(\mathcal{A}_1)}) \subseteq \mathcal{L}(M^{\orakel(\mathcal{A}_2)})
        \]        
    \end{definition}
    
    \begin{lemma}
        Es gibt kein maximales Orakel.
    \end{lemma}
    
\end{comment}

\begin{satz}
    Sei $\mathcal{A}$ ein Advice mit $\CAL{\CART, \CALeft, \orakel({\mathcal{A}})} \neq \CAL{\CALT, \CALeft, \orakel({\mathcal{A}})}$.
    Dann $\CAL{\CART, \CALeft} \neq \CAL{\CALT, \CALeft}$.
\end{satz}
\begin{proof}
    Angenommen $\CAL{\CART, \CALeft, \orakel({\mathcal{A}})} \neq \CAL{\CALT, \CALeft, \orakel({\mathcal{A}})}$.
    
    Wegen $\CAL{\CART, \CALeft, \orakel({\mathcal{A}})} \subseteq \CAL{\CALT, \CALeft, \orakel({\mathcal{A}})}$
    gibt es laut Annahme ein $L \in \CAL{\CALT, \CALeft, \orakel({\mathcal{A}})} \setminus \CAL{\CART, \CALeft, \orakel({\mathcal{A}})}$.
    Dann gibt es einen Automaten $C \in \CAM{\CALT, \CALeft}$ mit
    
    \[
        L = L_{\mathcal{A}, C} = \set{ w \in \Sigma^*}{\exists v \in \mathcal{A}(w): \comb(\Spvek{w; v}) \in L(C)}
    \]
    
    Angenommen, es gäbe einen Automaten $C' \in \CAM{\CART, \CALeft}$ mit $L_{C'} = L_C$.
    Dann $L_{\mathcal{A}, C} = \set{ w \in \Sigma^*}{\exists v \in \mathcal{A}(w): \comb(\Spvek{w; v}) \in L(C')} = L_{\mathcal{A}, C'}$ und damit
    $L \in \CAL{\CART, \CALeft, \orakel({\mathcal{A}})}$.
    Widerspruch zur Annahme! Also $L(C) \in \CAL{\CALT, \CALeft} \setminus \CAL{\CART, \CALeft}$.
\end{proof}


\section{\texorpdfstring{$\CAM{\CART, \CALeft}$}{CA\^RT-L}-verträgliche Advices}

\begin{definition}[$\mathcal{A}$ ist $M$-verträglich]
    Sei $M \subseteq \ECA$ eine Menge von Zellularautomaten und $\mathcal{A}$ ein Advice.
    $\mathcal{A}$ heißt $M$-verträglich, falls gilt:
    \[
        \mathcal{L}(M) = \mathcal{L}(M^\mathcal{A})
    \]
\end{definition}

\begin{satz}
    \label{lemmaIgnoriereAdvice}
    Sei ein Advice $\mathcal{A}$ und eine Menge 
    $F \subseteq {\Nz}^{\Nz}$ von Funktionen gegeben.
    Dann gilt: $\mathcal{L}(\CAM{ \mathrm{time}(F), \CALeft }) \subseteq  \mathcal{L}(\CAM{ \mathrm{time}(F), \CALeft, \orakel(\mathcal{A})})$.
\end{satz}
\begin{proof}
    Der Advice kann ignoriert werden, indem $\Sigma \times \Gamma$ auf $\Sigma$ reduziert wird.
\end{proof}







\begin{definition}
    Seien $k, n \in \Nz$ zwei Zahlen. Die Funktion $\mathrm{ones}$ gibt ein Wort der Länge $n$ zurück, sodass die $k$ ersten Zeichen Einsen sind.
    \[
        \mathrm{ones}(k, n) := (\chr{1}^k \chr{0}^n)[1..n]
    \]
\end{definition}

\begin{exmp}
    \[
        \mathrm{ones}(2, 5) = \chr{11000}
    \]
    \[
        \mathrm{ones}(6, 5) = \chr{11111}
    \]        
\end{exmp}

\begin{definition}
    Sei $\Gamma$ ein endliches Alphabet.
    Ein $F$-haltender Zellularautomat $C \in \ECA^\CAF$ berechnet eine Funktion $f: \Sigma^* \to \Gamma^*$, wenn
    $\Gamma \subseteq Q_C$, $\#_C \not\in \Gamma$ und 
    \[
        \forall w \in \Sigma^*: \Delta^{\ctime_C(w)}_{C}([w]) = [f(w)]
    \]
    gilt.
\end{definition}

\begin{satz}
    \label{lemmaEinfachesOrakel}
    Wenn $\mathcal{A}: \Sigma^* \to \Gamma^*$ eine Funktion ist mit $\forall w \in \Sigma^*: |f(w)| = |w|$,
    die von einem $F$-haltenden Zellularautomaten $B$ mit
    $\ctime_C(\cdot) = k$ und $k \in \Nz$ berechnet wird,
    dann ist $\mathcal{A}$ $\CAM{\CART, \CALeft}$-verträglich.
\end{satz}
\begin{proof}
    Die Inklusion \enquote{$\subseteq$} der $\CAM{\CART, \CALeft}$-Verträglichkeit folgt durch ~\cref{lemmaIgnoriereAdvice}.
    
    Sei nun $L \in \CAL{\CART, \CALeft, \orakel({\mathcal{A}})}$, $\bar{C} \in \CAM{\CART, \CALeft, \orakel({\mathcal{A}})}$ sodass $L_{\bar{C}} = L$.
    Sei $C := C_{\bar{C}}$ der zugrundeliegende $\Sigma \times \Gamma$ erkennende Zellularautomat,
    der mithilfe des Advices $\mathcal{A}$ die Sprache $L$ in Realzeit erkennt.
    
    Es gibt einen Zellularautomaten $I$, der in $k$ Schritten die Identität berechnet. Folglich gibt es einen Zellularautomaten $A$, der die Funktion $f: \Sigma^* \to (\Sigma \times \Gamma)^*$ mit
    $f(w) = \comb(\Spvek{w; f(w)})$ in $k$ Schritten berechnet, in dem $I$ und $B$ für $k$ Schritte parallel ausgeführt werden.
    
    Konstruiere den Automaten $C'$ durch Hintereinanderausführung der Automaten $A$ und $C$.
    Für ein Wort $w \in \Sigma^*$ braucht der Automat $C'$ $k + |w|$ viele Schritte und es gilt $L_{C'} = L$.
    
    Nach \cref{satzRealzeitSpeedup} gibt es nun einen Automaten $C'' \in \CAM{\CART, \CALeft}$
    mit $L_{C''} = L_{C'}$.
    Also $L \in \CAL{\CART, \CALeft}$.
\end{proof}

\begin{corollary}
    Sei $\Gamma = \B$, $k \in \Nz$.
    Folgende Advices sind $\CAM{\CART, \CALeft}$-verträglich:
    \begin{enumerate}
        \item $\mathcal{A}_1(w) := \mathrm{ones}(k, |w|)$ 
        \item $\mathcal{A}_2(w) := \mathrm{ones}(k, |w|)^{\mathrm{Rev}}$
    \end{enumerate}
\end{corollary}
\begin{proof}
    Die Funktionen $\mathcal{A}_i$ können offensichtlich
    von einem $F$-haltenden Zellularautomat in $k$ Schritten berechnet werden.
    Mit ~\cref{lemmaEinfachesOrakel} folgt die Behauptung.
\end{proof}

\begin{comment}
        $Q' := (Q_C \times \finset{ 0, \dots, c })$. Identifiziere $\Sigma' := \Sigma_C$ mit $(\finset{\chr{0}} \times \Sigma_C) \times \finset{0} $. Setze $\#' := (\#_C, 0)$.
        
        $\delta'(\Spvek{\#_C; k_a}, \Spvek{\Spvek{ \gamma_b; c_b}; k_b}, c) := \Spvek{ \Spvek{\chr{1}; c_b}; k_b + 1}$
        
        $\delta'(\Spvek{\Spvek{\gamma_a; c_a}; k_a}, \Spvek{\Spvek{\gamma_b; c_b}; k_b}, c)
            := \Spvek{ \Spvek{ \max \finset{\gamma_a, \gamma_b}; c_b}; k_b + 1}$
        
        $\delta'(a, \Spvek{q_b; k_b}, c) := \Spvek{q_b; k_b + 1}$
\end{comment}



\begin{comment}
    \begin{satz}
        Für $\mathcal{A}(w) := a_1...a_{|w|} \in \B^*$ mit \[
           a_i :=
           \begin{cases}
             \chr{1} & \text{falls } \exists j \in \Nz: i = 2^j \\
             \chr{0} & \text{sonst}
           \end{cases}
        \]
        gilt $\CAL{\CART, \orakel({\mathcal{A}})} = \CAL{\CART}$.
    \end{satz}
    
    
    
    \begin{definition}
        Setze $C_{exp} := (Q, \delta)$ mit $Q := \B \times (\B \cup \finset{ \perp })$.
        $Q$ setzt sich aus zwei Komponenten zusammen.
        Die erste Komponente eines Zustands zu einem Zeitpunkt $t$ an der Position $i$ signalisiert, ob $i+t$ eine Zweierpotenz sein kann.
        In der Abbildung wird dies durch die Farbe blau dargestellt.
        Die zweite Komponente bestimmt den Zustand des Filters. In der Abbildung kennzeichnet Grün einen geöffneten Filterzustand (1), Rot einen geschlossenen (0)
        und keine Markierung einen nicht initialisierten Filter ($\perp$).
        
        Die Zustandsübergangsfunktion $\delta$ ist dann wie folgt definiert:
        \[
            \delta(\cdot, (e, f), (e_R, f_R)) :=
            \begin{cases}
                (e_R, \perp) & \text{ falls } f = \perp \text{ und } f_R \neq 0 \\
                (e_R, 1) & \text{ falls } f = \perp \text{ und } f_R = 0 \\
                (0, f) & \text{ falls } f \neq \perp \text{ und } e_R = 0 \\
                (f, 1 - f) & \text{ falls } f \neq \perp \text{ und } e_R = 1
            \end{cases}
        \]
        
        Für eine Start-Konfiguration
        ${c_{exp}}_i = \begin{cases} (0, \perp) & \text{ falls } i < 1  \\ (1, 1) & \text{ falls } i = 1 \\ (1, \perp) & \text{ falls } i > 1 \end{cases}$
        ergibt sich folgendes Zustands-Zeit-Diagramm:
        
        \includegraphics[scale=.5]{exp1}
        
        Offensichtlich ist $\delta$ links-unabhängig.
        
        Nach Anwendung von \cref{linksunabhaengigSpeedup} ergibt sich folgendes Zustands-Zeit-Diagramm:
        \includegraphics[scale=.5]{exp2}
    \end{definition}
    
    
    \begin{lemma}
        \label{lemmaCExpDetail}
        Setze $c^t_i := \Delta^t_{C_{exp}}(c_{exp})_i$.
        Es bezeichne $(c^t_i)_e$ die erste Komponente 
        und $(c^t_i)_f$ die zweite Komponente des Tupels $c^t_i$.
        Es gilt:
        \begin{itemize}
            \item[1)] $(c^t_i)_e = 1$, falls $i \geq 2$.
            \item[2)] $(c^t_i)_e = 1$, falls $i+t \in 2^{2-i}\N$.
            \item[3)] $(c^t_i)_e = 1$, falls $i+t \in \set{2^k}{0 \leq k \leq 1-i}$.
            \item[4)] $(c^t_i)_e = 0$, falls $i < 2$ und $i+t \not\in 2^{2-i}\N \cup \set{2^k}{0 \leq k \leq 1-i}$.
            \item[5)] $(c^t_i)_f = \perp$, falls $i \geq 2$.
            \item[6)] $(c^t_i)_f = \perp$, falls $i + t < 2^{1-i}$.
            \item[7)] $(c^t_i)_f = \lfloor \frac{i+t}{2^{1-i}} \rfloor \; \mathrm{mod} \; 2$, falls $i < 2$ und $i + t \geq 2^{1-i}$.
        \end{itemize}
        
        
        \[
            (c^t_i)_e = \begin{cases}
                1 & \text{falls } 
            i \geq 2 \text{ oder } i+t \in (2^{2-i})\N \text{ oder } i+t \in \set{2^k}{0 \leq k \leq 1-i} \\
                0 & \text{sonst}
            \end{cases}
        \]
        und
        \[
            (c^t_i)_f = \begin{cases}
                \perp & \text{ falls } i \geq 2 \text{ oder } i + t < 2^{1-i} \\
                \lfloor \frac{i+t}{2^{1-i}} \rfloor \; \mathrm{mod} \; 2 & \text{ sonst }
            \end{cases}
        \]
    \end{lemma}
    \begin{proof}
        Beweis aller Aussagen durch simultane Induktion über $t$.
        Einsetzen zeigt, dass alle Aussagen für $t = 0$ gelten.
        
        Es gelten nun alle Aussagen für ein $t \geq 0$.
        
        \begin{itemize}
            \item[1), 5)]
                Sei $i \geq 2$.
                
                Wegen 1) und 5) gilt $(c^t_{i+1})_e = 1$,
                $(c^t_{i})_f = \perp$ und $(c^t_{i+1})_f = \perp \neq 0$.
                Damit $c^{t+1}_i = \delta_{C_{exp}}(?, (?, \perp), (1, \perp)) = (1, \perp)$.
            
            \item[2)]
                Sei $i+(t+1) = 2^{2-i}k$ für ein $k \in \N$. OBdA ist $i < 2$.
                
                Wegen $i+t \geq 2^{2-i} - 1 \geq 2^{1-i}$ folgt mit 7)
                \begin{align*}
                (c^t_i)_f
                & = \lfloor \frac{i+t}{2^{1-i}} \rfloor \; \mathrm{mod} \; 2 \\
                & = \lfloor \frac{2^{2-i}k - 1}{2^{1-i}} \rfloor \; \mathrm{mod} \; 2
                = \lfloor 2k - \frac{1}{2^{1-i}} \rfloor \; \mathrm{mod} \; 2 \\
                & = 2k-1 \; \mathrm{mod} \; 2 = 1
                \end{align*}
                
                Weiter gilt wegen 2)
                $(c^t_{i+1})_e = 1$, da $(i + 1) + t \in 2^{2-i}\N$.
                Also gilt $(c^{t+1}_i)_e = \delta_{C_{exp}}(?, (?, 1), (1, ?)) = 1$.
                
            \item[3)]
                Sei nun $i + (t + 1) \in \set{2^k}{0 \leq k \leq 1-i}$. OBdA ist $i < 2$.
                
                Wegen 6) gilt $(c^t_i)_f = \perp$, da $i + t < i + t + 1 \leq 2^{1 - i}$.
                Außerdem gilt wegen 3)
                $(c^t_{i+1})_e = 1$, da $i + t + 1 \in \set{2^k}{0 \leq k \leq 1-(i+1)} \; \cup \; \finset{2^{2-(i+1)}}$.
                Damit folgt $(c^{t+1}_i)_e = \delta_{C_{exp}}(?, (?, \perp), (1, ?))_e = 1$.
                
            \item[4)]
                Sei $i < 2$, $i + (t+1) \not\in 2^{2-i}\N \cup \set{2^k}{0 \leq k \leq 1-i}$.
                
                Angenommen, $(c^{t+1}_i)_e = 1$.
                Nach Definition von $\delta_{C_{exp}}$ folgt
                $(c^t_{i+1})_e = 1$ und $(c^t_i)_f \neq 0$.
                Wegen 4) gilt $i+1 \geq 2$ oder $(i+1)+t \in 2^{2-(i+1)}\N \; \cup \; \set{2^k}{0 \leq k \leq 1-(i+1)}$.
                Folglich $i = 1$ oder $i+t+1 \in (2^{2-i}\N) + 2^{1-i}$.
                Letzteres gilt auch im Fall $i = 1$.
                Damit gibt es ein $k \in \N$, sodass $i+t = 2^{2-i}k + 2^{1-i} - 1$.
                Da insbesondere $i + t \geq 2^{2-i} + 2^{1-i} - 1 > 2^{1-i}$ gilt, ergibt sich mit 7) folgendes:
                \begin{align*}
                    (c^t_i)_f
                    & = \lfloor \frac{i+t}{2^{1-i}} \rfloor \; \mathrm{mod} \; 2 \\
                    & = \lfloor \frac{2^{2-i}k + 2^{1-i} - 1}{2^{1-i}} \rfloor \; \mathrm{mod} \; 2
                    = \lfloor 2k + 1 - \frac{1}{2^{1-i}} \rfloor \; \mathrm{mod} \; 2 \\
                    & = 2k \; \mathrm{mod} \; 2
                    = 0
                \end{align*}
                Was der Annahme widerspricht. Damit gilt $(c^{t+1}_i)_e = 0$.
            
            \item[6)]
                Sei nun $i + (t+1) < 2^{1 - i}$.
                Mit 6) folgt wegen $i + t < 2^{1 - i}$, dass $(c^t_i)_f = \perp$.
                
                Angenommen, $(i+1)+t < 2^{1-(i+1)}$. Dann gilt wegen 6) $(c^t_{i+1})_f = \perp \neq 0$.
                Andernfalls, $(i+1)+t \geq 2^{1-(i+1)}$. Dann $2^{1-(i+1)} \leq (i+1)+t < 2^{1-i}$.
                Daraus folgt $\lfloor \frac{(i+1)+t}{2^{1-(i+1)}} \rfloor = 1$.
                Auch hier gilt mit 7) dann $(c^t_{i+1})_f = 1 \neq 0$.
                
                Es folgt $(c^{t+1}_i)_f = \delta_{C_{exp}}(?, (?, \perp), (?, ? \neq 0))_f = \perp$.
            
            \item[7)]
                Sei nun $i < 2$ und $i + (t + 1) \geq 2^{1-i}$.
                \begin{itemize}
                    \item[Fall 1)] $(c^t_i)_f = \perp$.
                        Wegen 7) gilt $i + t + 1 = 2^{1-i}$.
                        Dann $(c^t_{i+1})_f = \lfloor \frac{(i+1)+t}{2^{1-(i+1)}} \rfloor \; \mathrm{mod} \; 2 = 0$,
                        also $(c^{t+1}_i)_f = \delta_{C_{exp}}(?, (?, \perp), (?, 0))_f = 1 = \lfloor \frac{i+(t+1)}{2^{1-i}} \rfloor \; \mathrm{mod} \; 2$.
                    \item[Fall 2)] $(c^t_i)_f \neq \perp$.
                        \begin{itemize}
                            \item[Fall 2.1)] $(c^t_{i+1})_e = 0$.
                                Wegen 4) gilt dann $(i + 1) + t \not\in (2^{2-(i+1)})\N$.
                                Nach Definition von $\delta_{C_{exp}}$ gilt $(c^{t+1}_i)_f = (c^t_i)_f$.
                                
                                Angenommen $
                                (c^{t+1}_i)_f
                                = (c^t_i)_f
                                = \lfloor \frac{i+t}{2^{1-i}} \rfloor \; \mathrm{mod} \; 2
                                \neq \lfloor \frac{i+(t+1)}{2^{1-i}} \rfloor \; \mathrm{mod} \; 2
                                $.
                                Dann ergibt sich durch $i+(t+1) \in 2^{1-i}\N$ ein Widerspruch.
                            
                            \item[Fall 2.2)] $(c^t_{i+1})_e = 1$.
                                Wegen 4) ist dann $(i+1)+t \in 2^{2-(i+1)}\N$.
                                Nach Definition gilt $(c^{t+1}_i)_f = 1 - (c^t_i)_f$.
                                
                                Es bleibt also zu zeigen, dass
                                $\lfloor \frac{i+(t+1)}{2^{1-i}} \rfloor \; \mathrm{mod} \; 2
                                \neq \lfloor \frac{i+t}{2^{1-i}} \rfloor \; \mathrm{mod} \; 2$.
                                
                                Diese Ungleichung gilt wegen $i+1+t \in 2^{1-i}\N$.
                        
                        \end{itemize}
                \end{itemize}
                
                
                
            
        \end{itemize}
    \end{proof}
    \begin{lemma}
        \label{lemmaCExpProp2}
        Für $i < 2$ und $t < 3*2^{2-i} - i$ gilt: 
        \[
            (\Delta_{C_{exp}}^t(c_{exp})_i)_e = 1 \Leftrightarrow \exists k \in \Nz: i + t = 2^k
        \]
    \end{lemma}
    \begin{proof}
        Folgt direkt aus \cref{lemmaCExpDetail}.
    \end{proof}
    
    
    \begin{lemma}
        \label{lemmaNumberTheoryInequality}
        Sei $t, i \in \Nz$ mit $i \geq 1$ und $t \geq 2i - 2$.
        Dann $i - 2 < t < 3 * 2 ^ {2+t-i}-i$.
    \end{lemma}
    \begin{proof}
        Die erste Ungleichung ist trivial.
        Es gilt $1 \leq i \leq \frac{t}{2} + 1$.
        Wenn die zweite Ungleichung nicht gilt, gilt sie insbesondere für solche $i$ nicht, 
        die für ein festes $t$ maximal gewählt werden und damit für $i = \frac{t}{2} + 1$.
        Es bleibt zu zeigen, dass $\frac{3}{2}t + 1 < 6 * 2^{\frac{t}{2}}$.
        Diese Ungleichung gilt offensichtlich für $t = 0$ und $t = 1$ und kann
        leicht induktiv mit Schrittweite 2 auf alle natürlichen Zahlen fortgesetzt werden.
    \end{proof}
    
    \begin{satz}
        Es gibt einen Zellularautomaten $C$,
        sodass für $i \in \Z$ und $t \in \Nz$ mit $i - 2 < t < 3*2^{2-i+t}-i$ gilt:
        \[
            (\Delta^{t}_{C}(c_{exp})_i)_e = 1 \Leftrightarrow \exists k \in \Nz: i + t = 2^k
        \]
        Wegen \cref{lemmaNumberTheoryInequality} gilt die Aussage insbesondere,
        falls $i \geq 1$ und $t \geq 2i - 2$.
    \end{satz}
    \begin{proof}
        Nach \cref{linksunabhaengigSpeedup} gibt es einen Zellularautomaten $C$,
        sodass $\Delta^t_{C}(c_{exp})_i = \Delta^{2t}_{C_{exp}}(c_{exp})_{i-t}$.
        Aus den Voraussetzungen folgt, dass $i-t < 2$ und $2t < 3 * 2^{2-(i-t)}-(i-t)$.
        Es gilt damit mit \cref{lemmaCExpProp2}:
        \[
            i + t = (i-t) + 2t = 2^k
            \Leftrightarrow(\Delta^{2t}_{C_{exp}}(c_{exp})_{i-t})_e = 1
            \Leftrightarrow (\Delta^t_{C}(c_{exp})_{i})_e = 1
        \]        
        
    \end{proof}

\end{comment}

\begin{definition}[Präfixstabil]
    Ein Advice $\mathcal{A}$ heißt präfixstabil, wenn für alle $w, s \in \Sigma^*$ gilt:
    \[
        \mathcal{A}(w) =
        \set{ v[1, ..., |w|] } { v \in \mathcal{A}(ws) }
    \]
\end{definition}

\begin{lemma}
    \label{wrtRealtimeLengthIdealAdvice}
    
    Sei $\mathcal{A}$ ein präfixstabiles Advice, $k \in \N$ und $L \in \CAL{\CART, \CALeft, \orakel({\mathcal{A}})}$.
    Dann gilt für $L(k)$ aus \cref{wrtRealtimeLengthIdeal}:
    \[
        L(k) \in \CAL{\CART, \CALeft, \orakel({\mathcal{A}})}
    \]
    
    Sei $\mathcal{M}_k$ die Menge alle Sprachen, deren Wortlängen Vielfache von k sind.
    Es gilt dann folgende Implikation:
    \begin{align*}
        \CAL{\CART, \CALeft, \orakel({\mathcal{A}})} \cap \mathcal{M}_k  & = \CAL{\CART, \CALeft} \cap \mathcal{M}_k \\
        \Rightarrow
        \CAL{\CART, \CALeft, \orakel({\mathcal{A}})} & = \CAL{\CART, \CALeft}
    \end{align*}
\end{lemma}
\begin{proof}
    Analog zum Beweis von \cref{wrtRealtimeLengthIdeal}:
    Da $\mathcal{A}$ präfixstabil ist, können Zeichen am Ende des Wortes gelöscht werden, ohne dass sich
    das Advice auf dem übrig gebliebenen Teil des Wortes ändert.
    
    Ist nun $L \in \CAL{\CART, \CALeft, \orakel({\mathcal{A}})}$,
    gilt $L(k) \in \CAL{\CART, \CALeft, \orakel({\mathcal{A}})} \cap \mathcal{M}_k$
    und damit auch
    $L(k) \in \CAL{\CART, \CALeft} \cap \mathcal{M}_k$.
    Mit \cref{wrtRealtimeLengthIdeal} folgt dann $L \in \CAL{\CART, \CALeft}$.
\end{proof}


\begin{theorem}
    \label{thmAdviceMain}
    Sei $H = (Q, \delta) \in \CA$ ein Zellularautomat mit $\Sigma \subseteq Q$.
    Der folgende Advice $\mathcal{A}_H$ ist $\CAM{\CART, \CALeft}$-verträglich:
    \[
        \mathcal{A}_H(w) := v \in Q^{|w|} \text{ mit } v_i := \Delta_A^{i-1}([w])_1 \\
    \]
\end{theorem}
\begin{proof}
    Die Inklusion \enquote{$\subseteq$} der $\CAM{\CART, \CALeft}$-Verträglichkeit folgt durch ~\cref{lemmaIgnoriereAdvice}.
    Sei nun $L \in \CAL{\CART, \CALeft, \orakel({\mathcal{A}_H})}$.
    Da $\mathcal{A}_H$ präfixstabil ist, kann wegen \cref{wrtRealtimeLengthIdealAdvice} und \cref{endlichVieleAusnahmen}
    \ac{OBdA.} angenommen werden,
    dass die Wortlängen von $L$ alle Vielfache von $3$ sind und dass $L$ nicht das leere Wort enthält.

    Es existiert also ein Echtzeit-Zellularautomat $C$ mit $L_{\mathcal{A}, C} = L$. Nach \cref{satzRauteTot} kann der Rand von $C$ tot gewählt werden.
    Sei $w \in \Sigma^*$ mit $|w| \in 3\N$, $c := [w]$, $v := \mathcal{A}_H(w)$ und $c' := [\comb(\Spvek{ w; v})]$.
    Nach \cref{CAgfSpeedup} gibt es einen Automaten $C_1$ und eine Funktion $f_1$, sodass für $i \in \N$ gilt:
    \[
        f_1(\Delta_{C_2}^{2i+1}(c)_i) = (\Delta_C^{3i-3}(c)_1, \Delta_C^{3i-2}(c)_1, \Delta_C^{3i-1}(c)_1)
    \]
    
    Ferner lässt sich leicht ein Automat $K$ konstruieren und eine Funktion $f_2$ angeben, sodass für $i \geq 1$ gilt:
    \[
        f_2(\Delta_{K}^{2i+1}(c)_i) = (c_{3i-2}, c_{3i-1}, c_{3i})
    \]
    
    Indem $C_1$ und $K$ gleichzeitig ausgeführt werden, lässt sich ein Automat $M$ konstruieren und eine Funktion $f$ angeben,
    sodass für $n := \frac{|w|}{3}$ gilt:
    \[
        f(\Delta_M^t(c)_i) =
        \begin{cases}
            \mathrm{reset} & \text{falls } t = 0 \text{ und } i \in \finset{1, .., n} \\
            \mathrm{set}(\#\#\#) & \text{falls } t = 0 \text{ und } i \not\in \finset{1, .., n} \\
            \mathrm{set}(s_i)
                & \text{falls } i \in \finset{1, ..., n} \text{ und } t = 2 * i + 1 \\
            \mathrm{step} & \text{sonst}
        \end{cases}
    \]
    Wobei $s_i$ für $i \in \Z$ wie folgt definiert ist:
    \[
        s_i := \begin{cases}
                    \Spvek{ c_{3i-2} ; \Delta_C^{3i-3}(c)_1 }
                        \Spvek{ c_{3i-1} ; \Delta_C^{3i-2}(c)_1} 
                        \Spvek{ c_{3i} ; \Delta_C^{3i-1}(c)_1}
                        \in (\Sigma \times Q_H)^3
                        & \text{falls } i \in \finset{ 1, ..., n } \\
                    \#\#\# \in Q_C & \text{sonst} 
                \end{cases}
    \]
    
    Damit gilt aber gerade $s = S_3(c')$ wobei $S_3(c')$ die komprimierte $c'$-Konfiguration aus \cref{factorSpeedupConstruction} ist.
    Setze $S := S_3(C)$ auf die Speedup-Konstuktion, die drei Schritte in einem simuliert. Setze nun $A := A(M, f, S, \finset{\#})$.
    Dann gilt nach \cref{timeNakamuraConstruction}:
    \[
        (\Delta^{ |w| }_A(c)_1)_a = \Delta_S^{n - 1}(s)
    \]
    Da $\#$ in $C$ und damit auch $\#\#\#$ in $S$ tot ist, gilt damit nach \cref{factorSpeedupConstructionCorrectness}:
    \[
        \gamma_\#((\Delta^{ |w| }_A(c)_1)_a) = \gamma_\#(\Delta_S^{n-1}(s)) = \Delta_C^{|w| - 1}(c')
    \]
    
    Mit entsprechender Wahl der akzeptierenden Zustände und einem konstanten Speedup von einem Schritt
    folgt damit $L \in \CAL{\CART, \CALeft}$.
\end{proof}

\begin{corollary}
    Sei $L \in \CAL{\CART, \CALeft}$.
    Der folgende Advice $\mathcal{A}_L$ ist $\CAM{\CART, \CALeft}$-verträglich:
    \[
        \mathcal{A}_L(w) := v \in \B^{|w|} \text{ mit } v_i :=
        \begin{cases}
            1 & \text{ falls } w_{[1...i]} \in L \\
            0 & \text{ sonst }
        \end{cases}
    \]
\end{corollary}
\begin{proof}
    Nach Wahl von $L$ gibt es einen Zellularautomaten $C \in \CAM{\CART, \CALeft}$
    mit $L_C = L$. Damit gilt für ein $w \in \Sigma_C^*$ und alle $t \in \finset{0, ..., |w|-1}$:
    \[
        \Delta_C^{t}([w])_1 \in
        \begin{cases}
            F_C^+ & \text{falls } w_{[1...t+1]} \in L \\
            Q \setminus F_C^+ & \text {sonst}
        \end{cases}
    \]
    Indem Zustände aus $F_C^+$ als $1$ und Zustände aus $Q \setminus F_C^+$ als $0$ aufgefasst werden, 
    folgt mit \cref{thmAdviceMain} die Behauptung.
\end{proof}


\begin{satz}
    Der folgende Advice $\mathcal{A}$ ist $\CAM{\CART, \CALeft}$-verträglich:
    \[
        \mathcal{A}(w) := \mathrm{ones}(2^{\lfloor \log_2{|w|} \rfloor - 1}, |w|) \in \B^*
    \]
\end{satz}
\begin{proof}
    
\end{proof}
