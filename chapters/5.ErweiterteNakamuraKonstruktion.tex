\chapter{Erweiterte Nakamura-Konstruktion zur asynchronen Simulation}
\label{chap:ErweiterteNakamuraKonstr}

Dieses Kapitel bildet die Grundlage für das beeindruckende \cref{thmAdviceMain}:
Die hier vorgestellte Konstruktion erlaubt es, einen Automaten $C$ durch einen Automaten $A$
zu simulieren,
wobei ein gegebener Automat $M$ die Simulation steuern kann.
Dabei können dynamisch Startzustände in der Simulation von $C$ neu- und zurückgesetzt werden.
Die Konstruktion kommt ohne globale Synchronisation aus,
insbesondere wenn die Startzustände von $C$ zu unterschiedlichen Zeiten von $M$ festgelegt \acs{bzw.} neu gesetzt werden.

Die ursprüngliche Nakamura-Konstruktion, wie in \cite{worsch2010note} beschrieben,
zeigt, wie ein Zellularautomat so umgebaut werden kann, dass er auch bei asynchroner 
Ausführung noch korrekt arbeitet.
Von der ursprünglichen Konstruktion ist in dieser erweiterten Konstruktion
allerdings nicht mehr viel übrig geblieben, außerdem dient diese Konstruktion einem anderen Zweck.
Die hier vorgestellte Konstruktion kann aber weiterhin genutzt werden,
um Zellularautomaten so umzubauen, dass sie auch asynchron korrekt ausgeführt werden.

\section{Definition}

\begin{definition}[Erweiterte Nakamura-Konstruktion $A(M, f, C, P)$]
    \label{erweiterteNakamuraKonstruktion}
    Es seien zwei Zellularautomaten $M, C \in \CA$, eine Funktion $f: Q_M \to \mathrm{Op}$
    mit 
    \[
        \mathrm{Op} := \set{ \mathrm{reset}, \mathrm{set}(q), \mathrm{step} }{ q \in Q_C }
    \]        
    und eine Menge $P \subseteq Q_C$ von passiven, initialen Zuständen gegeben.
    Der Automat $M$ gibt Steuerbefehle über die Funktion $f$ und der Automat $C$ ist der asynchron simulierte Automat.
    Mithilfe der Menge $P$ lässt sich in bestimmten Fällen
    die benötigte Zeit zum Simulieren verbessern.
    
    Setze $N := \finset{ 0, ..., 5 }$, $Q' := (Q_C^3 \times N \times \Nz) \cup \finset{ \perp }$ und $Q := Q' \times Q_M$.
    Die Komponenten eines Elements $q \in Q_C^3$ seien symbolisch mit $q_a$, $q_{a-1}$ und $q_0$ bezeichnet: $q = (q_a, q_{a-1}, q_0)$.
    $q_a$ meint dabei den aktuellen Zustand des simulierten Automaten, $q_{a-1}$ den Zustand des vorherigen Schrittes (sofern dieser existiert) und $q_0$ den Zustand der simulierten Zelle in der Anfangskonfiguration.
    Diese Notation lässt sich auf die erste Komponente von $Q'$ und dann $Q$ übertragen. Die zweite Komponente eines Elements $q \in Q' \setminus \finset{ \perp }$ wird mit $q_t$ und die dritte mit $q_{t'}$ bezeichnet. $q_t$ speichert \enquote{grob}, wieviele Schritte im simulierten Automaten für diese Zelle schon berechnet wurden. $q_{t'}$ speichert diese Zahl exakt und dient nur zur Beweisführung. Da die Komponente $q_{t'}$
    nach Konstruktion nicht relevant für das sichtbare Verhalten des Automaten sein wird, existiert ein äquivalenter Automat, der ohne diese Komponente auskommt. Dessen Zustandsmenge ist dann endlich.
    Für $q \in Q$ bezeichne $q_q$ die erste Komponente und $q_m$ die zweite.
    
    Definiere zunächst $\delta': {Q'}^3 \to Q'$ für $a, b, c \in Q'$ für den Fall $\perp \in \finset{ a, b, c }$:
    \[
        \delta'(a, b, c) := 
        \begin{cases}
            \perp & \text{falls } b = \perp, \\
            ((b_0, b_0, b_0), 0, 0) & \text{sonst.}
        \end{cases}
    \]
    Dem Zustand $\perp \in Q'$ kommt damit die Bedeutung zu, nichts über den Zustand der Zelle im simulierten Automaten zu wissen.
    Die Nachbarn einer Zelle in diesem Zustand können dann offensichtlich keinen Schritt berechnen.
    
    Definiere für $t \in N$ die Funktion $s(t)$ zur Bestimmung eines Nachfolgers in $N$ (siehe \cref{wirkungVonS}):
    \[
        s(t) :=
        \begin{cases}
            t + 1 & \text{falls } t < 5, \\
            3 & \text{falls } t = 5.
        \end{cases}
    \]
    
    \begin{figure}[h]
        \centering
        \includesvg{Nakamura_s.svg}
        \caption{Wirkung von s}
        \label{wirkungVonS}
    \end{figure}
    
    
    Definiere $q^t$ zum Zugriff auf den möglicherweise in $q$ gespeicherten Zustand der Zelle zum Zeitpunkt $t'$ mit $s(t') = t$
    für $q \in Q'$ und $t \in N$:
    \[
        q^{t} :=
        \begin{cases}
            q_{0} & \text{falls } q_0 \in P \text{ und } q_t = 0, \\
            q_{a-1} & \text{falls } q_0 \not\in P \text{ und } q_t = t, \\
            q_a & \text{falls } q_0 \not\in P \text{ und } s(q_t) = t, \\
            \perp & \text{sonst.}
        \end{cases}
    \]
    
    Falls $q_0$ von Anfang an tot ist,
    beinhaltet $q$ Zustandsinformationen
    der simulierten Zelle für alle Zeitpunkte,
    da tote Zustände in jedem Schritt gleich bleiben.
    
    Definiere nun $\delta'$ für $a, b, c \in Q'$ falls $\perp \not\in \finset{ a, b, c}$:
    \[
        \delta'(a, b, c) := 
        \begin{cases}
            \Spvek{(b_0, b_0, b_0), 0, 0}
            & \text{falls } b_0 \in P \text{ und } b_t = 0, \\
            \Spvek{(\delta(a^t, b^t, c^t), b^t, b_0), t, b_{t'} + 1}
            & \text{sonst, falls} \perp \not\in \finset{ a^t, c^t } \text{ für } t := s(b_t), \\
            \Spvek{(\delta(a^t, b^t, c^t), b^t, b_0), t, b_{t'}}
            & \text{sonst, falls} \perp \not\in \finset{a^t, c^t} \text{ für } t := b_t, \\
            \Spvek{(\delta(a_0, b_0, c_0), b_0, b_0), 1, 1}
            & \text{sonst.}
        \end{cases}
    \]
    
    Der erste Fall hält tote Zustände in der Simulation fest, da sie nach Definition konstant sind. Dies ermöglicht nach Definition von $q^t$ eine schnellere Simulation, da nicht auf Zellen in solch einem Zustand gewartet werden muss. Aus Komplexitätsgründen werden nur solche toten Zustände berücksichtigt, die von $M$ mit $\mathrm{set(q)}$ schon direkt gesetzt wurden, was die Bedingung $b_t = 0$ erklärt.
    
    Im zweiten Fall wird der nächste Schritt des simulierten Automaten für die zu aktualisierende Zelle berechnet.
    Dies ist offensichtlich nur möglich, wenn
    die Nachbarzellen Informationen zum aktuellen Schritt gespeichert haben.
    
    Andernfalls, wenn die Nachbarzellen nur Informationen zum letzten Schritt gespeichert haben,
    greift Fall drei und die zu aktualisierende Zelle berechnet den aktuellen Schritt neu.
    Dies ist wichtig, falls sich Zellen zurücksetzen.
    
    Im vierten Fall setzt die Zelle die Simulation
    lokal zurück und berechnet wieder den ersten Schritt der Simulation.
    
    
    Abschließend ist $\delta$ für $a, b, c \in Q$ nun wie folgt definiert:
    \[
        \delta(\Spvek{a_q, a_m}, \Spvek{b_q, b_m}, \Spvek{c_q, c_m}) := 
        \begin{cases}
            \text{Setze } c := \delta_M(a_m, b_m, c_m), \\
            \Spvek{((q, q, q), 0, 0), c} & \text{falls } f(c) = \mathrm{set}(q), \\
            \Spvek{\perp, c} & \text{falls } f(c) = \mathrm{reset}, \\
            \Spvek{\delta'(a_q, b_q, c_q), c} & \text{falls } f(c) = \mathrm{step}.
        \end{cases}
    \]
    
    Damit kann der Automat $M$ die Simulation kontrollieren
    und diese für einzelne Zellen auf einen neuen Anfangszustand zurücksetzen. Die Konstruktion stellt sicher,
    dass sich benachbarte Zellen einer solchen zurückgesetzten Zelle
    korrekt verhalten.
    
    Definiere nun $A(M, f, C, P) := (Q, \delta)$.
    
    Wenn nun eine Konfiguration $c_M : \Z \to Q_M$ von $M$
    gegeben ist, kann sie als Konfiguration $c: \Z \to Q$ von $A(M, f, C, P)$
    aufgefasst werden:
    \[
        c_i :=
        \begin{cases}
            (((q, q, q), 0, 0), (c_M)_i) & \text{falls } f((c_M)_i) = \mathrm{set}(q), \\
            (\perp, (c_M)_i) & \text{ sonst.}
        \end{cases}
    \]
\end{definition}

Um über die Korrektheit der Konstruktion sprechen zu können, sind vorerst noch weitere Definitionen nötig.

\begin{definition}[Sichtbare Startzustandszeit $st_o(t, i)$ und Startkonfiguration $c(t, i)$, sichtbare aktiven Zellen $A(t, i)$]
    Es seien $M, C \in \CA$, $f$ und $P$ wie in \cref{erweiterteNakamuraKonstruktion} und eine Konfiguration $c_M: \Z \to Q_M$ gegeben.
    Seien $t \in \Nz$, $i \in \Z$ und $o \in \mathrm{Op}$.
    Definiere die sichtbare Startzustandszeit $\mathrm{st}_{M, C, f, P, o}(t, i) := \mathrm{st}_o(t, i)$:
    \[
        \mathrm{st}_o(t, i) :=
        \begin{cases}
            t' &
                \text{falls } f(\Delta_M^{t'}(c_M)_i) = o \\
                & \text{ für } t' := \max \set{ k \in \finset{0, ..., t} }{ f(\Delta_M^{k}(c_M)_i) \neq \mathrm{step} },
            \\
            \infty & \text{sonst.}
        \end{cases}
    \]
    Offensichtlich gilt $o_1 = o_2$, falls $st_{o_1}(t, i) \neq \infty \land st_{o_2}(t, i) \neq \infty$ für $o_1, o_2 \in \mathrm{Op}$.
    Setze $st_{\mathrm{set}}(t, i) := \min \set{ st_{\mathrm{set}(q)}(t, i)}{ q \in Q_C }$.
    
    \newcommand{\setq}{\scriptsize $\mathrm{set}$}
    \newcommand{\setp}{\scriptsize $\mathrm{set}'$}
    
    \begin{figure}[h!]
        \begin{center}
        \includesvg[width=360pt]{images/SichtbareStartzustandszeit}
        \end{center}
        \caption{Sichtbare Startkonfiguration}
        \label{fig:SichtbareStartkonfiguration}
        
        Ein (gelb oder grün markiertes) \enquote{$\mathrm{set}$} in der Abbildung bedeutet,
        dass der Automat $M$ über $f$ an entsprechender Position und zum entsprechenden Zeitpunkt
        einen $\mathrm{set}(q)$-Befehl an $A$ übermittelt (das konkret gewählte $q$ ist für dieses Beispiel egal, taucht aber als entsprechend gewähltes $q$ in der sichtbaren Startzustandskonfiguration wieder auf).
        Ein \enquote{$\mathrm{set}'$} bedeutet, dass ein toter Zustand $q$ aus $P$ gesetzt wird.
        Runde Zustände (in weiß) stellen ein \enquote{$\mathrm{step}$} dar.
        
        So ist die sichtbare Startzustandszeit $st_{\mathrm{set}}(1, 6)$ in der Abbildung $\infty$,
        während $st_{\mathrm{set}}(2, 5) = 1$ gilt.
        
        Die fünf- und sechseckigen Zustände (in grün) reflekterien die sichtbare Startkonfiguration aus
        Sicht des viereckigen Zustandes (blau gefärbt, in der Abbildung $t = 4$ und $i = 3$).
        Die Sechsecke stellen dabei die von $c(t,i)_j$ eingefügten Rauten dar.
        
        Der viereckige Zustand reflektiert einen Zustand des simulierten Automaten für einen Zeitpunkt,
        zu dem die so eingefügten Rauten allerdings noch keinen Einfluss nehmen konnten.
        Dieser Zeitpunkt kann also nur zwischen $0$ und $2$ liegen. Tatsächlich kann die Konstruktion
        nur einen Schritt simulieren, da der von $(t, i)$ sichtbare Startzustand für die Zelle $2$ erst zum Zeitpunkt $3$ erfolgt.
        
        Das eingezeichnete, rechts-unendliche Intervall kennzeichnet die von $(t, i)$ aus sichtbaren aktiven Zellen.
        Da \enquote{$\mathrm{set}'$} bei $t = 3$ und $i = 5$ von $(t, i)$ aus nicht sichtbar ist,
        spielt der gesetzte Zustand aus $P$ an dieser Stelle keine Rolle.
        
    \end{figure}
    
    Definiere die von $(t, i)$ aus sichtbare Startkonfiguration $c_{M, f, \#}(t, i)_j := c(t, i)$
    für $\# \in Q_C$, $t \in \Nz$ und $p \in \Z$ wie folgt:
    \[
        c(t, i)_j :=
        \begin{cases}
            q & 
                \text{falls } \mathrm{st}_{\mathrm{set}(q)}(t - |i-j|, j) \neq \infty,
             \\
            \# & \text{sonst.}
        \end{cases}
    \]
    
    Definiere die von $(t, i)$ aus sichtbare tote rechte Zelle $\mathrm{dR}(t, i)$ und linke Zelle $\mathrm{dL}(t, i)$:
    \begin{align*}
        \mathrm{dR}(t, i) & := \min (\set{ j \in \Z}{j \geq i \land c(t, i)_j \in P} \cup \finset{ \infty }) \\
        \mathrm{dL}(t, i) & := \max (\set{ j \in \Z}{j \leq i \land c(t, i)_j \in P} \cup \finset{ -\infty })
    \end{align*}
    
    Definiere damit die von $(t, i)$ aus sichtbaren aktiven Zellen $A(t, i)$:
    \[
        A(t, i) := \finset{ \mathrm{dL}(t, i), ..., \mathrm{dR}(t, i)}
    \]
    
    \cref{fig:SichtbareStartkonfiguration} veranschaulicht die eingeführten Definitionen.
\end{definition}

\section{Korrektheit und Simulationsgeschwindigkeit}

Aufgrund der Komplexität der erweiterten Nakamura-Konstruktion wird die Korrektheit
und die Zeit, die für die Simulation benötigt wird, separat betrachtet.
Erst \cref{timeNakamuraConstruction} zeigt, dass die Konstruktion überhaupt Fortschritte in der Simulation macht.
Der folgende Satz zeigt, dass wenn die Konstruktion Schritte des zu simulierenden Zellularautomaten berechnet, diese korrekt sind.

\begin{satz}[Korrektheit der erweiterten Nakamura-Konstruktion]
    \label{nakamura_korrektheit}
    Es seien $M, C \in \CA$, $f$ und $P$ wie in \cref{erweiterteNakamuraKonstruktion} und eine Konfiguration $c_M: \Z \to Q_M$ gegeben.
    Sei $c$ die Auf"|fassung von $c_M$ als Konfiguration von $A := A(M, f, C, P)$.
    
    Es gelten folgende Aussagen für $t \in \Nz$, $i \in \Z$:
    
    \begin{enumerate}
        \item
            $A$ führt $M$ korrekt aus:
            \[
                (\Delta_A^t(c)_i)_m = \Delta_M^t(c)_i
            \]
        \item
            Sei $c \in \finset{-1, 1}$ mit $\perp \not \in \finset{a := (\Delta_A^t(c)_i)_q, b := (\Delta_A^t(c)_{i+c})_q}$.
            Benachbarte Zellen mit ähnlicher \enquote{groben} Zeit haben ähnliche \enquote{exakte} Zeit:
            \[
                s(a_{t}) = s(b_{t}) \Rightarrow a_{t'} = b_{t'} \text{ und }  a_{t} = s(b_{t}) \Rightarrow a_{t'} = b_{t'} + 1
            \]
            Für $t \in \finset{0, 1, 2}$ gilt offensichtlich:
            \[
                a_{t} = t \Rightarrow a_{t'} = t
            \]
        \item
            Sei $q := (\Delta_A^t(c)_i)_q$ mit $q \neq \perp$.
            Dann reflektiert $q_a$ aus Sicht von $(t, i)$ den korrekt simulierten Zustand:
            \[
                q_a = \Delta_C^{q_{t'}}(c_{M, f, \#}(t, i))_i
            \]
            Analog reflektiert $q_{a-1}$ den vorherigen Zustand, falls $q_t > 0$:
            \[
                q_{a-1} = \Delta_C^{q_{t' - 1}}(c_{M, f, \#}(t, i))_i
            \]
            Für $q_t = 0$ gilt $q_{a-1} = q_a$.
    \end{enumerate}
\end{satz}
\begin{proof}
    \begin{enumerate}
        \item Offensichtlich.
        \item Kann induktiv leicht gezeigt werden.
            Beim Rücksetzen durch $\mathrm{set}(q)$ werden benachbarte Zellen ebenfalls auf ihren Startzustand zurückgesetzt, um diese Invariante zu erhalten.
        \item
            In dieser Arbeit wird aus Zeitgründen auf einen formalen Beweis dieser Aussage verzichtet.
            Die Komplexität der Definition der Konstruktion zieht sehr viele Fallunterscheidungen nach sich,
            die ohne geschickte Beweisführung nicht mehr nachvollziehbar sind.
            Der Leser überzeuge sich anhand von Beispielen von der Richtigkeit dieser Aussage.
    \end{enumerate}
\end{proof}

\begin{satz}[Zeitbedarf der erweiterten Nakamura-Konstruktion in einfachen Fällen]
    \label{timeNakamuraConstruction}
    Es seien $M, C \in \CA$, $f$ und $P$ wie in \cref{erweiterteNakamuraKonstruktion} und eine Konfiguration $c_M: \Z \to Q_M$ gegeben.
    Sei außerdem $c$ die Auf"|fassung von $c_M$ als Konfiguration von $A := A(M, f, C, P)$.
    
    Für $t \in \Nz$, $i \in \Z$ und $q := (\Delta_{A}^t(c)_i)_q$ setze:
    \[
        \mathrm{tc}(t, i) := \max \set{ \hat{t} \in \Nz}
            {
                \forall k \in \finset{-\hat{t}, ..., \hat{t} } \cap A(t, i):
                \mathrm{st}_{\mathrm{set}}(t-|k|, i+k) \leq t - \hat{t}
            }
    \]
    \begin{enumerate}
        \item \label{timeNakamuraConstruction_1}
            Für $q \neq \perp$ gilt $q_{t'} \leq \mathrm{tc}(t, i)$.
        \item \label{timeNakamuraConstruction_2}
            Für $q \neq \perp$ gilt sogar  $q_{t'} = \mathrm{tc}(t, i)$, falls für alle $k \in \finset{-\hat{t}, ..., \hat{t} } \cap A(t, i)$ gilt:
            \[
                t_1(t, k) > t_2(t, k)
                \; \land \;
                st_{\mathrm{reset}}(t_1(t, k) - 1, i + k) \leq t_2(t, k)
            \]
            Wobei
            \[
                t_1(t, k) := \min \finset{ t_2(t, k - 1), t_2(t, k + 1) }
            \]
            und
            \[
                t_2(t, k) := st_{\mathrm{set}}(t - |k|, i + k)
            \]
            
            Diese zusätzlichen Bedingungen stellen sicher,
            dass das Neusetzen von Zuständen erst nach einem Zurücksetzen passiert
            und dass das Zurücksetzen ausreichend früh geschieht, sodass benachbarte Zellen
            in ihrer Simulation auf das Neusetzen warten und nicht ihre Simulation mit veralteten Startzuständen weiterführen.
            Warten sie nämlich nicht, müssen sie ebenfalls zurückgesetzt werden und die Simulation läuft langsamer ab.
            
            
        \item \label{timeNakamuraConstruction_3}
            Sei $c': \Z \to Q_C$ eine Konfiguration von $C$ und $n$ so, dass $c'_i \neq \# \Leftrightarrow i \in \finset{ 1, ..., n }$.
            Angenommen, $P \subseteq \finset{ \# } $ und für $i \in \Z$ und $t \in \Nz$ gelte wie in \cref{fig:NakamuraZeit} dargestellt:
            \[
                f(\Delta_M^t(c_M)_i) =
                \begin{cases}
                    \mathrm{reset} & \text{falls } t = 0 \text{ und } i \in \finset{1, ..., n}, \\
                    \mathrm{set}(c'_i) & \text{falls } t = 0 \text{ und } i \not\in \finset{1, ..., n}, \\
                    \mathrm{set}(c'_i) & \text{falls } i \in \finset{1, ..., n} \text{ und } t = 2i + 1, \\
                    \mathrm{step} & \text{sonst.}
                \end{cases}
            \]
            
            Dann gilt für $3 \leq t \leq 3n$:
            \[
                (\Delta^{t}_A(c)_1)_a = \Delta^{\lfloor \frac{t}{3} \rfloor - 1}_C(c')
            \]
     \end{enumerate}
\end{satz}
\begin{proof}
    \begin{itemize}
        \item[\ref{timeNakamuraConstruction_1},] % hack
         \ref{timeNakamuraConstruction_2}
            Wie in \cref{nakamura_korrektheit} wird auch für diese Aussagen
            auf einen formalen Beweis verzichtet. Der Leser überzeuge sich auch hier durch Beispiele,
            wie in \cref{fig:NakamuraZeit} gezeigt, von der Korrektheit der Aussage.

        \item[\ref{timeNakamuraConstruction_3}] Folgt aus \ref{timeNakamuraConstruction_2}.

            \newcommand{\nreset}{\scriptsize $\mathrm{reset}$}
            \newcommand{\setq}{\scriptsize $\mathrm{set}$}
            \begin{figure}[h!]
                \begin{center}
                \includesvg[width=210pt]{images/NakamuraZeit}
                \end{center}
                \caption{Simulierte Zeit im Vergleich zur Simulationszeit}
                \label{fig:NakamuraZeit}
                Die Abbildung ist analog zu \cref{fig:SichtbareStartkonfiguration} zu verstehen.
                Viereckige Zustände (in gelb) geben dabei an,
                dass für eine simulierte Zelle der nächste Zustand berechnet wurde.
                Die Zahl eines Zustands gibt an, wie viele Schritte für diese Zelle schon simuliert wurden.
                Nicht in jedem Simulations-Schritt kann auch ein simulierter Schritt berechnet werden,
                da die dafür notwendigen Startzustände erst zu einem späteren Simulations-Zeitpunkt gesetzt werden.
            \end{figure}
    
            Es gilt offensichtlich: 
            \[
                st_\mathrm{set}(t, i) =
                    \begin{cases}
                        0 & \text{falls } t \not\in \finset{1, ..., n}, \\
                        \infty & \text{sonst, falls } t \leq 2i, \\
                        2i + 1 & \text{sonst.}
                    \end{cases}
            \]
            
            Sei $\hat{t} \in \finset{0, ..., n}$ so, dass
            $t \in \finset{3\hat{t} + 3, ..., 3\hat{t} + 5}$.
            Sei außerdem $k \in \finset{-\hat{t}, ..., \hat{t}}$.
            
            \begin{itemize}
                \item[1. Fall:] $-\hat{t} \leq k \leq -1$.
                    
                    Dann $\mathrm{st}_{\mathrm{set}}(t-|k|, 1+k) = 0 \leq t - \hat{t}$.
                    
                \item[2. Fall:] $0 \leq k \leq \hat{t}$.
                
                    Da $t \geq 3\hat{t}+3$, folgt $t-|k| > 2k+2$ und
                    damit $\mathrm{st}_{\mathrm{set}}(t-|k|, 1+k) = 2k+3 \leq 2\hat{t} + 3 \leq t - \hat{t}$.
                    
                    Da $t \leq 3\hat{t} + 5$, folgt $t-|\hat{t}+1| \leq 2(\hat{t}+1)+2$ und
                    damit $\mathrm{st}_{\mathrm{set}}(t-|k|, 1+k) = \infty$.
            \end{itemize}
            
            Also $\mathrm{tc}(t, i) = \hat{t} = \lfloor \frac{t}{3} \rfloor - 1$ wodurch mit \cref{nakamura_korrektheit} die Behauptung folgt - schließlich ist $c'$ die einzig sichtbare Konfiguration.
    \end{itemize}
\end{proof}