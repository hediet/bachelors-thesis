\chapter{Grundlagen}
\label{chap:Grundlagen}

\section{Notation}

In diesem Abschnitt werden Besonderheiten der in dieser Arbeit verwendeten Notation geklärt.

Sofern nicht anders vermerkt, bezeichnen $\Sigma$ und $\Gamma$ beliebige endliche Mengen, auch Alphabete genannt.

In dieser Arbeit werden die Zeichen eines Wortes mit 1 beginnend durchnummeriert.
Für ein Wort $w \in \Sigma^*$ gilt also die Identität $w = w_1w_2...w_{|w|}$.
Der Ausdruck $w[m...n]$ meint das Teilwort von $w$, das die Zeichen $w_m$, $w_{m+1}$, ... bis einschließlich $w_n$ umfasst.

$\mathcal{P}(M)$ bezeichne die Potenzmenge einer Menge, also die Menge aller ihrer Teilmengen.

Weiter bezeichne $\B := \finset{ \chr{0}, \chr{1}}$ das Binär-Alphabet.

Wenn ein Tupel $T$ durch $T := (A, B, ..., Z)$ definiert wird, wobei $A$, $B$, ..., $Z$ symbolische Variablen sind,
die sich auf einen entsprechenden Wert beziehen,
bezeichnen $A_T$, $B_T$, ..., $Z_T$ die Komponenten des Tupels $T$. Es gilt dann offensichtlich: $T = (A_T, B_T, ..., Z_T)$.

Für eine Fallunterscheidung
\[
    x = \begin{cases}
        1 & \text{falls } A \\
        2 & \text{sonst, falls } B \\
        3 & \text{sonst, falls } C \\
        4 & \text{sonst}
    \end{cases}
\]
meint ein \enquote{sonst}, dass alle darüberstehenden Fälle nicht zutreffen.
So ist $x = 2$, falls $B$ gilt, aber nicht $A$ und $x = 3$, falls weder $A$ noch $B$ gilt, aber $C$.

\section{Definition Zellularautomat}

In diesem Abschnitt werden grundlegende Objekte und Definitionen im Zusammenhang mit den in dieser Arbeit betrachteten Zellularautomaten eingeführt.

\begin{definition}[Zellularautomat]
    Ein eindimensionaler Zellularautomat mit einer Nachbarschaft von Radius 1 wird in dieser Arbeit als Tupel $(Q, \delta)$ definiert,
    wobei $Q$ eine endliche, nicht-leere Menge ist und
    $\delta$ eine Abbildung mit $\delta: Q^3 \to Q$.
    $Q$ wird auch Zustandsmenge und $\delta$ lokale Überführungsfunktion genannt.
    Es bezeichne $\CA$ die Menge aller solcher Tupel.
    In dieser Arbeit werden nur solche Zellularautomaten betrachtet, weswegen die Dimension und Nachbarschaft im Folgenden nicht mehr erwähnt wird.
\end{definition}

Es werden nun Begriffe eingeführt, mithilfe derer bestimmte Verhaltensweisen von Zuständen in Zellularautomaten gefordert werden können.

\begin{definition}
    Sei $(Q, \delta) \in \CA$ ein Zellularautomat.
    \begin{itemize}
        \item $R \subseteq Q$ heißt passive Zustandsmenge, wenn $\forall a, b, c \in R: \delta(a, b, c) = b$.
        \item $q \in Q$ heißt passiver Zustand, wenn $\finset{q}$ eine passive Zustandsmenge ist.
        \item $R \subseteq Q$ heißt $\delta$-abgeschlossene Menge, wenn $\forall b \in R: \forall a, c \in Q: \delta(a, b, c) \in R$.
        \item $q \in Q$ heißt toter Zustand, wenn $\finset{q}$ eine $\delta$-abgeschlossene Menge ist.
        \item $\delta$ heißt links \acs{bzw.} rechts-unabhängig,
            wenn $\forall a_1, a_2, b, c \in Q: \delta(a_1, b, c) = \delta(a_2, b, c)$
            \acs{bzw.}
            wenn $\forall a, b, c_1, c_2 \in Q: \delta(a, b, c_1) = \delta(a, b, c_2)$.
        \item $q \in Q$ heißt initial, wenn
            $\delta(a, b, c) = q \Rightarrow b = q$.
    \end{itemize}
\end{definition}

Die globale räumliche Situation eines Zellularautomaten zu einem festen Zeitpunkt wird mithilfe einer Konfiguration beschrieben.

\begin{definition}[Konfiguration]
    Eine Konfiguration über einer endlichen Zustandsmenge $Q$ ist eine Abbildung $c \in Q^{\mathbb{Z}}$.
    Der Zustand an Position $i$ einer Konfiguration $c$ wird durch $c_i := c(i)$ beschrieben.
\end{definition}

Die lokalen Regeln eines Zellularautomaten bestimmen, wie aus einer Konfiguration eine Folgekonfiguration entsteht.

\begin{definition}[Folgekonfiguration]
    Die Funktion $\Delta_C$ bildet in Abhängigkeit eines
    Zellularautomaten $C = (Q, \delta) \in \CA$ eine Konfiguration $c$ auf die Folgekonfiguration $\Delta_C(c)$ ab:
    \[ 
        \Delta_C: Q^\Z \to Q^\Z, \;
        (\Delta_C(c))(i) := \delta(c_{i - 1}, c_i, c_{i + 1})
    \]
    
    $\Delta_C^t(c)$ bildet eine Konfiguration $c$ auf die $t$-te Folgekonfiguration für ein $t \in \Nz$ ab.
    Es gilt $\Delta_C^0 = \mathrm{id}$ und $\Delta_C^t = \Delta_C \circ \Delta_C^{t-1}$.
\end{definition}

Ausgehend von einer Konfiguration und dem Begriff einer Folgekonfiguration kann ein Zellularautomat nun verwendet werden,
um ein Berechnungsmodell zu bilden. Zunächst werden aber Objekte eingeführt, um von verschiedenen solcher Modelle abstrahieren zu können.

\begin{definition}[$\Sigma^*$ erkennender Zellularautomat]
    Ein $\Sigma^*$ erkennender Zellularautomat ist ein Tupel $C := (Q, \delta, \Sigma, \#, L, S)$ 
    mit $(Q, \delta) \in \CA$, $\Sigma \subseteq Q$, $\# \in Q \setminus \Sigma$ und $L \subseteq \Sigma^*$ beliebig.
    $S$ ist ein beliebiges Objekt derart, dass die Klasse aller solcher $S$ eine Menge bildet.
    
    \# wird bei der Einbettung von Wörtern in Konfigurationen als Rand verwendet.
    Im Gegensatz zu üblichen Definitionen von Zellularautomaten zur Erkennung von Sprachen, die etwa fordern, dass $\#$ passiv oder gar tot ist,
    gibt es in dieser Definition keine Einschränkungen.
    Es wird später gezeigt, dass $\#$ immer passiv und in den in dieser Arbeit relevanten Fällen auch tot gewählt werden kann.

    Es bezeichne $\ECA$ die Menge aller solcher Tupel. Wegen der Forderung an $S$ ist diese Menge wohldefiniert, aber nicht notwendigerweise abzählbar.
    Durch Spezialisierung der Sprache $L$, die in dieser Definition keiner Einschränkung unterliegt,
    und dem ebenfalls frei wählbaren Objekt $S$ werden im Folgenden verschiedene Typen von Zellularautomaten eingeführt.
    Die Menge $\ECA$ erlaubt es, von diesen Spezialisierungen zu abstrahieren.
    
    Für eine Teilmenge $M \subseteq \ECA$ definiere die von $M$ erzeugte Sprachklasse:
    \[
        \mathcal{L}(M) := \set{ L_C }{ C \in M }
    \]
\end{definition}

\begin{definition}[$\ECA$-Funktor]
    Eine Funktion $\mathcal{F}: \Pot(\ECA) \to \Pot(\ECA)$ heißt $\ECA$-Funktor.
    Seien $\mathcal{F}_1, ..., \mathcal{F}_n$ $\ECA$-Funktoren. Definiere:
    \[
        \ECA^{ \mathcal{F}_1 \text{-...-} \mathcal{F}_n } := (\mathcal{F}_n \circ ... \circ \mathcal{F}_1)(\ECA)
    \]
    
    Im Laufe dieser Arbeit werden eine Reihe solcher Funktoren eingeführt.
    Funktoren ermöglichen es, bequem eine Klasse von Zellularautomaten zu spezialisieren oder in eine andere zu transformieren.
\end{definition}

Bisher wurde noch nicht beschrieben, wie Zellularautomaten mit endlichen Wörtern interagieren. Dazu werden diese zu einer Konfiguration fortgesetzt.

\begin{definition}[Einbettung von Wörtern in Konfigurationen]
   Für einen erkennenden Zellularautomaten $(Q, \delta, \Sigma, \#, L, S) \in \ECA$
   lässt sich jedes Wort $w \in \Sigma^*$ wie folgt zu einer Konfiguration fortsetzen:
   \[
       [w]: \Z \to Q, \;
       [w](p) :=
       \begin{cases} 
          w_p & p \geq 1 \land p \leq |w| \\
          \# & \text{sonst}
       \end{cases}
   \]
\end{definition}

\begin{exmp}[Konfiguration]
    Für $w = \chr{abc}$ wird $[w]$ durch folgende Tabelle dargestellt:
    
    \begin{center}
        \addvbuffer[8pt 8pt]{\begin{tabular}
           { | c | c | c | c | c | c | c | c | c | c | c | c | c  | c  | c |}\hline
             $p$ & ... & -2 & -1 & 0 & 1 & 2 & 3 & 4 & 5 & 6 & ... \\ \hline
          $[w]_p$ & ... & \# & \# & \# & \chr{a} & \chr{b} & \chr{c} & \# & \# & \# & ... \\ \hline
        \end{tabular}}
    \end{center}
    
    Die Position einer Zelle in einer Konfiguration wird hier durch die Variable $p$ beschrieben.
\end{exmp}


\subsection{Definition \texorpdfstring{$F$}{F}-haltender Zellularautomat}

Ausgehend von den eingeführten Objekten kann nun ein erstes Berechnungsmodell zum konkreten Erkennen von Sprachen definiert werden.

\begin{definition}[$F$-haltender Zellularautomat]
    Es sei ein Zellularautomat $C := (Q, \delta, \Sigma, \#, L, (F, F^+, p)) \in \ECA$
    mit $F \subseteq Q$, $F^+ \subseteq F$ und $p : \Nz \to \Nz$, $p(0) = 0$ gegeben.
    
    Anschaulich gesprochen repräsentiert $F$ die Menge der finalen Zustände, $F^+$ die Menge der akzeptierenden finalen Zustände und
    $F^- := F \setminus F^+$ die Menge der nicht-akzeptierenden finalen Zustände.
    $p$ ist eine Funktion, die abhängig von der Länge des Eingabeworts die Position der Zelle
    angibt, welche anzeigt, ob das Eingabewort akzeptiert wird.
    
    
    Für ein Wort $w \in \Sigma^*$ gibt die Funktion $\ctime_{C}$ an, wie viele Schritte der Zellularautomat $C$ braucht, um das Wort $w$ zu akzeptieren:
    \[
        \ctime_{C}: \Sigma^* \to \Nz \cup \{ \infty \}, \; w \mapsto \min (\set{ t \in \Nz }{ \Delta_C^t([w])_{p(|w|)} \in F } \cup \{ \infty \})
    \]
    
    Dies lässt sich auf die Wortlänge übertragen:
    \[
        \cworsttime_{C}: \Nz \to \Nz \cup \{ \infty \}, \;  n \mapsto \max \set{ \ctime_{C}(w) }{ w \in \Sigma^n }
    \]
    
    $C$ heißt \emph{$F$-haltend}, wenn $L = L'$ mit
    \[
        L' := \set{ w \in \Sigma^* }{ 
            \ctime_{C}(w) \neq \infty \wedge \Delta_C^{\ctime_{C}(w)}([w])_{p(|w|)} \in F^+ }
    \]
    Das $F$ in $F$-haltend ist symbolisch gemeint und bezieht sich nicht auf den Wert der Menge $F$.
    
    Mit folgender Definition des $\ECA$-Funktors $\CAF$ bezeichnet $\ECA^\CAF$ die Menge aller $F$-haltenden Zellularautomaten:
    \[
        \CAF(M \subseteq \ECA) := \set{ C \in M }{ C \text{ ist $F$-haltend} }
    \]
    
    Ist $\mathcal{F}$ ein $\ECA$-Funktor, sodass die Menge $\set{p_C}{C \in \ECA^{\CAF \text{-} \mathcal{F}}}$ abzählbar ist,
    dann ist auch die Menge $\ECA^{\CAF \text{-} \mathcal{F}}$ abzählbar.
    Die später eingeführten Funktoren $\CALeft$, $\CARight$ und $\CAMid$ haben diese Eigenschaft.
\end{definition}

\begin{comment}
\begin{definition}[Entscheider]
    Ein $F$-haltender Zellularautomat $C$ heißt Entscheider, wenn $\cworsttime_C(\Nz) \subseteq \Nz$,
    $\ctime_C$ also nie den Wert $\infty$ annimmt.
\end{definition}
\end{comment}

\subsection{Definition \texorpdfstring{$t$}{t}-haltender Zellularautomat}

Analog werden die $t$-haltenden Zellularautomaten definiert, bei denen keine finalen Zustände das Ende der Berechnung signalisieren,
sondern beliebige Funktionen. Auf die Unterschiede zu $F$-haltenden Automaten wird in \cref{sec:Vergleich_t_haltend_F_haltend} eingegangen.

\begin{definition}[$t$-haltender Zellularautomat]
    Gegeben ein Zellularautomat $C := (Q, \delta, \Sigma, \#, L, (F^+, t, p)) \in \ECA$
    mit $F^+ \subseteq Q$ und $t, p: \Nz \to \Nz$, $t(0) = 0$, $p(0) = 0$.
    $F^+$ repräsentiert die Menge der akzeptierenden Zustände und $t$ und $p$ Funktionen,
    die abhängig von der Länge des Eingabeworts angeben,
    wie viele Schritte auszuführen sind \acs{bzw.} welche Zelle entscheidet, ob das Wort zu akzeptieren ist.
    
    $C$ heißt \emph{$t$-haltend}, wenn $L = L'$ mit
    \[
        L' := \set{ w \in \Sigma^+ }{ \Delta_C^{t(|w|)}([w])_{p(|w|)} \in F^+ }
    \]
    Das $t$ in $t$-haltend ist symbolisch gemeint und bezieht sich nicht auf die konkrete Funktion $t$.

    Mit folgender Definition des $\ECA$-Funktors $\CAT$ bezeichnet $\ECA^\CAT$ die Menge aller $t$-haltender Automaten:
    \[
        \CAT(M \subseteq \ECA) := \set{ C \in M }{ C \text{ ist $t$-haltend} }
    \]
\end{definition}



\subsection{Verschiedene Funktoren}

Es werden nun verschiedene Funktoren definiert, über die $t$- oder $F$-haltende Automaten weiter eingeschränkt werden können.

\begin{definition}[$\CALeft$-, $\CAMid$- und $\CARight$-\ECA-Funktoren]
    Ein $t$- oder $F$-haltender Zellularautomat $C$ heißt
    \begin{itemize}
        \item \emph{links-erkennend}, falls $p_C(n) = 1$ (definiert $\ECA$-Funktor $\CALeft$),
        \item \emph{mittig-erkennend}, falls $p_C(n) = \lceil n / 2 \rceil$ (definiert $\ECA$-Funktor $\CAMid$) und
        \item \emph{rechts-erkennend}, falls $p_C(n) = n$ (definiert $\ECA$-Funktor $\CARight$).
    \end{itemize}
\end{definition}


\begin{definition}[$\mathrm{time}$-, $\mathrm{\CART}$-, $\mathrm{\CALT}$- und $\mathrm{Mid}$-\ECA-Funktoren]

    Sei $F \subseteq {\Nz}^{\Nz}$ eine Menge von Funktionen. Definiere:
    \[
        \mathrm{time}(F)(M \subseteq \ECA) := \set{ C \in \CAF(M) }{ \cworsttime_C \in F } \; \union \; \set{ C \in \CAT(M) }{ t_C \in F }
    \]
    
    Mit folgenden Definitionen ist $\CAM{\CALeft, \CART}$ \acs{bzw.} $\CAM{\CALeft, \CALT}$ die Menge der links-erkennenden Realzeit- \acs{bzw.} Linearzeit-Zellularautomaten:

    \begin{itemize}
        \item $\mathrm{\CART} := \mathrm{time}(\finset{n \mapsto n - 1}) \circ \CAT$
        \item $\mathrm{\CALT} := \mathrm{time}(O(n)) \circ \CAF$
        \item $\mathrm{Mid} := \mathrm{time}(\finset{n \mapsto \lfloor n/2 \rfloor}) \circ \CAT \circ M $
    \end{itemize}   
\end{definition}


\begin{definition}[$\LOCA$- und $\ROCA$-\ECA-Funktoren]
    Der $\LOCA$ \acs{bzw.} $\ROCA$ Funktor wählt diejenigen links- \acs{bzw.} rechts-erkennenden Automaten aus, deren Zustandsübergangsfunktion links- \acs{bzw.} rechts-unabhängig ist.
\end{definition}

\section{Vergleich von \texorpdfstring{$t$}{t}-haltenden und \texorpdfstring{$F$}{F}-haltenden Automaten}
\label{sec:Vergleich_t_haltend_F_haltend}

Es folgt ein Lemma für $F$-haltende Automaten. Es zeigt, dass solche Automaten stets so umgebaut werden können,
dass wenn sie einmal einen akzeptierenden oder ablehnenden Zustand einnehmen, diesen nicht mehr verlassen, sich also nicht mehr umentscheiden können.
Ganz trivial ist das Lemma nicht, schließlich können Zustände aus $F$ auch während der Berechnung in Zellen angenommen werden,
die von der Funktion $p$ nicht ausgewählt werden.

\begin{lemma}[$F^+$ und $F^-$ können $\delta$-abgeschlossen gewählt werden]
    \label{lemmaAbgeschlosseneMenge}
    
    Sei $C = (Q, \delta, \Sigma, \#, L, (F, F^+, p))$ ein $F$-haltender Zellularautomat.
    Dann existiert ein $F$-haltender Zellularautomat $C' = (Q', \delta', \Sigma, \#, L', (F', F'^+, p))$ mit $\ctime_{C} = \ctime_{C'}$ und $L = L'$,
    wobei $F'^+$ und $F'^-$ $\delta$-abgeschlossene Mengen sind.
\end{lemma}
\begin{proof}
    Setze $Q' := Q \times \finset{{-1}, 0, 1}$ und identifiziere $\Sigma$ mit $\Sigma \times \finset{0}$ und $\#$ mit $(\#, 0)$.
    Wähle $\delta'$ wie folgt:
    \[
       \delta'((a, f_a), (b, f_b), (c, f_c)) :=
       \begin{cases}
         (\delta(a, b, c), f_b)  & \text{falls } f_b \neq 0 \\
         (\delta(a, b, c), {-1}) & \text{falls } f_b = 0 \text{ und } \delta(a, b, c) \in F^- \\
         (\delta(a, b, c), {1}) & \text{falls } f_b = 0 \text{ und }  \delta(a, b, c) \in F^+ \\
         (\delta(a, b, c), {0}) & \text{sonst}
       \end{cases}
    \]
    
    Setze $F' := Q \times \finset{{-1}, 1}$ und $F'^+ := Q \times \finset{1}$.
    Dann sind die Mengen $F'^+$ und $F'^-$ offensichtlich $\delta$-abgeschlossen.
    
    Sei $w \in \Sigma^*, p \in \Z, t \in N_0$ und $(q, f) := \Delta_{C'}^t([w])_p$.
    Dann gilt $q = \Delta_{C}^t([w])_p$. $(q, f) \not\in F'$ impliziert $f = 0$ und damit, dass $\forall k \in \finset{ 0, ..., t }: \Delta_{C}^k([w])_p \not\in F$.
    Anderseits folgt aus $(q, f) \in F'$, dass es ein $k \in \finset{0, ...,  t}$ gibt, sodass $\Delta_{C}^k([w])_p \in F$.
    Da der Automat bei finalen Zuständen stoppt, gilt dann wegen der ersten Überlegung, dass $k = t$.
    Daraus folgt, dass $\ctime_{C} = \ctime_{C'}$ und $L = L'$.
\end{proof}


$t$-haltende Automaten müssen in Abhängigkeit zur Wortlänge stets eine Position und einen Zeitpunkt angeben,
zu dem feststehen muss, ob das Wort akzeptiert oder abgelehnt werden soll.
$F$-haltende Automaten haben die Möglichkeit, nicht zu terminieren.
Da die Funktion zur Wahl des Zeitpunktes bei einem $t$-haltenden Automaten nicht berechenbar sein muss,
sind $t$-haltende Automaten trotzdem nicht weniger mächtig als $F$-haltende, wie der folgende Satz zeigt.

\begin{satz}[$t$-haltende Automaten sind mindestens so mächtig wie $F$-haltende]
    Sei $C = (Q, \delta, \Sigma, \#, L, (F, F^+, p))$ ein $F$-haltender Zellularautomat.
    Dann gibt es einen $t$-haltenden Zellularautomat $C'$ mit $L_{C'} = L_C$, $p_{C'} = p_C$ und $\forall n \in \Nz: t_{C'}(n) \leq \cworsttime_C(n)$.
\end{satz}
\begin{proof}
    Wegen~\cref{lemmaAbgeschlosseneMenge} sind \acs{OBdA.} $F^+$ und $F^-$ $\delta$-abgeschlossene Mengen.
    
    Betrachte folgende Funktion $t$:
    \[
        t: \Nz \to \Nz, \; n \mapsto \max(\finset{ 0 } \cup \set{ \ctime_C(w) }{ w \in \Sigma^n, \; \ctime_C(w) \neq \infty })
    \]

    Wähle $C' := (Q, \delta, \Sigma, \#, L', (F^+, t)) \in \ECA^\CAT$.
    Es gilt offensichtlich $\forall n \in \Nz: t(n) \leq \cworsttime_C(n)$.
    
    Sei $w \in L_C$. Dann gilt $\ctime_C(w) \neq \infty$ und $\Delta_{C}^{\ctime_C(w)}([w])_{p(|w|)} \in F^+$.
    Wegen $t(|w|) \geq \ctime_C(w)$ und weil $F^+$ $\delta$-abgeschlossen ist, folgt $\Delta_{C'}^{t(|w|)}([w])_{p(|w|)} \in F^+$,
    also $w \in L_{C'}$.
    
    Sei $w \in L_{C'}$. Dann gilt $f_1 := \Delta_{C'}^{t(|w|)}([w])_{p(|w|)} \in F^+$.
    Dann ist $\ctime_C(w) \leq t(|w|)$, da in diesem Fall $t(|w|) = 0 \Rightarrow \ctime_C(w) = 0$.
    Nach Definition von $\ctime$ folgt $f_2 := \Delta_{C}^{\ctime_C(w)}([w])_{p(|w|)} \in F$.
    Wäre $f_2 \in F^-$, dann wäre auch $f_1 \in F^- = F \setminus F^+$, da auch $F^-$ $\delta$-abgeschlossen ist.
    Also folgt $f_2 \in F^+$. Also $w \in L_{C'}$.
    
    Insgesamt ergibt sich $L_C = L_{C'}$.
\end{proof}

\begin{remark}
    Das $t$ aus dem vorherigen Beweis ist im Allgemeinen tatsächlich auch nicht berechenbar - andernfalls
    wären semientscheidbare Sprechen stets entscheidbar, da sich Turingmaschinen auf Zellularautomaten reduzieren lassen und umgekehrt.
\end{remark}

Unter bestimmten Annahmen an die Funktion $t$ zur Zeitauswahl können allerdings $t$-haltende Automaten auch zu $F$-haltenden Automaten umgebaut werden.

\begin{satz}[$F$-haltende Automaten sind unter Annahmen mindestens so mächtig wie $t$-haltende]
    Sei $C$ ein $t$-haltender Zellularautomat.
    Wenn es einen $F$-haltenden Zellularautomaten $X$ gibt
    mit $p_X = p_C$, $|\Sigma_X| = 1$ und $\cworsttime_X = t_C$,
    dann gibt es auch einen $F$-haltenden Zellularautomat $C'$ mit $L_{C'} = L_C$, $p_{C'} = p_P$ und $\cworsttime_{C'} = t_C$.
\end{satz}
\begin{proof}
    Führe $X$ und $C$ parallel aus, indem $\delta_{C'}$ durch $\delta_X$ und $\delta_C$ auf $Q_{C'} := Q_X \times Q_C$ definiert wird.
    
    Benutze $X$, um zu erkennen, wann eine Zelle in einen Final-Zustand gehen soll,
    indem $F_{C'} := \set{ (q_1, q_2) }{ (q_1, q_2) \in Q_X \times Q_C, \; q_1 \in F_X } $ gewählt wird.
    
    Benutze $C$, um zu erkennen, ob dieser Final-Zustand akzeptierend oder ablehnend sein soll,
    indem $F_{C'}^+ := \set{ (q_1, q_2) }{ (q_1, q_2) \in Q_X \times Q_C, \; q_1 \in F_X, q_2 \in F_C^+} $ gewählt wird.
    Der so definierte Automat $C'$ erfüllt die Behauptung.
\end{proof}

Folgende Proposition zeigt, dass die Annahmen aus dem vorherigen Satz notwendig sind.

\begin{proposition}[$t$-haltende Automaten sind mächtiger als $F$-haltende]
    Es gibt einen $t$-haltenden links-erkennenden Zellularautomaten mit
    $t(n) \in \finset{0, 1}$,
    welcher eine nicht-semientscheidbare Sprache erkennt.
    Diese Sprache kann offensichtlich nicht
    von einem $F$-haltenden Automaten erkannt werden,
    da sie sonst semientscheidbar wäre.
\end{proposition}
\begin{proof}
    Sei $L \subseteq \Sigma^*$ eine nicht-semientscheidbare Sprache und $\mathrm{cod}: \Nz \to \Sigma^*$ eine berechenbare surjektive Funktion.
    Dann ist $L' := \set{ w \in \Sigma^* }{ \mathrm{cod}(|w|) \in L }$ auch nicht-semientscheidbar.
    Setze $t(n) := 1_{cod(n) \in L}$ und $p(n) := 1$.
    Konstruiere $t$-haltenden Zellularautomat mit $\Sigma \cap F^+ = \emptyset$ und $\delta(\cdot, \cdot, \cdot) \in F^+$.
    Dieser Automat erkennt dann offensichtlich $L'$.
\end{proof}

\begin{comment}
    \begin{proposition}
        Sei $C$ ein $F$-haltender links-erkennender Zellularautomat und $n_0 \in \N$ so, dass $\cworsttime_C(n_0) < n_0$.
        Dann $\forall n \geq n_0: \cworsttime_C(n) = \cworsttime_C(n_0)$.
    \end{proposition}
    \begin{proof}
        Bei der Eingabe eines Wortes $w$ der Länge $n_0$ merkt $C$ nicht mehr, wenn $w$ verlängert wird.
        Ausführlicherer Beweis folgt.
    \end{proof}
\end{comment}

\section{Nützliche Eigenschaften}

Die folgenden zwei Sätze sind äußerst wichtig, um den Rand von Zellularautomaten in den Griff zu kriegen.
Üblicherweise wird schon in der Definition von Zellularautomaten ein toter Rand gefordert, was einige elegante Konstruktionen verhindert.
Es wird zunächst gezeigt, dass der Rand ohne Beschränkung der Allgemeinheit passiv und initial gewählt werden kann.

\begin{satz}[Wahl eines passiven und initialen Randes]
    Sei $C = (Q, \delta, \Sigma, \#, L, (F, F^+, p)) \in \ECA^\CAF$.
    Dann kann $C$ so zu $C'$ mit $L(C) = L(C')$ umgebaut werden, dass $\#_{C'}$ ein passiver und initialer Zustand ist.
    Gleichzeitig bleibt dabei die Links- \acs{bzw.} Rechts-Unabhängigkeit von $\delta_C$ erhalten.
\end{satz}
\begin{proof}
    Setze $Q_{C'} := Q^2 \cup \finset { \#_{C'} }$. Identifiziere $\Sigma$ mit $ \Sigma \times \finset{\#}$ und $\#_{C'}$ mit $\Spvek{\perp; \perp}$, $\perp \not\in Q$.

    Für $q \in Q \cup \finset{\perp}$ und $p \in Q$ definiere folgende abkürzende Notation $q^p \in Q$:
    \[
        q^p := 
        \begin{cases}
            q & \text{falls } q \neq \perp \\
            p & \text{sonst}
        \end{cases}
    \]
    
    Es sei $\delta_{C'}$ definiert durch:
    \[
       \delta'(\Spvek{a_1; a_2}, \Spvek{b_1; b_2}, \Spvek{c_1; c_2}) :=
       \begin{cases}
         \Spvek{  \delta(a_1 ^ {x}, b_1 ^ {x}, c_1 ^ {x}); \delta(x, x, x)  } 
         & \text{falls }
         \finset{x} = \finset{a_2, b_2, c_2} \setminus \finset{ \perp } \\
         \#'
         & \text{sonst}
       \end{cases}
    \]
    Falls $\delta$ links-unabhängig ist, streiche $a_2$ aus der Menge in obiger Definition.
    Falls $\delta$ rechts-unabhängig ist, streiche $c_2$.
    
    Offensichtlich ist nach dieser Konstruktion $\#_C'$ sowohl passiv als auch initial.
    
    Setze $t_r := t_l := t$.
    Falls $\delta$ links-unabhängig ist, setze jedoch $t_l := 0$.
    Falls $\delta$ rechts-unabhängig ist, setze stattdessen $t_r := 0$.
    
    Es reicht nun, für alle $t \in \Nz$ und $i \in \Z$ Folgendes zu zeigen:
    \[
        \Delta_{C'}^t([w])_i = \begin{cases}
            \Spvek{\Delta_C^t([w])_i; \Delta^t_{C}([\varepsilon])} &
                \text{falls } -t_r < i \leq t_l + |w| \\
            \#' & \text{sonst}
        \end{cases}
    \]
    
    Definiere $c^t_i := \Delta_{C'}^t([w])_{i-1}$.
    Die Behauptung gilt offensichtlich für $t = 0$.
    Die Behauptung gelte nun für ein $t \in \Nz$.
    Angenommen, $-(t+1)_r < i \leq (t+1)_l + |w|$.
    Dann folgt nach Induktionsvermutung und Definition von $\delta'$: \[
    \finset{ (c^t_{i-1})_2, (c^t_{i})_2, (c^t_{i+1})_2} \setminus \finset{ \perp } = \finset{ \Delta^t_{C}([\varepsilon])} =: \finset{x}
    \]
    
    Angenommen, $i - 1 \leq -t_r$. Dann folgt mit der Induktionsvermutung $((c^t_{i-1})_1)^x = \Delta^t_{C}([\varepsilon]) = \Delta^t_{C}([w])_{i-1}$.
    Ansonsten $((c^t_{i-1})_1)^x = \Delta^t_C([w])_{i-1}$.
    
    Angenommen, $i + 1 > t_l + |w|$. Dann $((c^t_{i+1})_1)^x = \Delta^t_{C}([\varepsilon]) = \Delta^t_{C}([w])_{i+1}$.
    Ansonsten $((c^t_{i+1})_1)^x = \Delta^t_C([w])_{i+1}$.
    
    Analog gilt für $i = -t_r$, $i = t + 1 + |w|$ und alle anderen $i$ im angenommenen Intervall $(c^t_{i})^x = \Delta^t_C([w])_i$
    
    Folglich $c^{t+1}_i = \Spvek{\Delta^{t+1}_C([w])_i; \Delta^{t+1}_{C}([\varepsilon])}$
    
    Angenommen, $i \leq -(t+1)_r$. Dann folgt mit der Induktionsannahme $c^t_{i} = c^t_{i-1} = \#'$.
    Außerdem gilt $c^t_{i+1} = \#'$ oder $\delta$ ist rechts-unabhängig.
    Angenommen, $i > (t+1)_l + |w|$. Auch dann gilt nach Induktionsannahme $c^t_{i} = c^t_{i+1} = \#'$.
    Analog gilt dann $c^t_{i-1} = \#'$ oder $\delta$ ist links-unabhängig.
    
    In allen Fällen gilt dann $c^{t+1}_i = \#'$.
\end{proof}

Für Linearzeit-Zellularautomaten kann der Rand sogar ohne Beschränkung der Allgemeinheit tot gewählt werden.
Da ein toter Rand einen Speicherverbrauch nach sich zieht, der nur linear von der Wortlänge abhängt, folgt aus den Platzhierarchie-Sätzen über Turingmaschinen,
dass der Rand für allgemeine Zellularautomaten nicht tot gewählt werden kann:
Schließlich können Zellularautomaten von Turingmaschinen mit polynomiellen Mehraufwand für Zeit und Platz simuliert werden.

\begin{satz}[Wahl eines toten Randes bei Linearzeit-Zellularautomaten]
    \label{satzRauteTot}
    Sei $C \in \CAM{\CALeft, \CALT}$ ein Linearzeit-Zellularautomat.
    Dann kann $C$ ohne Veränderung der Laufzeit so zu $C'$ mit $L(C) = L(C')$ umgebaut werden, dass $\#_{C'}$ ein toter Zustand ist.
    Für eine passive Zustandsmenge $M \subseteq \Sigma_C$ mit $\#_C \in M$ ist
    die entsprechende Teilmenge von $\Sigma_{C'}$ auch passiv.
\end{satz}
\begin{proof}
    Da $C$ ein Linearzeit-Zellularautomat ist, gibt
    es ein $k \in \Nz$, sodass $\forall n \in \Nz: \cworsttime_C(n) \leq kn$.
    Setze $K := \finset{-k, ..., k}$ und  $Q' := Q_C^K \cup \finset{ \# }$.
    Die Elemente von $f \in Q_C^K$ können auf $\Z$ durch $f(n \in \Z \setminus K) := \#_C$ fortgesetzt werden.

    Für $i \in K$, $q \in Q'$ und $c \in Q_C$ definiere folgende abkürzende Notation:
    \[
        q_i^c :=
        \begin{rcases}
            \begin{dcases}
                c & \text{falls } q = \# \\
                q(i) & \text{sonst.}
            \end{dcases}
        \end{rcases}
        \in Q_C
    \]

    Definiere $\delta'$ für $a, c \in Q'$ und $b \in Q^K_C$ durch
    
    \begin{alignat*}{2}
        & \delta'(a, \#, c) & := & \; \# \\
        & \delta'(a, b, c)(i \in K) & := &
        \begin{cases}
            \delta(a_i^{b(i - 1)}, b(i), c_i^{b(i + 1)}) & \text{falls } i \text{ gerade ist.} \\
            \delta(c_i^{b(i - 1)}, b(i), a_i^{b(i + 1)}) & \text{sonst.}
        \end{cases}
    \end{alignat*}
    
    Bette $c \in \Sigma$ in $Q'$ ein durch $\phi(c)(i \in K) := \begin{cases} c & \text{ falls } i = 0 \\ \#_C & \text { sonst.} \end{cases}$, $\phi(c) \in Q'$.
    Setze $F' := \set{ q \in Q_C^K }{ q(0) \in F }$
    und $F'^+ := \set{ q \in Q_C^K }{ q(0) \in F^+ }$.
    
    Es bleibt nun die Korrektheit dieser Konstruktion zu zeigen.
    
    Setze $f_{n \in \Nz}(i \in \Z) :=
    \begin{cases} 
        ((i - 1) \; \mathrm{mod} \; n) + 1  & \textrm{ falls } (i \; \mathrm{mod} \; 2n) \leq n \\
        n - ((i - 1) \; \mathrm{mod} \; n)  & \textrm{ sonst}
    \end{cases}$
    .
    \newline
    
    Folgende Tabelle zeigt einige Werte von $f_4$:
    \begin{center}
        \addvbuffer[8pt 8pt]{\begin{tabular}
           { | c | c | c | c | c | c | c | c | c | c | c | c | c  | c  | c |}\hline
             $i$ & -1 & 0 & 1 & 2 & 3 & 4 & 5 & 6 & 7 & 8 & 9 & 10 & 11 & 12 \\ \hline
          $f_4(i)$ & 2 & 1 & 1 & 2 & 3 & 4 & 4 & 3 & 2 & 1 & 1 & 2  & 3  & 4  \\ \hline
        \end{tabular}}
    \end{center}
    
    $f_n$ hat einen steigenden und einen fallenden Teil, die Funktion pendelt zwischen $1$ und $n$
    und es gilt stets $|f_n(i) - f_n(i+1)| \leq 1$.
    
    Für die Korrektheit reicht es nun zu zeigen, dass für $w \in \Sigma^*$, $t \in \finset{0, ..., k|w|}$ und $p \in \finset{-k|w|+t, ..., k|w|-t}$ gilt:
    \[
        \Delta_C^t([w])_p
            = (\Delta_{C'}^t([w])_{f_{|w|}(p)})(\lfloor \frac{p - 1}{|w|} \rfloor)
    \]
    Diese Aussage kann leicht induktiv gezeigt werden.
    Dabei müssen die Fälle betrachtet werden,
    dass $f_{|w|}(p+1)$ \acs{bzw.} $f_{|w|}(p+1)$
    kleiner, größer oder gleich $f_{|w|}(p)$ sein kann.
    Abhängig davon befindet sich die Zelle mit Position $p$ dann direkt neben dem Rand $\#_{C'}$ \acs{bzw.}
    entscheidet sich die Parität von $\lfloor \frac{p - 1}{|w|} \rfloor$.
    
    \begin{comment}
    $P_{t \in \Nz, n \in \N} := \finset{-kn+t, ..., kn-t}$.
    
        Sei $w \in \Sigma^*$.
        Beweis der Behauptung durch Induktion über $t$.

        Sei $t = 0$, $p \in \finset{1, ..., |w|}$.
        Dann $\Delta_{C'}^0([w])_p = w_p
        = \phi(w_p)(0)
        = (\Delta_{C'}^t([w])_{p})(0)
        = (\Delta_{C'}^t([w])_{f_{|w|}(p)})(\lfloor \frac{p - 1}{|w|} \rfloor)$.
        Sei nun $p \in P_{0,|w|} \setminus \finset{1, ..., |w|}$.
        Dann $\lfloor \frac{p - 1}{|w|} \rfloor \in K \setminus \finset{0}$ und
        $\Delta_{C'}^0([w])_p = \#_C =
        (\Delta_{C'}^t([w])_{f_{|w|}(p)})(\lfloor \frac{p - 1}{|w|} \rfloor)$.
    
    
        Sei $1 \leq t \leq k|w|$.
        Dann gilt das auch, ist aber mühselig. Am liebsten
        würde ich ja einen formalen, Computer-gestützten Beweis führen...
    \end{comment}
\end{proof}

Der folgende praktische Satz besagt, dass es auf endlich viele Ausnahmen nicht ankommt.
\begin{satz}
    \label{endlichVieleAusnahmen}
    Sei $C$ ein Zellularautomat,
    $L \subseteq \Sigma^*$ eine Sprache und $M \subseteq \Sigma^*$, $|M| \leq \infty$  eine endliche Menge von Wörtern, sodass $L_C \setminus M = L \setminus M$.
    Dann existiert ein Zellularautomat $C'$, der $L$ in der selben Zeit erkennt.
\end{satz}
\begin{proof}
    Offensichtlich sind dann $M_1 := L \cap M$ und $M_2 := (\Sigma^* \setminus L) \cap M$ auch endlich.
    $M_1$ und $M_2$ sind disjunkt und ihre Vereinigung ist genau $M$.
    $M_1$ enthält die Wörter, die $L$ hat, aber $L_C$ möglicherweise nicht, während $M_2$ die Wörter enthält, die $L$ nicht hat, $L_C$ aber möglicherweise schon.
    Da endliche Sprachen regulär sind und reguläre Sprache von Echtzeit-Zellularautomaten erkannt werden können,
    existieren Zellularautomaten $C_1$ und $C_2$ mit $L_{C_1} = M_1$ und $L_{C_2} = M_2$.
    Akzeptiert weder $C_1$ noch $C_2$ ein Wort $w$, dann akzeptiert es $C$ genau dann, wenn $w \in L$.
    Werden also die Automaten $C$, $C_1$ und $C_2$ parallel ausgeführt, kann leicht ein Echtzeit-Zellularautomat $C'$ konstruiert werden mit $L_{C'} = L$.
\end{proof}