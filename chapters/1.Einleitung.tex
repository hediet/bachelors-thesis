\chapter{Einleitung}

Mithilfe von Zellularautomaten können räumlich und zeitlich diskrete, dynamische Systeme formal beschrieben werden.
Das globale Verhalten eines solchen Zellularautomaten ergibt sich dabei aus lokalen Regeln.
Dies ermöglicht und erzwingt hochgradig paralleles Vorgehen, da räumlich unabhängige Bereiche auch unabhängig voneinander betrachtet und ausgeführt werden können.

Zellularautomaten können auch zum Erkennen von formalen Sprachen verwendet werden und stellen dadurch ein alternatives Berechnungsmodell zur Turingmaschine dar.
Offensichtlich berechnungsäquivalent zu Turingmaschinen, werfen Zellularautomaten durch ihr paralleles Vorgehen eine Reihe von interessanten Fragestellungen auf,
die sich in der Art bei Turingmaschinen nicht ergeben.

Eine der interessantesten und immer noch offenen Fragestellungen in diesem Bereich ist,
ob Linearzeit-Zellularautomaten Sprachen erkennen können, die von Echtzeit-Zellularautomaten nicht erkannt werden können.
Ausgehend von dieser Fragestellung werden in dieser Arbeit verschiedene Erweiterungen von Echtzeit-Zellularautomaten untersucht,
um besser zu verstehen, wie mächtig diese sind.

Zunächst werden in \cref{chap:Grundlagen} verschiedene Typen von Zellularautomaten definiert, verglichen und teilweise bekannte Resultate
in Entsprechung zu den in dieser Arbeit eingeführten Definitionen neu bewiesen.
Anschließend werden in Vorbereitung auf folgende Kapitel in \cref{chap:LinksunabhAuto}, \cref{chap:SpeedupKonstr} und \cref{chap:ErweiterteNakamuraKonstr}
eine Reihe von interessanten Sätzen gezeigt.
Insbesondere die Konstruktion aus \cref{chap:ErweiterteNakamuraKonstr}
zur Simulation von Echtzeit-Zellularautomaten innerhalb von Echtzeit mit Rücksetzfunktionalität
erlaubt neuartige Ansätze, um Erweiterungen von Echtzeit-Zellularautomaten auf normale Echtzeit-Zellularautomaten zurückzuführen.
Solche Erweiterungen werden in \cref{chap:AdvAuto} betrachtet.
In \cref{chap:EingeschrAuto} werden abschließend eingeschränkte Echtzeit-Zellularautomaten untersucht, um Einsicht darüber zu erhalten,
wie leicht Echtzeit-Zellularautomaten hinsichtlich ihrer Mächtigkeit eingeschränkt werden können.

