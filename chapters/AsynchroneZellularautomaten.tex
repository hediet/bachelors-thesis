
\begin{definition}[Partieller Übergang]
    Für $M \subseteq \Z$ und $c \in Q_C^\Z$ definiere:
    \[
         \prescript{M}{} \! \Delta_C(c)_i :=
         \begin{cases}
            \Delta_C(c)_i & \text{falls }i \in M \\
            c_i & \text{sonst}
         \end{cases}
    \]
    
    Für $w \in \Pot({\Z})^*$ und $c \in Q_C^\Z$ definiere:
    \[
        \Delta^w_C := \prescript{w_{|w|}}{} \! \Delta_C \circ ... \circ \prescript{w_1}{} \! \Delta_C
    \]
\end{definition}

\begin{definition}[Asynchroner Zellularautomat]
    Es sei ein Zellularautomat $C := (Q, \delta, \Sigma, \#, L, (F, F^+, p)) \in \ECA$
    mit $F \subseteq Q$, $F^+ \subseteq F$ und $p : \Nz \to \Nz$, $p(0) = 0$ gegeben.

    Es bezeichne $F$ die Menge der finalen Zustände, $F^+$ die Menge der akzeptierenden finalen Zustände und
    $F^- := F \setminus F^+$ die Menge der nicht-akzeptierenden finalen Zustände.
    $p$ ist eine Funktion, die abhängig von der Länge des Eingabeworts die Position der Zelle
    angibt, welche anzeigt, ob das Eingabewort akzeptiert wird.
    
    Weiter seien $F^+_C$ und $F^-_C$ $\delta_C$-abgeschlossen und es gelte für $\forall w \in \Sigma_C^*$:
    \[
        \forall a \in \Pot({\Z})^*: \exists b \in \Pot({\Z})^*:
            \Delta^{a+b}_C([w])_{{p_C}(|w|)} \in
            \begin{cases}
                F_C^+ & \text{ falls } w \in L(C) \\ 
                F_C^- & \text{ sonst }
            \end{cases}
    \]
    
    $C$ heißt asynchroner Zellularautomat, falls $L = L'$ mit
    \[
        L' := \set{ w \in \Sigma^* }{ \exists a \in \Pot({\Z})^*: \Delta^{a}_C([w])_{{p_C}(|w|)} \in F_C^+  }
    \]
\end{definition}


\begin{proposition}
    Sei $C$ ein asynchroner Zellularautomat, $w \in \Sigma^*$.
    Dann gilt 
    \[
            \forall a \in \Pot({\Z})^*: \not\exists b \in \Pot({\Z})^*:
            \Delta^{a+b}_C([w])_{{p_C}(|w|)} \in
            \begin{cases}
                F_C^+ & \text{ falls } w \not\in L(C) \\ 
                F_C^- & \text{ sonst }
            \end{cases}
    \]
    Damit ist die Akzeptanz unabhängig von der Wahl der zu aktualisierenden Zellen.
\end{proposition}
\begin{proof}
    Folgt aus der Definition von asynchronen Zellularautomaten und der Forderung, dass $F^+_C$ und $F^-_C$ $\delta_C$-abgeschlossen sind.
\end{proof}


\begin{definition}[$k$-uniform]
    Sei $C$ ein asynchroner Zellularautomat, $k \in \N$.
    Es bezeichne $\Pot(\Z)_{\leq k}$ die Menge aller Teilmengen von $\Z$, die höchstens $k$ Elemente enthalten.
    
    $C$ heißt $k$-uniform, falls für $\forall w \in \Sigma_C^*$ gilt:
    \[
        \forall a \in \Pot({\Z})^*: \exists b \in \Pot({\Z})_{\leq k}^*:
            \Delta^{a+b}_C([w])_{{p_C}(|w|)} \in
            \begin{cases}
                F_C^+ & \text{ falls } w \in L(C) \\ 
                F_C^- & \text{ sonst }
            \end{cases}
    \]
    Mit anderen Worten: Für $k$-uniforme asynchrone Zellularautomaten reicht es, in jedem Schritt maximal
    $k$ Zellen zu aktualisieren.
\end{definition}

\begin{satz}
    Es gibt einen asynchronen Zellularautomaten, der für kein $k \in \N$ $k$-uniform ist.
\end{satz}

\begin{satz}
    Jeder $F$-haltende Zellularautomat kann in einen $1$-uniformen asynchronen Zellularautomat umgebaut werden.
\end{satz}

\begin{satz}
    SAT kann von einem $1$-uniformen Zellularautomaten in $O(n)$ vielen Schritten gelöst werden.
\end{satz}
