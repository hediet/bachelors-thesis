\chapter{Links-unabhängige Zellularautomaten}
\label{chap:LinksunabhAuto}

Zellen in links-unabhängigen Zellularautomaten hängen in ihrer Berechnung nicht von ihrem linken Nachbarn ab.
Informationen können sich also nur nach links verbreiten. Diese Einschränkung vereinfacht bestimmte Konstruktionen, was in \cref{chap:SpeedupKonstr} ausgenutzt wird.
In diesem Kapitel werden Zusammenhänge zwischen unbeschränkten und linksunabhängigen Zellularautomaten gezeigt.
Ähnliche Zusammenhänge werden in \cite{Choffrut1984} gezeigt.

Es ist allgemein bekannt,
dass $\CAL{\CART, \CALeft, \LOCA}$ eine strikte Teilmenge von $\CAL{\CART, \CALeft}$ ist und
dass $\CAL{\CALT, \CALeft, \LOCA}$ mit $\CAL{\CART, \CARight}$ zusammenfällt (siehe \cite{Kutrib2009}).

Der erste Satz zeigt, dass unbeschränkte Zellularautomaten zu einem links-unabhängigen Automaten
umgebaut werden können. Dabei verschiebt sich die Berechnung räumlich nach links und wird um den Faktor 2 langsamer.
\begin{satz}
    \label{zellautoZuLinksunabhaengig}
    Sei $C = (Q, \delta) \in \CA$ ein Zellularautomat (wie in \cref{fig:NormalZuRechtsUnabh}).
    Dann gibt es einen links-unabhängigen Zellularautomaten $C' = (Q' \supseteq Q, \delta')$ (wie in \cref{fig:NormalZuRechtsUnabh2}),
    sodass
    für $\forall c \in Q^{\Z}, \forall t \in \Nz, \forall i \in \Z$ gilt:
    \[
            \Delta_{C'}^{t}(c)_i =
            \begin{cases}
                \Delta_C^{\frac{t}{2}}(c)_{i+\frac{t}{2}} & \text{ falls } $t$ \text{ gerade, } \\
                (\Delta_C^{\frac{t - 1}{2}}(c)_{i+\frac{t - 1}{2}}, \Delta_C^{\frac{t-1}{2}}(c)_{i+\frac{t+1}{2}})  & \text{ sonst.}
            \end{cases}
    \]
        
    \begin{figure}[h!]
        \centering
        \includesvg[width=185pt]{images/NormalZuLinksUnabh_Norm}
        \caption{Teilweise Ausführung des Automaten $C$}
        \label{fig:NormalZuRechtsUnabh}
    \end{figure}
   \begin{figure}[h!]
        \centering
        \includesvg[width=185pt]{images/NormalZuLinksUnabh_Luh}
        \caption{Teilweise Ausführung des links-unabhängigen Automaten $C'$}
        \label{fig:NormalZuRechtsUnabh2}
    \end{figure}

\end{satz}
\begin{proof}
    Führe für diesen Beweis folgende Notation ein: ${c'}^{t}_i := \Delta^{t}_{C'}(c)_i$ und $c^t_i := \Delta^{t}_{C}(c)_i$.
    
    Setze $Q' := Q \cup Q^2$. Definiere $\delta'$ für $ b, c \in Q$ wie folgt:
    \[
        \delta'(\cdot, b, c) := (b, c) \in Q'
    \]
    und für $b, c \in Q \times Q$:
    \[
        \delta'(\cdot, (b_1, b_2), (c_1, c_2)) := \delta(b_1, b_2, c_2)
    \]
    Definiere $\delta'(\cdot, b, c)$ beliebig für alle anderen Fälle. Offensichtlich ist $\delta'$ dann links-unabhängig.
    
    Sei $c \in Q^{\Z}$. Beweis der Behauptung durch Induktion über $t$. Für $t = 0$ trivial.
    
    Sei nun $t > 0, i \in \Z$.
    
    1. Fall: $t$ ist ungerade.
    Dann gilt nach Induktionsannahme
    \[
        \forall j \in \Z: {c'}^{t-1}_j = c^{\frac{t-1}{2}}_{j+\frac{t-1}{2}} \in Q
    \]
    
    Und damit
    \begin{align*}
        & {c'}^{t}_i \\
        &= \delta'({c'}^{t-1}_{i-1}, {c'}^{t-1}_i, {c'}^{t-1}_{i+1}) \\
        &= ({c'}^{t-1}_i, {c'}^{t-1}_{i+1})) \\
        &= (c^{\frac{t-1}{2}}_{i+\frac{t-1}{2}}, c^{\frac{t-1}{2}}_{i+\frac{t+1}{2}})
    \end{align*}
    
    2. Fall: $t$ ist gerade. Damit $t \geq 2$.
    Dann gilt nach Induktionsannahme
    \[
        \forall j \in \Z: {c'}^{t-1}_j
            = (c^{\frac{t-2}{2}}_{j+\frac{t-2}{2}}, c^{\frac{t-2}{2}}_{j+\frac{t}{2}}) \in Q \times Q
    \]
    
    Und damit
    \begin{align*}
        & {c'}^{t}_i \\
        &= \delta'({c'}^{t-1}_{i-1}, {c'}^{t-1}_i, {c'}^{t-1}_{i+1}) \\
        &= \delta'(q,  (c^{\frac{t-2}{2}}_{i+\frac{t-2}{2}}, c^{\frac{t-2}{2}}_{i+\frac{t}{2}}),
             (c^{\frac{t-2}{2}}_{i+1+\frac{t-2}{2}}, c^{\frac{t-2}{2}}_{i+1+\frac{t}{2}})) \\
        &= \delta(c^{\frac{t-2}{2}}_{i+\frac{t-2}{2}}, c^{\frac{t-2}{2}}_{i+\frac{t}{2}},
                c^{\frac{t-2}{2}}_{i+1+\frac{t}{2}}) \\
        &= c^{\frac{t}{2}}_{i+\frac{t}{2}}
    \end{align*}
    
\end{proof}

Der nächste Satz zeigt die andere Richtung: Links-unabhängige Zellularautomaten können
wieder zu einem Automaten ohne Einschränkung umgebaut werden. Das allein überrascht nicht, schließlich
sind links-unabhängige Automaten eine Spezialisierung der Zellularautomaten ohne Einschränkung.
Interessant ist jedoch, dass sich dabei die Berechnung nach rechts verschieben und um den Faktor 2 beschleunigen lässt.
\begin{satz}
    \label{linksunabhaengigZuZellauto}
    Sei $C = (Q, \delta) \in \CA$ ein links-unabhängiger Zellularautomat (wie in \cref{fig:LinksunabhZuNormal1} gezeigt).
    Dann gibt es einen Zellularautomaten $C' = (Q' \supseteq Q, \delta')$ (wie in \cref{fig:LinksunabhZuNormal2} gezeigt),
    sodass für $\forall c \in Q^{\Z}$, $\forall t \in \Nz$, $\forall i \in \Z$ gilt:
    \[
        \Delta^t_{C'}(c)_{i} = \Delta^{2t}_C(c)_{i-t}
    \]
    
    \begin{figure}[!ht]
        \centering
        \includesvg[width=190pt]{images/LinksunabhZuNormal_Luh}
        \caption{Teilweise Ausführung des links-unabhängigen Automaten $C$}
        \label{fig:LinksunabhZuNormal1}
    \end{figure}
   \begin{figure}[!ht]
        \centering
        \includesvg[width=190pt]{images/LinksunabhZuNormal_Norm}
        \caption{Teilweise Ausführung des Automaten $C'$}
        \label{fig:LinksunabhZuNormal2}
    \end{figure}

\end{satz}
\begin{proof}
    Führe für diesen Beweis folgende Notation ein: ${c'}^{t}_i := \Delta^{t}_{C'}(c)_i$ und $c^t_i := \Delta^{t}_{C}(c)_i$.
    
    Da $Q$ nicht leer ist, gibt es ein $q \in Q$.
    Setze $Q' := Q$ und für $a, b, c \in Q'$:
    \[
        \delta'(a, b, c) := \delta(q, \delta(q, a, b), \delta(q, b, c))
    \]
    
    Sei $c \in Q^{\Z}$. Beweis der Behauptung durch Induktion über $t$. Für $t = 0$ trivial. Sei $t > 0$, $i \in \Z$.

    Nach Induktionsannahme gilt $\forall j \in \Z: {c'}^{t-1}_{j+t-1} = c^{2t-2}_j$.
    
    Es folgt:
    \begin{align*}
        & {c'}^{t}_{j+t} \\
        &= \delta'({c'}^{t-1}_{j+t-1}, {c'}^{t-1}_{j+t}, {c'}^{t-1}_{j+t+1}) \\
        &= \delta'(c^{2t-2}_{j}, c^{2t-2}_{j+1}, c^{2t-2}_{j+2}) \\
        &= \delta(q, \delta(q, c^{2t-2}_{j}, c^{2t-2}_{j+1}), \delta(q, c^{2t-2}_{j+1}, c^{2t-2}_{j+2})) \\
        &= \delta(q, c^{2t-1}_j, c^{2t-1}_{j+1} ) \\
        &= c^{2t}_j 
    \end{align*}
\end{proof}

Mit beiden Sätzen zusammen kann ein Automat zunächst zu einem links-unabhängigen Automaten umgebaut,
dann transformiert und wieder zurück umgebaut werden. Dies wird in \cref{CAgfSpeedup} aus dem nächsten Kapitel ausgenutzt.