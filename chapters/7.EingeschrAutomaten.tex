\chapter{Eingeschränkte Zellularautomaten}
\label{chap:EingeschrAuto}

In diesem Kapitel wird die Auswirkung der Forderung nach passiven Zellen im Eingabewort
bei Echtzeit-Zellularautomaten untersucht.

\section{Definition}

Zunächst wird definiert, inwiefern Echtzeit-Zellularautomaten eingeschränkt werden.
Dazu werden Advices aus dem vorherigen Kapitel verwendet und ein entsprechender Einschränkungs-Funktor definiert.

\begin{definition}
    Zu einer gegebenen Funktion $d: \Nz \to \N$ kann ein $\Sigma$-$\B$-Advice $d^*$ wie folgt konstruiert werden:
    \[
        d^*(w) := \mathrm{ones}(d(|w|), |w|)^{Rev}
    \]        

    Dies definiert den $\CARestr$-$\ECA$-Funktor zum Einschränken von Zellularautomaten:
    \begin{multline*}
        \CARestr(d)(M \subseteq \ECA) := \\
            \set{ C \in \orakel(d^*)(M) }{
                (\Sigma_C \times \finset{ \texttt{0} }) \union \finset{ \#_C } \text{ ist eine passive Zustandsmenge} }
    \end{multline*}
\end{definition}

Es mag auf den ersten Blick seltsam sein, dass Advices, die Zellularautomaten \acs{ggf.} mächtiger machen,
dazu verwendet werden, Zellularautomaten einzuschränken.
Tatsächlich werden die Sprachklassen durch den $\CARestr$-Funktor nicht nur verkleinert,
wie \cref{einschraenkungUnentscheidbar} zeigt.
Mit sinnvollen Wahlen der Einschränkungsfunktion $d$ ergeben sich aber auch sinnvolle Einschränkungen.

\section{Ergebnisse}

Zunächst wird bemerkt, dass der $k$-Schritt Speedup für Zellularautomaten aus \cref{satzEchtzeitSpeedup}
auch für eingeschränkte Zellularautomaten gilt.

\begin{satz}[$k$-Schritt Speedup für eingeschränkte Zellularautomaten]
    \label{satzEingeschraenktEchtzeitSpeedup}
    Sei $d$ eine Funktion $d: \Nz \to \N$. Es gilt:
    \[
        \CAL{\CART, \CALeft, \CARestr(d)} = \CAL{\CAT, \CALeft, \mathrm{time}(\set{ n \mapsto \max \finset{0,  n + k - 1} }{ k \in \Nz }), \CARestr(d)}
    \]
\end{satz}
\begin{proof}
    Siehe Beweis von \cref{satzEchtzeitSpeedup}:
    \cref{satzRauteTot} erhält passive \#-enthaltende Mengen und 
    in der Konstruktion des Automaten $C'$ in \cref{satzEchtzeitSpeedup}
    kann $\Sigma_C \times \finset{\chr{0}}$ mit sich selbst und
    $q \in \Sigma_C \times \finset{\chr{1}}$ mit $\phi_{\#}(q)$ identifiziert werden, ohne dass der Beweis seine Gültigkeit verliert.
    Die Zelle am rechten Rand ist schließlich immer Element von $q \in \Sigma_C \times \finset{\chr{1}}$.
    Passive \#-enthaltende Mengen bleiben also insgesamt erhalten, sodass der konstruierte Automat
    den Bedingungen eingeschränkter Zellularautomaten entspricht.
\end{proof}

\subsection{Relation \texorpdfstring{$\sim_{k,d,L}$}{sim\_kdL}}

Es hat sich als beweistechnisch sinnvoll herausgestellt, eine Relation $\sim_{k,d,L}$ als Verallgemeinerung zur Nerode-Äquivalenzrelation zu betrachten.

\begin{definition}
    Seien $k \in \Nz$, $d$ eine Funktion $d: \Nz \to \N$ und $L \subseteq \Sigma^*$.
    Angelehnt an die Nerode-Äquivalenzrelation wird auf $\Sigma^*$ die Relation $\sim_{k,d,L}$ definiert.
    Seien dazu $v_1, v_2 \in \Sigma^*$.
    \[
        v_1 \sim_{k,d,L} v_2 \; :\Leftrightarrow \; \forall w \in \Sigma^*: d(|wv_1|) = d(|wv_2|) = k \Rightarrow (wv_1 \in L \Leftrightarrow wv_2 \in L)
    \]
    
    $\sim_{k,d,L}$ ist allerdings im Allgemeinen keine Äquivalenzrelation!
\end{definition}

Eng mit der eingeführten Relation $\sim_{k,d,L}$ ist die Funktion $f_{C,k}$ verwandt, wie \cref{satzFEqualsImpliesEquiv} zeigt.
\begin{definition}
    Sei $C \in \CAM{\CART, \CALeft, \CARestr(d)}$ für ein $d: \Nz \to \N$. Definiere die Funktion $f_{C, k}: \Sigma^* \to Q_C^*$
    wie in \cref{fig:RestrAutomata_fCk} gezeigt:
    \[
        f_{C, k}(v) := w[1..\min(k+1, |v|)] \; \text{ mit } \; w := \Delta_C^{\max(0, |v|-k-1)}(  \joinw(\Spvek{ v; \mathrm{ones}(k, |v|)^{Rev} })  )
    \]
    
    \begin{figure}[h!]
        \begin{center}
        \includesvg{fCk}
        \end{center}
        \caption{Berechnung von $f_{C,k}(v)$ für $k = 2$}
        \label{fig:RestrAutomata_fCk}
        Dunkelgraue Zustände sind passiv. Hellgraue Zustände sind für das Akzeptanzverhalten relevant.
    \end{figure}
    
\end{definition}

\begin{satz}
    \label{satzFEqualsImpliesEquiv}
    Sei $C \in \CAM{\CART, \CALeft, \CARestr(d)}$ für ein $d: \Nz \to \N$. Es gilt für alle $k \in \Nz$ und $v_1, v_2 \in \Sigma^*$:
    \[
        f_{C, k}(v_1) = f_{C, k}(v_2) \Rightarrow v_1 \sim_{k,d,L} v_2
    \]
\end{satz}
\begin{proof}
    Sei $k \in \Nz$. Seien $v_1, v_2 \in \Sigma^*$ so, dass $f_{C, k}(v_1) = f_{C, k}(v_2)$.
    Sei $w \in \Sigma^*$ mit $d(|wv_1|) = d(|wv_2|) = k$.
    Es bleibt zu zeigen, dass $wv_1 \in L$ genau dann gilt, wenn $wv_2 \in L$.
    
    Angenommen, $|v_1| < k + 1$. Dann 
    \[
        |f_{C, k}(v_2)| = |f_{C, k}(v_1)| = |v_1| < k + 1
    \]
    und damit $|f_{C, k}(v_2)| = \min(k+1, |v_2|) = |v_2|$, also $|v_1| = |v_2|$.
    Wegen $\max(0, |v_{i \in \finset{1,2}}|-k-1) = 0$ folgt $v_1 = v_2$, also gilt offensichtlich $v_1 \sim_{k,d,L} v_2$.
    
    Seien nun \acs{OBdA.} $|v_1|, |v_2| > k$.
    Setze $u_{i \in \finset{1, 2}} := \joinw(\Spvek{ wv_i; d^*(wv_i) })$
    und $c^t_i := \Delta_C^{t}( u_i )$.
    
    Es gilt wegen der Forderung nach passiven Zuständen (rote Markierung in \cref{fig:RestrAutomata_fCk_Equiv}, graue Zellen sind passiv): 
    \[(c_1^{|v_1|-k-1})_{[1..|w|]} = \joinw(\Spvek{ w; \chr{0}^{|w|} }) = (c_2^{|v_2|-k-1})_{[1..|w|]}\]
    
    Und es gilt (blaue Markierung in \cref{fig:RestrAutomata_fCk_Equiv}): 
    \[(c_1^{|v_1|-k-1})_{[|w|+1..|w|+k+1]} = |f_{C, k}(v_1)| = |f_{C, k}(v_2)| = (c_2^{|v_2|-k-1})_{[|w|+1..|w|+k+1]}\]
    
    
    Also gilt zusammen (grüne Markierung in \cref{fig:RestrAutomata_fCk_Equiv}): 
    \[(c_1^{|v_1|-k-1})_{[1..|w|+k+1]} = (c_2^{|v_2|-k-1})_{[1..|w|+k+1]}\]
    
    \begin{figure}[h!]
        \centering
        \includesvg[width=380pt]{fCkImpliesEquiv}
        \caption{$f_{C,k}(v_1) = f_{C,k}(v_2) \Rightarrow v_1 \sim_{k,d,L} v_2$}
        \label{fig:RestrAutomata_fCk_Equiv}
    \end{figure}
    
    Und damit ist auch der Zustand $f$ aus \cref{fig:RestrAutomata_fCk_Equiv} in beiden Fällen gleich,
    es folgt also schließlich $wv_1 \in L \; \Leftrightarrow \; wv_2 \in L$.
\end{proof}

Damit ergibt sich dann folgende praktische Aussage.

\begin{corollary}
    \label{satzanzahleqivclass}
    Sei $C \in \CAM{\CART, \CALeft, \CARestr(d)}$ für ein $d: \Nz \to \N$, $k \in \Nz$ und
    $M \subseteq \Sigma^*$ eine Menge paarweiser nicht-ähnlicher Wörter bezüglich $\sim_{k,d,L}$. Es gilt:
    \[
        |M| \leq |f_{C, k}(\Sigma^*)| \leq |Q_C|^{k + 2}
    \]
\end{corollary}
\begin{proof}
    Angenommen, $|M| > |f_{C, k}(\Sigma^*)|$. Wegen des Schubfachprinzips gibt es $m_1, m_2 \in M$ mit $f_{C,k}(m_1) = f_{C,k}(m_2)$ und wegen \cref{satzFEqualsImpliesEquiv} folgt mit $m_1 \sim_{k,d,L} m_2$ ein Widerspruch zur Wahl von $M$.
    
    Die zweite Ungleichung folgt aus $|f_{C,k}(\cdot)| \leq k + 1$.
\end{proof}

\subsection{Konstante Einschränkung}

Nicht verwunderlich, kann ein Zellularautomat, der nur endlich viele Zellen zur Berechnung verwenden darf,
auch nur reguläre Sprachen erkennen. Dies ist eine direkte Folgerung aus \cref{satzanzahleqivclass}.

\begin{satz}
    $\forall c \in \N, d(\cdot) := c: \CAL{\CART, \CALeft, \CARestr(d)} = \textrm{REG}$
\end{satz}
\begin{proof}
    Da $\sim_{c,d,L}$ für eine Sprache $L$ gerade mit der Nerode-Äquivalenzrelation übereinstimmt,
    folgt aus \cref{satzanzahleqivclass}, dass es nur endlich viele Nerode-Äquivalenzklassen gibt und damit die Regularität von $L$.
    
    Umgekehrt ist die rechteste Zelle der Eingabe nie passiv.
    Zum Erkennen einer regulären Sprache $L$ kann dann der endliche Automat, der $L^{Rev}$ erkennt, benutzt werden,
    um einen entsprechenden Zellularautomaten zu konstruieren, der dann trotz Einschränkung $L$ erkennt.
\end{proof}

Sind es zwar weiterhin nur endliche viele Zellen, die die Berechnung benutzen darf,
\acs{z.B.} eine oder zwei, hängt die genaue Anzahl
aber wieder frei von der Länge des Eingabewortes ab,
so lassen sich sogar unentscheidbare Sprachen erkennen.

\begin{satz}
    \label{einschraenkungUnentscheidbar}
    $\exists d: \Nz \to \N$ mit $d(\cdot) \leq 2$, sodass $\CAL{\CART, \CALeft, \CARestr(d)}$ eine unentscheidbare Sprache enthält.
\end{satz}
\begin{proof}
    Es gibt einen Zellularautomaten, der das Eingabewort genau dann akzeptiert, wenn vom Advice genau zwei Zellen mit einer $\chr{1}$ markiert wurden.
    Die Funktion $d$ kann so abhängig von der Länge des Wortes Informationen über unentscheidbare Sprachen an den Zellularautomaten weitergeben.
\end{proof}

\subsection{Vergleich verschiedener Einschränkungen}

Nicht verwunderlich, schränkt es einen Echtzeit-Zellularautomat kaum ein, wenn zur Berechnung nur endlich viele Zellen nicht verwenden darf.
\begin{satz}
    Sei $c \in \Nz$. Dann gilt für $d(n) := n - c$:
    \[
        \CAL{\CART, \CALeft, \CARestr(d)} = \CAL{\CART, \CALeft}
    \]
\end{satz}
\begin{proof}
    \cref{bestimmteEinfacheAdvices} zeigt die Inklusion \enquote{$\subseteq$}.
    Für die andere Inklusion kann die asynchrone Simulations-Konstruktion aus \cref{chap:ErweiterteNakamuraKonstr}
    verwendet werden, um dann mit einem entsprechenden Speedup-Faktor (siehe \cref{factorSpeedupConstruction})
    die eigentliche Berechnung ausführen, sobald nach $c$ Schritten keine Zelle im Eingabewort mehr passiv ist.
\end{proof}

Das wohl interessanteste Ergebnis dieses Kapitels ist das folgende: Je weniger Zellen dem Automat für die Berechnung zur
Verfügung stehen (asymptotisch in Abhängigkeit zur Wortlänge gesehen), desto weniger Sprachen kann er erkennen.

\begin{satz}
    \label{d1WenigerAlsd2}
    Seien $d_1, d_2: \Nz \to \N$ zwei Funktionen, sodass $d_1(n) \leq \frac{n}{2}$ und $\liminf\limits_{n \rightarrow \infty} \frac{d_2(n)}{d_1(n)} = 0$.
    Dann gibt es ein $L \in \CAL{\CART, \CALeft, \CARestr(d_1)} \setminus \CAL{\CART, \CALeft, \CARestr(d_2)}$.
\end{satz}
\begin{proof}
    Wähle $L := \set{ w^{Rev}vw }{ v, w \in \B^*, \; |w| = d_1(|wvw|) }$.
    
    Es gilt $L \in \CAL{\CART, \CALeft, \CARestr(d_1)}$, da die nicht-passiven Zellen genau das letzte Vorkommen von $w$ markieren.
    Die aktiven Zeichen wandern dann nach links und versuchen, den Anfang von $w^{Rev}$ zu finden.
    Bereits gefunden Anfänge werden dann Schritt für Schritt verlängert. Anfänge, die nicht fortgesetzt werden können, werden verworfen.

    Es gilt aber auch $L \not\in \CAL{\CART, \CALeft, \CARestr(d_2)}$:
    Sei $n \in \Nz$ und $k := d_1(n)$. Wegen $d_1(n) \leq \frac{n}{2}$ gibt es ein $v \in \B^*$, sodass $n = |v| + 2k$.
    Seien $w_1, w_2 \in \B^k$ zwei verschiedene Wörter der Länge $k$.
    Dann gilt $w_1^{Rev}vw_1 \in L$, aber $w_2^{Rev}vw_1 \not\in L$, also $w_1 \not\sim_{d_2(n),d,L} w_2$.
    Damit ist $\B^k$ eine Menge paarweiser nicht-ähnlicher Wörter und $|\B^k| = 2^k$.
    
    Angenommen, es gibt ein $C \in \CAM{\CART, \CALeft, \CARestr(d_2)}$ mit $L_C = L$.
    
    Nach \cref{satzanzahleqivclass} gilt:
    \[
        \forall n \in \Nz:  2^{d_1(n)} = |\B^k| \leq {|Q_C|}^{d_2(n)+2} = 2^{\log_2(|Q_C|)(d_2(n)+2)}
    \]
    
    Also gilt: 
    \[
        1 \leq \log_2(|Q_C|) ( \frac{ d_2(n) }{ d_1(n) } + \frac{ 2 }{ d_1(n) })
    \]
    Laut Annahme gibt es nun eine monoton steigende Folge $a_n$, sodass $\frac{d_2(a_n)}{d_1(a_n)}$ beliebig klein wird.
    Da dafür $d_1(a_n)$ beliebig groß wird, folgt ein Widerspruch zur Annahme.
\end{proof}

\begin{remark}
    \cref{d1WenigerAlsd2} besagt allerdings nicht, dass die Sprachklassen für langsamer wachsende Einschränkungen Teilmengen sind:
    \cref{einschraenkungUnentscheidbar} zeigt ja gerade, dass selbst Automaten mit fast konstanter Einschränkung
    unentscheidbare Sprachen entscheiden können, die offensichtlich nicht von Automaten mit Einschränkung $d(n) = \frac{n}{2}$ erkannt werden können.
\end{remark}



\begin{comment}

\begin{satz}
    $\forall d \in O(n): \CAL{\CALT, \CALeft, \CARestr(d)} = \CAL{\CALT, \CALeft, \orakel(d^*)}$
\end{satz}



\begin{satz}
    $d(n) := \lceil n / 2 \rceil: \CAL{\CART, \CALeft \CARestr(d)}  \supseteq \CAL{\CART, \CALeft}$
\end{satz}
\end{comment}

\begin{comment}
\begin{satz}
    $\forall c \in \N, d(n) := \lceil n / c \rceil: \CAL{\CART, \CALeft \CARestr(d)}  \supseteq \CAL{\CART, \CALeft}$
\end{satz}
\begin{proof}
    Bisher nur für $c = 2$.
\end{proof}

\begin{satz}
    $\forall c \in \N, d(n) := n - 2^{\lfloor \log_2{n} \rfloor - 1}: \CAL{\CART, \CALeft, \CARestr(d)} = \CAL{\CART, \CALeft}$
\end{satz}
\begin{proof}
    Kompliziert.
\end{proof}
\end{comment}
