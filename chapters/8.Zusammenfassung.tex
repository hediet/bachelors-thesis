\chapter{Zusammenfassung}

Wie erwartet, konnte die usprüngliche Fragestellung, ob Echtzeit mit Linearzeit zusammenfällt,
nicht weiter geklärt werden - das Problem bleibt offen.

Allerdings hat \cref{chap:AdvAuto} gezeigt, dass Echtzeit-Zellularautomaten erstaunlich mächtig sind
und mit \cref{thmAdviceMain} ein bemerkenswertes Werkzeug zur Verfügung gestellt,
mit dem Echtzeit-Zellularautomaten in gewisser Hinsicht hintereinandergeschaltet werden können,
ohne dabei in den Bereich der Linearzeit zu geraten.
Es wurden Advices aufgezeigt, für die es lohnenswert sein könnte, sie weiter auf Echtzeit-Verträglichkeit zu untersuchen.

Die Erweiterte Nakamura-Konstruktion aus \cref{chap:ErweiterteNakamuraKonstr} hat sich durch seine Rücksetz-Funktionalität
von Startzuständen ebenfalls als sehr praktisches Werkzeug zum asynchronen Simulieren von Automaten in Echtzeit herausgestellt.
Detailliertere Korrektheitsbeweise der Konstruktion stehen allerdings noch aus.

\cref{chap:EingeschrAuto} hat gezeigt, dass sich Echtzeit-Zellularautomaten
durch Vorgabe von passiven Zellen in der Eingabe bis hin zu regulären Sprachen einschränken lassen.
Es bleibt offen, für welche Einschränkungen sich eine Hierarchie in den jeweiligen Sprachklassen ergibt.
\cref{d1WenigerAlsd2} zeigt, dass Automaten, die stärkeren Einschränkungen unterliegen, bestimmte Sprachen
nicht mehr erkennen können.
\cref{einschraenkungUnentscheidbar} zeigt allerdings, dass mitunter die stärksten Einschränkungen
plötzlich unentscheidbare Sprachen erkennen lassen, die weniger starke Einschränkungen noch ausgeschlossen haben.
