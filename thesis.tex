\documentclass{template/wissdoc}

% ------------------------------------------------------------------
% Weitere packages:
\usepackage[numbers,sort&compress]{natbib}
\usepackage{svg}
\usepackage[utf8]{inputenc}
\usepackage{etoolbox}
\usepackage{amsmath,amssymb,amsthm,mathrsfs,amsfonts,dsfont,mathtools,stmaryrd,verbatim,faktor,fancyhdr,verbatimbox,csquotes}
\usepackage{courier}
\usepackage[ngerman]{babel}
\usepackage{wrapfig}
\usepackage[nohyperlinks,nolist]{acronym}
\usepackage{graphicx}
\usepackage{float}
\usepackage{cleveref}

%% ---------------- end of usepackages -------------

%% Informationen für die PDF-Datei
\hypersetup{
 pdfauthor={Henning Dieterichs},
 pdftitle={Vergleich der Sprachklassen verschiedener Typen von Zellularautomaten}
 pdfsubject={Zellularautomaten},
 pdfkeywords={Zellularautomaten, Realzeit, Speedup}
}

% Macros, nicht unbedingt notwendig
\newcommand{\bin}{\ensuremath{\mathbb{B}}}
\newcommand{\bins}{\ensuremath{\mathbb{B}^*}}
\newcommand{\finset}[1]{\{ {#1} \}}

\newcommand{\set}[2]{ \{ {#1} \; | \; {#2} \} }
\newcommand{\Z}{\mathbb{Z}}
\newcommand{\Nz}{\mathbb{N}_0}
\newcommand{\N}{\mathbb{N}}
\newcommand{\B}{\mathbb{B}}

\newcommand{\chr}[1]{\texttt{#1}}




\makeatletter
\newcommand{\Spvek}[2][r]{%
  \gdef\@VORNE{1}
  \left(\hskip-\arraycolsep%
    \begin{array}{#1}\vekSp@lten{#2}\end{array}%
  \hskip-\arraycolsep\right)}

\def\vekSp@lten#1{\xvekSp@lten#1;vekL@stLine;}
\def\vekL@stLine{vekL@stLine}
\def\xvekSp@lten#1;{\def\temp{#1}%
  \ifx\temp\vekL@stLine
  \else
    \ifnum\@VORNE=1\gdef\@VORNE{0}
    \else\@arraycr\fi%
    #1%
    \expandafter\xvekSp@lten
  \fi}
\makeatother


\newcommand{\Pot}{\mathcal{P}}
\newcommand{\union}{\cup}
\newcommand{\intersect}{\cap}



\newcommand{\splitw}{\textrm{split}}
\newcommand{\joinw}{\textrm{join}}


\newcommand{\CA}{\textrm{CA}^*}


\newcommand{\sep}[2][\text{-}]{%
  \def\nextitem{\def\nextitem{#1}}% Separator
  \renewcommand*{\do}[1]{\nextitem{##1}}% How to process each item
  \docsvlist{#2}% Process list
}

\newcommand{\CAL}[1]{\ensuremath{\mathcal{L}( \ECA^{ \sep{#1} } })}
\newcommand{\CAM}[1]{\ensuremath{\ECA^{ \sep{#1} }}}

\newcommand{\ECA}{\textrm{CA}}


% F-haltende Automaten
\newcommand{\CAF}{\textrm{FH}}

% t-haltende Automaten
\newcommand{\CAT}{\textrm{TH}}

\newcommand{\CALeft}{\textrm{L}}
\newcommand{\CARight}{\textrm{R}}
\newcommand{\CAMid}{\textrm{M}}

\newcommand{\CART}{\textrm{RT}}
\newcommand{\CALT}{\textrm{LT}}


\newcommand{\ctime}{\operatorname{time}}
\newcommand{\cworsttime}{\operatorname{worst-time}}


\newcommand{\LOCA}{\operatorname{OCA}_L}
\newcommand{\ROCA}{\operatorname{OCA}_R}


\newcommand{\CARestr}{\mathrm{Restr}}

\newcommand{\orakel}{\textrm{Adv}}

\newcommand{\advA}{\ensuremath{\mathcal{A}}}

% Print URLs not in Typewriter Font
\def\UrlFont{\rm}

\newcommand{\blankpage}{% Leerseite ohne Seitennummer, nächste Seite rechts
 \clearpage{\pagestyle{empty}\cleardoublepage}
}

%% +++++++++++++++++++++++++++++++++++++++++++
%% Setup
%% +++++++++++++++++++++++++++++++++++++++++++

\newtheorem{lemma}{Lemma}[section]

\theoremstyle{definition}
\newtheorem{definition}[lemma]{Definition}

\theoremstyle{plain}
\newtheorem{theorem}[lemma]{Theorem}
\newtheorem{corollary}[lemma]{Korollar}
\newtheorem{satz}[lemma]{Satz}
\newtheorem{proposition}[lemma]{Proposition}

\theoremstyle{remark}
\newtheorem{exmp}[lemma]{Beispiel}
\newtheorem*{remark}{Anmerkung}


\crefname{figure}{Abbildung}{Abbildungen}
\crefname{theorem}{Theorem}{Theoreme}
\crefname{corollary}{Korollar}{Korollare}
\crefname{lemma}{Lemma}{Lemmas}
\crefname{satz}{Satz}{Sätze}
\crefname{definition}{Definition}{Definitionen}
\crefname{chapter}{Kapitel}{Kapitel}
\crefname{section}{Abschnitt}{Abschnitte}

\let\Proof=\proof
\let\endProof\endproof 
\renewenvironment{proof}{\begin{Proof}[\bfseries\proofname]}{\end{Proof}}
\renewcommand\qedsymbol{$\square$}

\renewcommand{\labelenumi}{\Roman{enumi}.}

\graphicspath{{images/}}

\newcommand{\bin}{\ensuremath{\mathbb{B}}}
\newcommand{\bins}{\ensuremath{\mathbb{B}^*}}
\newcommand{\finset}[1]{\{ {#1} \}}

\newcommand{\set}[2]{ \{ {#1} \; | \; {#2} \} }
\newcommand{\Z}{\mathbb{Z}}
\newcommand{\Nz}{\mathbb{N}_0}
\newcommand{\N}{\mathbb{N}}
\newcommand{\B}{\mathbb{B}}

\newcommand{\chr}[1]{\texttt{#1}}




\makeatletter
\newcommand{\Spvek}[2][r]{%
  \gdef\@VORNE{1}
  \left(\hskip-\arraycolsep%
    \begin{array}{#1}\vekSp@lten{#2}\end{array}%
  \hskip-\arraycolsep\right)}

\def\vekSp@lten#1{\xvekSp@lten#1;vekL@stLine;}
\def\vekL@stLine{vekL@stLine}
\def\xvekSp@lten#1;{\def\temp{#1}%
  \ifx\temp\vekL@stLine
  \else
    \ifnum\@VORNE=1\gdef\@VORNE{0}
    \else\@arraycr\fi%
    #1%
    \expandafter\xvekSp@lten
  \fi}
\makeatother


\newcommand{\Pot}{\mathcal{P}}
\newcommand{\union}{\cup}
\newcommand{\intersect}{\cap}



\newcommand{\splitw}{\textrm{split}}
\newcommand{\joinw}{\textrm{join}}


\newcommand{\CA}{\textrm{CA}^*}


\newcommand{\sep}[2][\text{-}]{%
  \def\nextitem{\def\nextitem{#1}}% Separator
  \renewcommand*{\do}[1]{\nextitem{##1}}% How to process each item
  \docsvlist{#2}% Process list
}

\newcommand{\CAL}[1]{\ensuremath{\mathcal{L}( \ECA^{ \sep{#1} } })}
\newcommand{\CAM}[1]{\ensuremath{\ECA^{ \sep{#1} }}}

\newcommand{\ECA}{\textrm{CA}}


% F-haltende Automaten
\newcommand{\CAF}{\textrm{FH}}

% t-haltende Automaten
\newcommand{\CAT}{\textrm{TH}}

\newcommand{\CALeft}{\textrm{L}}
\newcommand{\CARight}{\textrm{R}}
\newcommand{\CAMid}{\textrm{M}}

\newcommand{\CART}{\textrm{RT}}
\newcommand{\CALT}{\textrm{LT}}


\newcommand{\ctime}{\operatorname{time}}
\newcommand{\cworsttime}{\operatorname{worst-time}}


\newcommand{\LOCA}{\operatorname{OCA}_L}
\newcommand{\ROCA}{\operatorname{OCA}_R}


\newcommand{\CARestr}{\mathrm{Restr}}

\newcommand{\orakel}{\textrm{Adv}}

\newcommand{\advA}{\ensuremath{\mathcal{A}}}

%% Einstellungen für das gesamte Dokument

% Trennhilfen
% Wichtig! 
% Im ngerman-paket sind zusätzlich folgende Trennhinweise enthalten:
% "- = zusätzliche Trennstelle
% "| = Vermeidung von Ligaturen und mögliche Trennung (bsp: Schaf"|fell)
% "~ = Bindestrich an dem keine Trennung erlaubt ist (bsp: bergauf und "~ab)
% "= = Bindestrich bei dem Worte vor und dahinter getrennt werden dürfen
% "" = Trennstelle ohne Erzeugung eines Trennstrichs (bsp: und/""oder)

% Trennhinweise fuer Woerter hier beschreiben
\hyphenation{
    Zell-u-lar-au-to-ma-ten
}

% Index-Datei öffnen
\ifnotdraft{\makeindex}
%%%%%%%%%%%%%% includeonly %%%%%%%%%%%%%%%%%%%
% Es werden nur die Teile eingebunden, die hier 
% aufgefuehrt sind!
% \includeonly{%
% titelseite,%
% erklaerung,% Ist in KA Pflicht für Diplomarbeiten
% einleitung,% Motivation, Zielsetzung, Gliederung
% grundlagen,% Grundlagen 
% analyse,   % Problembeschreibung (Detail) und Related Work
% entwurf,   % Beschreibung der Problemlösung (Konzepte, allg. Architektur, ...)
% implemen,  % Beschreibung der Umsetzung/Implementierung
% eval,      % Nachweis und Auswertung
% zusammenf  % Zusammenfassung der Ergebnisse und Ausblick
% }
%%%%%%%%%%%%%%%%%%%%%%%%%%%%%%%%%%%%%%%%%%%%%%

\begin{document}

\frontmatter
\pagenumbering{roman}
\ifnotdraft{
 \include{template/titelseite}
 \blankpage % Leerseite auf Titelrückseite
 \thispagestyle{empty}
\vspace*{32\baselineskip}
\hbox to \textwidth{\hrulefill}
\par
Ich erkläre hiermit, dass ich die vorliegende Arbeit selbständig verfasst und
keine anderen als die angegebenen Quellen und Hilfsmittel benutzt, 
die wörtlich oder inhaltlich übernommenen Stellen als solche kenntlich 
gemacht und die Satzung des KIT zur Sicherung guter wissenschaftlicher 
Praxis in der jeweils gültigen Fassung beachtet habe.

\vspace*{2cm}
Karlsruhe, den 25. Juli 2018\hfill \hbox to 8cm{\hrulefill}

%%%%%%%%%%%%%%%%%%%%%%%%%%%%%%%%%%%%%%%%%%%%%%%%%%%%%%%%%%%%%%%%%%%%%%%%
%% Hinweis:
%%
%% Diese Erklärung wird von der Prüfungsordnung für Diplom-, Master,
%% und Bachelorarbeiten verlangt und ist zu unterschreiben. 
%% Für Studienarbeiten ist diese Erklärung nicht zwingend notwendig, 
%% schadet aber auch nicht.
%%%%%%%%%%%%%%%%%%%%%%%%%%%%%%%%%%%%%%%%%%%%%%%%%%%%%%%%%%%%%%%%%%%%%%%%
\clearpage







 \blankpage % Leerseite auf Erklärungsrückseite
}
%
%% *************** Hier geht's ab ****************
%% ++++++++++++++++++++++++++++++++++++++++++
%% Verzeichnisse
%% ++++++++++++++++++++++++++++++++++++++++++
\ifnotdraft{
{\parskip 0pt\tableofcontents} % toc bitte einzeilig
\blankpage
%\listoffigures
%\blankpage
%\listoftables
%\blankpage
}

%% ++++++++++++++++++++++++++++++++++++++++++
%% Akronyme
%% ++++++++++++++++++++++++++++++++++++++++++
\begin{acronym}
 \acro{bzw.}{Beziehungsweise}
 \acro{OBdA.}{Ohne Beschränkung der Allgemeinheit}
\end{acronym}


%% ++++++++++++++++++++++++++++++++++++++++++
%% Hauptteil
%% ++++++++++++++++++++++++++++++++++++++++++

\mainmatter
\pagenumbering{arabic}
\chapter{Einleitung}

Mithilfe von Zellularautomaten können räumlich und zeitlich diskrete, dynamische Systeme formal beschrieben werden.
Das globale Verhalten eines solchen Zellularautomaten ergibt sich dabei aus lokalen Regeln.
Dies ermöglicht und erzwingt hochgradig paralleles Vorgehen, da räumlich unabhängige Bereiche auch unabhängig voneinander betrachtet und ausgeführt werden können.

Zellularautomaten können auch zum Erkennen von formalen Sprachen verwendet werden und stellen dadurch ein alternatives Berechnungsmodell zur Turingmaschine dar.
Offensichtlich berechnungsäquivalent zu Turingmaschinen, werfen Zellularautomaten durch ihr paralleles Vorgehen eine Reihe von interessanten Fragestellungen auf,
die sich in der Art bei Turingmaschinen nicht ergeben.

Eine der interessantesten und immer noch offenen Fragestellungen in diesem Bereich ist,
ob Linearzeit-Zellularautomaten Sprachen erkennen können, die von Realzeit-Zellularautomaten nicht erkannt werden können.
Ausgehend von dieser Fragestellung werden in dieser Arbeit verschiedene Erweiterungen von Realzeit-Zellularautomaten untersucht,
um besser zu verstehen, wie mächtig diese sind.

Zunächst werden in \cref{chap:Grundlagen} verschiedene Typen von Zellularautomaten definiert, verglichen und teilweise bekannte Resultate
in Entsprechung zu den in dieser Arbeit eingeführten Definitionen neu bewiesen.
Anschließend werden in Vorbereitung auf folgende Kapitel in \cref{chap:LinksunabhAuto}, \cref{chap:SpeedupKonstr} und \cref{chap:ErweiterteNakamuraKonstr}
eine Reihe von interessanten Sätzen gezeigt.
Insbesondere die Konstruktion aus \cref{chap:ErweiterteNakamuraKonstr}
zur Simulation von Realzeit-Zellularautomaten innerhalb von Realzeit mit Rücksetzfunktionalität
erlaubt neuartige Ansätze, um Erweiterungen von Realzeit-Zellularautomaten auf normale Realzeit-Zellularautomaten zurückzuführen.
Solche Erweiterungen werden in \cref{chap:AdvAuto} betrachtet.
In \cref{chap:EingeschrAuto} werden abschließend eingeschränkte Realzeit-Zellularautomaten untersucht, um Einsicht darüber zu erhalten,
wie leicht Realzeit-Zellularautomaten hinsichtlich ihrer Mächtigkeit eingeschränkt werden können.


\chapter{Grundlagen}
\label{chap:Grundlagen}

\section{Notation}

In diesem Abschnitt werden Besonderheiten der in dieser Arbeit verwendeten Notation geklärt.

Sofern nicht anders vermerkt, bezeichnen $\Sigma$ und $\Gamma$ beliebige endliche Mengen, auch Alphabete genannt.

In dieser Arbeit werden die Zeichen eines Wortes mit 1 beginnend durchnummeriert.
Für ein Wort $w \in \Sigma^*$ gilt also die Identität $w = w_1w_2...w_{|w|}$.
Der Ausdruck $w[m...n]$ meint das Teilwort von $w$, das die Zeichen $w_m$, $w_{m+1}$, ... bis einschließlich $w_n$ umfasst.

$\mathcal{P}(M)$ bezeichne die Potenzmenge einer Menge, also die Menge aller ihrer Teilmengen.

Weiter bezeichne $\B := \finset{ \chr{0}, \chr{1}}$ das Binär-Alphabet.

Wenn ein Tupel $T$ durch $T := (A, B, ..., Z)$ definiert wird, wobei $A$, $B$, ..., $Z$ symbolische Variablen sind,
die sich auf einen entsprechenden Wert beziehen,
bezeichnen $A_T$, $B_T$, ..., $Z_T$ die Komponenten des Tupels $T$. Es gilt dann offensichtlich: $T = (A_T, B_T, ..., Z_T)$.

Für eine Fallunterscheidung
\[
    x = \begin{cases}
        1 & \text{falls } A, \\
        2 & \text{sonst, falls } B, \\
        3 & \text{sonst, falls } C, \\
        4 & \text{sonst.}
    \end{cases}
\]
meint ein \enquote{sonst}, dass alle darüberstehenden Fälle nicht zutreffen.
So ist $x = 2$, falls $B$ gilt, aber nicht $A$ und $x = 3$, falls weder $A$ noch $B$ gilt, aber $C$.

\section{Definition Zellularautomat}

In diesem Abschnitt werden grundlegende Objekte und Definitionen im Zusammenhang mit den in dieser Arbeit betrachteten Zellularautomaten eingeführt.

\begin{definition}[Zellularautomat]
    Ein eindimensionaler Zellularautomat mit einer Nachbarschaft von Radius 1 wird in dieser Arbeit als Tupel $(Q, \delta)$ definiert,
    wobei $Q$ eine endliche, nicht-leere Menge ist und
    $\delta$ eine Abbildung mit $\delta: Q^3 \to Q$.
    $Q$ wird auch Zustandsmenge und $\delta$ lokale Überführungsfunktion genannt.
    Es bezeichne $\CA$ die Menge aller solcher Tupel.
    In dieser Arbeit werden nur solche Zellularautomaten betrachtet, weswegen die Dimension und Nachbarschaft im Folgenden nicht mehr erwähnt wird.
\end{definition}

Es werden nun Begriffe eingeführt, mithilfe derer bestimmte Verhaltensweisen von Zuständen in Zellularautomaten gefordert werden können.

\begin{definition}
    Sei $(Q, \delta) \in \CA$ ein Zellularautomat.
    \begin{itemize}
        \item $R \subseteq Q$ heißt passive Zustandsmenge, wenn $\forall a, b, c \in R: \delta(a, b, c) = b$.
        \item $q \in Q$ heißt passiver Zustand, wenn $\finset{q}$ eine passive Zustandsmenge ist.
        \item $R \subseteq Q$ heißt $\delta$-abgeschlossene Menge, wenn $\forall b \in R: \forall a, c \in Q: \delta(a, b, c) \in R$.
        \item $q \in Q$ heißt toter Zustand, wenn $\finset{q}$ eine $\delta$-abgeschlossene Menge ist.
        \item $\delta$ heißt links \acs{bzw.} rechts-unabhängig,
            wenn $\forall a_1, a_2, b, c \in Q: \delta(a_1, b, c) = \delta(a_2, b, c)$
            \acs{bzw.}
            wenn $\forall a, b, c_1, c_2 \in Q: \delta(a, b, c_1) = \delta(a, b, c_2)$.
        \item $q \in Q$ heißt initial, wenn
            $\delta(a, b, c) = q \Rightarrow b = q$.
    \end{itemize}
\end{definition}

Die globale räumliche Situation eines Zellularautomaten zu einem festen Zeitpunkt wird mithilfe einer Konfiguration beschrieben.

\begin{definition}[Konfiguration]
    Eine Konfiguration über einer endlichen Zustandsmenge $Q$ ist eine Abbildung $c \in Q^{\mathbb{Z}}$.
    Der Zustand an Position $i$ einer Konfiguration $c$ wird durch $c_i := c(i)$ beschrieben.
\end{definition}

Die lokalen Regeln eines Zellularautomaten bestimmen, wie aus einer Konfiguration eine Folgekonfiguration entsteht.

\begin{definition}[Folgekonfiguration]
    Die Funktion $\Delta_C$ bildet in Abhängigkeit eines
    Zellularautomaten $C = (Q, \delta) \in \CA$ eine Konfiguration $c$ auf die Folgekonfiguration $\Delta_C(c)$ ab:
    \[ 
        \Delta_C: Q^\Z \to Q^\Z, \;
        (\Delta_C(c))(i) := \delta(c_{i - 1}, c_i, c_{i + 1})
    \]
    
    $\Delta_C^t(c)$ bildet eine Konfiguration $c$ auf die $t$-te Folgekonfiguration für ein $t \in \Nz$ ab.
    Es gilt $\Delta_C^0 = \mathrm{id}$ und $\Delta_C^t = \Delta_C \circ \Delta_C^{t-1}$.
\end{definition}

Ausgehend von einer Konfiguration und dem Begriff einer Folgekonfiguration kann ein Zellularautomat nun verwendet werden,
um ein Berechnungsmodell zu bilden. Zunächst werden aber Objekte eingeführt, um von verschiedenen solcher Modelle abstrahieren zu können.

\begin{definition}[$\Sigma^*$ erkennender Zellularautomat]
    Ein $\Sigma^*$ erkennender Zellularautomat ist ein Tupel $C := (Q, \delta, \Sigma, \#, L, S)$ 
    mit $(Q, \delta) \in \CA$, $\Sigma \subseteq Q$, $\# \in Q \setminus \Sigma$ und $L \subseteq \Sigma^*$ beliebig.
    $S$ ist ein beliebiges Objekt derart, dass die Klasse aller solcher $S$ eine Menge bildet.
    
    \# wird bei der Einbettung von Wörtern in Konfigurationen als Rand verwendet.
    Im Gegensatz zu üblichen Definitionen von Zellularautomaten zur Erkennung von Sprachen, die etwa fordern, dass $\#$ passiv oder gar tot ist,
    gibt es in dieser Definition keine Einschränkungen.
    Es wird später gezeigt, dass $\#$ immer passiv und in den in dieser Arbeit relevanten Fällen auch tot gewählt werden kann.

    Es bezeichne $\ECA$ die Menge aller solcher Tupel. Wegen der Forderung an $S$ ist diese Menge wohldefiniert, aber nicht notwendigerweise abzählbar.
    Durch Spezialisierung der Sprache $L$, die in dieser Definition keiner Einschränkung unterliegt,
    und dem ebenfalls frei wählbaren Objekt $S$ werden im Folgenden verschiedene Typen von Zellularautomaten eingeführt.
    Die Menge $\ECA$ erlaubt es, von diesen Spezialisierungen zu abstrahieren.
    
    Für eine Teilmenge $M \subseteq \ECA$ definiere die von $M$ erzeugte Sprachklasse:
    \[
        \mathcal{L}(M) := \set{ L_C }{ C \in M }
    \]
\end{definition}

\begin{definition}[$\ECA$-Funktor]
    Eine Funktion $\mathcal{F}: \Pot(\ECA) \to \Pot(\ECA)$ heißt $\ECA$-Funktor.
    Seien $\mathcal{F}_1, ..., \mathcal{F}_n$ $\ECA$-Funktoren. Definiere:
    \[
        \ECA^{ \mathcal{F}_1 \text{-...-} \mathcal{F}_n } := (\mathcal{F}_n \circ ... \circ \mathcal{F}_1)(\ECA)
    \]
    
    Im Laufe dieser Arbeit werden eine Reihe solcher Funktoren eingeführt.
    Funktoren ermöglichen es, bequem eine Klasse von Zellularautomaten zu spezialisieren oder in eine andere zu transformieren.
\end{definition}

Bisher wurde noch nicht beschrieben, wie Zellularautomaten mit endlichen Wörtern interagieren. Dazu werden diese zu einer Konfiguration fortgesetzt.

\begin{definition}[Einbettung von Wörtern in Konfigurationen]
   Für einen erkennenden Zellularautomaten $(Q, \delta, \Sigma, \#, L, S) \in \ECA$
   lässt sich jedes Wort $w \in \Sigma^*$ wie folgt zu einer Konfiguration fortsetzen:
   \[
       [w]: \Z \to Q, \;
       [w](p) :=
       \begin{cases} 
          w_p & p \geq 1 \land p \leq |w|, \\
          \# & \text{sonst.}
       \end{cases}
   \]
\end{definition}

\begin{exmp}[Konfiguration]
    Für $w = \chr{abc}$ wird $[w]$ durch folgende Tabelle dargestellt:
    
    \begin{center}
        \addvbuffer[8pt 8pt]{\begin{tabular}
           { | c | c | c | c | c | c | c | c | c | c | c | c | c  | c  | c |}\hline
             $p$ & ... & -2 & -1 & 0 & 1 & 2 & 3 & 4 & 5 & 6 & ... \\ \hline
          $[w]_p$ & ... & \# & \# & \# & \chr{a} & \chr{b} & \chr{c} & \# & \# & \# & ... \\ \hline
        \end{tabular}}
    \end{center}
    
    Die Position einer Zelle in einer Konfiguration wird hier durch die Variable $p$ beschrieben.
\end{exmp}


\subsection{Definition \texorpdfstring{$F$}{F}-haltender Zellularautomat}

Ausgehend von den eingeführten Objekten kann nun ein erstes Berechnungsmodell zum konkreten Erkennen von Sprachen definiert werden.

\begin{definition}[$F$-haltender Zellularautomat]
    Es sei ein Zellularautomat $C := (Q, \delta, \Sigma, \#, L, (F, F^+, p)) \in \ECA$
    mit $F \subseteq Q$, $F^+ \subseteq F$ und $p : \Nz \to \Nz$, $p(0) = 0$ gegeben.
    
    Anschaulich gesprochen repräsentiert $F$ die Menge der finalen Zustände, $F^+$ die Menge der akzeptierenden finalen Zustände und
    $F^- := F \setminus F^+$ die Menge der nicht-akzeptierenden finalen Zustände.
    $p$ ist eine Funktion, die abhängig von der Länge des Eingabeworts die Position der Zelle
    angibt, welche anzeigt, ob das Eingabewort akzeptiert wird.
    
    
    Für ein Wort $w \in \Sigma^*$ gibt die Funktion $\ctime_{C}$ an, wie viele Schritte der Zellularautomat $C$ braucht, um das Wort $w$ zu akzeptieren:
    \[
        \ctime_{C}: \Sigma^* \to \Nz \cup \{ \infty \}, \; w \mapsto \min (\set{ t \in \Nz }{ \Delta_C^t([w])_{p(|w|)} \in F } \cup \{ \infty \})
    \]
    
    Dies lässt sich auf die Wortlänge übertragen:
    \[
        \cworsttime_{C}: \Nz \to \Nz \cup \{ \infty \}, \;  n \mapsto \max \set{ \ctime_{C}(w) }{ w \in \Sigma^n }
    \]
    
    $C$ heißt \emph{$F$-haltend}, wenn $L = L'$ mit
    \[
        L' := \set{ w \in \Sigma^* }{ 
            \ctime_{C}(w) \neq \infty \wedge \Delta_C^{\ctime_{C}(w)}([w])_{p(|w|)} \in F^+ }
    \]
    Das $F$ in $F$-haltend ist symbolisch gemeint und bezieht sich nicht auf den Wert der Menge $F$.
    
    Mit folgender Definition des $\ECA$-Funktors $\CAF$ bezeichnet $\ECA^\CAF$ die Menge aller $F$-haltenden Zellularautomaten:
    \[
        \CAF(M \subseteq \ECA) := \set{ C \in M }{ C \text{ ist $F$-haltend} }
    \]
    
    Ist $\mathcal{F}$ ein $\ECA$-Funktor, sodass die Menge $\set{p_C}{C \in \ECA^{\CAF \text{-} \mathcal{F}}}$ abzählbar ist,
    dann ist auch die Menge $\ECA^{\CAF \text{-} \mathcal{F}}$ abzählbar.
    Die später eingeführten Funktoren $\CALeft$, $\CARight$ und $\CAMid$ haben diese Eigenschaft.
\end{definition}

\begin{comment}
\begin{definition}[Entscheider]
    Ein $F$-haltender Zellularautomat $C$ heißt Entscheider, wenn $\cworsttime_C(\Nz) \subseteq \Nz$,
    $\ctime_C$ also nie den Wert $\infty$ annimmt.
\end{definition}
\end{comment}

\subsection{Definition \texorpdfstring{$t$}{t}-haltender Zellularautomat}

Analog werden die $t$-haltenden Zellularautomaten definiert, bei denen keine finalen Zustände das Ende der Berechnung signalisieren,
sondern beliebige Funktionen. Auf die Unterschiede zu $F$-haltenden Automaten wird in \cref{sec:Vergleich_t_haltend_F_haltend} eingegangen.

\begin{definition}[$t$-haltender Zellularautomat]
    Gegeben ein Zellularautomat $C := (Q, \delta, \Sigma, \#, L, (F^+, t, p)) \in \ECA$
    mit $F^+ \subseteq Q$ und $t, p: \Nz \to \Nz$, $t(0) = 0$, $p(0) = 0$.
    $F^+$ repräsentiert die Menge der akzeptierenden Zustände und $t$ und $p$ Funktionen,
    die abhängig von der Länge des Eingabeworts angeben,
    wie viele Schritte auszuführen sind \acs{bzw.} welche Zelle entscheidet, ob das Wort zu akzeptieren ist.
    
    $C$ heißt \emph{$t$-haltend}, wenn $L = L'$ mit
    \[
        L' := \set{ w \in \Sigma^+ }{ \Delta_C^{t(|w|)}([w])_{p(|w|)} \in F^+ }
    \]
    Das $t$ in $t$-haltend ist symbolisch gemeint und bezieht sich nicht auf die konkrete Funktion $t$.

    Mit folgender Definition des $\ECA$-Funktors $\CAT$ bezeichnet $\ECA^\CAT$ die Menge aller $t$-haltender Automaten:
    \[
        \CAT(M \subseteq \ECA) := \set{ C \in M }{ C \text{ ist $t$-haltend} }
    \]
\end{definition}



\subsection{Verschiedene Funktoren}

Es werden nun verschiedene Funktoren definiert, über die $t$- oder $F$-haltende Automaten weiter eingeschränkt werden können.

\begin{definition}[$\CALeft$-, $\CAMid$- und $\CARight$-\ECA-Funktoren]
    Ein $t$- oder $F$-haltender Zellularautomat $C$ heißt
    \begin{itemize}
        \item \emph{links-erkennend}, falls $p_C(n) = 1$ (definiert $\ECA$-Funktor $\CALeft$),
        \item \emph{mittig-erkennend}, falls $p_C(n) = \lceil n / 2 \rceil$ (definiert $\ECA$-Funktor $\CAMid$) und
        \item \emph{rechts-erkennend}, falls $p_C(n) = n$ (definiert $\ECA$-Funktor $\CARight$).
    \end{itemize}
\end{definition}


\begin{definition}[$\mathrm{time}$-, $\mathrm{\CART}$-, $\mathrm{\CALT}$- und $\mathrm{Mid}$-\ECA-Funktoren]
    \label{def:ZeitFunktoren}
    Sei $F \subseteq {\Nz}^{\Nz}$ eine Menge von Funktionen. Definiere:
    \[
        \mathrm{time}(F)(M \subseteq \ECA) := \set{ C \in \CAF(M) }{ \cworsttime_C \in F } \; \union \; \set{ C \in \CAT(M) }{ t_C \in F }
    \]
    
    Mit folgenden Definitionen ist $\CAM{\CALeft, \CART}$ \acs{bzw.} $\CAM{\CALeft, \CALT}$ die Menge der links-erkennenden Echtzeit- \acs{bzw.} Linearzeit-Zellularautomaten:

    \begin{itemize}
        \item $\mathrm{\CART} := \mathrm{time}(\finset{n \mapsto n - 1}) \circ \CAT$
        \item $\mathrm{\CALT} := \mathrm{time}(O(n)) \circ \CAF$
        \item $\mathrm{Mid} := \mathrm{time}(\finset{n \mapsto \lfloor n/2 \rfloor}) \circ \CAT \circ M $
    \end{itemize}   
\end{definition}

Echtzeit-Zellularautomaten werden über $t$-haltende Automaten definiert, da ein
$F$-haltender Automat bei einem Wort $w$ als Eingabe offensichtlich nicht in Zeit $|w|-1$ halten kann -
die erste Zelle sieht zu diesem Zeitpunkt noch nicht das Ende des Wortes.
$t$-haltende Automaten hingegen können bestimmte Eigenschaften eines Wortes $w$ in Zeit $|w|-1$ erkennen
und erkennen diese Eigenschaft damit gleichzeitig auch für alle Präfixe.

Linearzeit-Zellularautomaten werden über $F$-haltende Automaten definiert,
da in der Menge $O(n)$ unberechenbare Funktionen enthalten sind.
$F$-haltende Automaten müssen selbstständig ihr Berechnungsende angeben, sodass
unberechenbare Zeiteinschränkungen nicht problematisch werden.
Diese Thematik wird in \cref{sec:Vergleich_t_haltend_F_haltend} weiter untersucht.

\begin{definition}[$\LOCA$- und $\ROCA$-\ECA-Funktoren]
    Der $\LOCA$ \acs{bzw.} $\ROCA$ Funktor wählt diejenigen links- \acs{bzw.} rechts-erkennenden Automaten aus, deren Zustandsübergangsfunktion links- \acs{bzw.} rechts-unabhängig ist.
\end{definition}

\section{Vergleich von \texorpdfstring{$t$}{t}-haltenden und \texorpdfstring{$F$}{F}-haltenden Automaten}
\label{sec:Vergleich_t_haltend_F_haltend}

Es folgt ein Lemma für $F$-haltende Automaten. Es zeigt, dass solche Automaten stets so umgebaut werden können,
dass wenn sie einmal einen akzeptierenden oder ablehnenden Zustand einnehmen, diesen nicht mehr verlassen, sich also nicht mehr umentscheiden können.
Ganz trivial ist das Lemma nicht, schließlich können Zustände aus $F$ auch während der Berechnung in Zellen angenommen werden,
die von der Funktion $p$ nicht ausgewählt werden.

\begin{lemma}[$F^+$ und $F^-$ können $\delta$-abgeschlossen gewählt werden]
    \label{lemmaAbgeschlosseneMenge}
    
    Sei $C = (Q, \delta, \Sigma, \#, L, (F, F^+, p))$ ein $F$-haltender Zellularautomat.
    Dann existiert ein $F$-haltender Zellularautomat $C' = (Q', \delta', \Sigma, \#, L', (F', F'^+, p))$ mit $\ctime_{C} = \ctime_{C'}$ und $L = L'$,
    wobei $F'^+$ und $F'^-$ $\delta$-abgeschlossene Mengen sind.
\end{lemma}
\begin{proof}
    Setze $Q' := Q \times \finset{{-1}, 0, 1}$ und identifiziere $\Sigma$ mit $\Sigma \times \finset{0}$ und $\#$ mit $(\#, 0)$.
    Wähle $\delta'$ wie folgt:
    \[
       \delta'((a, f_a), (b, f_b), (c, f_c)) :=
       \begin{cases}
         (\delta(a, b, c), f_b)  & \text{falls } f_b \neq 0, \\
         (\delta(a, b, c), {-1}) & \text{falls } f_b = 0 \text{ und } \delta(a, b, c) \in F^-, \\
         (\delta(a, b, c), {1}) & \text{falls } f_b = 0 \text{ und }  \delta(a, b, c) \in F^+, \\
         (\delta(a, b, c), {0}) & \text{sonst.}
       \end{cases}
    \]
    
    Setze $F' := Q \times \finset{{-1}, 1}$ und $F'^+ := Q \times \finset{1}$.
    Dann sind die Mengen $F'^+$ und $F'^-$ offensichtlich $\delta$-abgeschlossen.
    
    Sei $w \in \Sigma^*, p \in \Z, t \in N_0$ und $(q, f) := \Delta_{C'}^t([w])_p$.
    Dann gilt $q = \Delta_{C}^t([w])_p$. $(q, f) \not\in F'$ impliziert $f = 0$ und damit, dass $\forall k \in \finset{ 0, ..., t }: \Delta_{C}^k([w])_p \not\in F$.
    Anderseits folgt aus $(q, f) \in F'$, dass es ein $k \in \finset{0, ...,  t}$ gibt, sodass $\Delta_{C}^k([w])_p \in F$.
    Da der Automat bei finalen Zuständen stoppt, gilt dann wegen der ersten Überlegung, dass $k = t$.
    Daraus folgt, dass $\ctime_{C} = \ctime_{C'}$ und $L = L'$.
\end{proof}


$t$-haltende Automaten müssen in Abhängigkeit zur Wortlänge stets eine Position und einen Zeitpunkt angeben,
zu dem feststehen muss, ob das Wort akzeptiert oder abgelehnt werden soll.
$F$-haltende Automaten haben die Möglichkeit, nicht zu terminieren.
Da die Funktion zur Wahl des Zeitpunktes bei einem $t$-haltenden Automaten nicht berechenbar sein muss,
sind $t$-haltende Automaten trotzdem nicht weniger mächtig als $F$-haltende, wie der folgende Satz zeigt.

\begin{satz}[$t$-haltende Automaten sind mindestens so mächtig wie $F$-haltende]
    Sei $C = (Q, \delta, \Sigma, \#, L, (F, F^+, p))$ ein $F$-haltender Zellularautomat.
    Dann gibt es einen $t$-haltenden Zellularautomat $C'$ mit $L_{C'} = L_C$, $p_{C'} = p_C$ und $\forall n \in \Nz: t_{C'}(n) \leq \cworsttime_C(n)$.
\end{satz}
\begin{proof}
    Wegen~\cref{lemmaAbgeschlosseneMenge} sind \acs{OBdA.} $F^+$ und $F^-$ $\delta$-abgeschlossene Mengen.
    
    Betrachte folgende Funktion $t$:
    \[
        t: \Nz \to \Nz, \; n \mapsto \max(\finset{ 0 } \cup \set{ \ctime_C(w) }{ w \in \Sigma^n, \; \ctime_C(w) \neq \infty })
    \]

    Wähle $C' := (Q, \delta, \Sigma, \#, L', (F^+, t)) \in \ECA^\CAT$.
    Es gilt offensichtlich $\forall n \in \Nz: t(n) \leq \cworsttime_C(n)$.
    
    Sei $w \in L_C$. Dann gilt $\ctime_C(w) \neq \infty$ und $\Delta_{C}^{\ctime_C(w)}([w])_{p(|w|)} \in F^+$.
    Wegen $t(|w|) \geq \ctime_C(w)$ und weil $F^+$ $\delta$-abgeschlossen ist, folgt $\Delta_{C'}^{t(|w|)}([w])_{p(|w|)} \in F^+$,
    also $w \in L_{C'}$.
    
    Sei $w \in L_{C'}$. Dann gilt $f_1 := \Delta_{C'}^{t(|w|)}([w])_{p(|w|)} \in F^+$.
    Dann ist $\ctime_C(w) \leq t(|w|)$, da in diesem Fall $t(|w|) = 0 \Rightarrow \ctime_C(w) = 0$.
    Nach Definition von $\ctime$ folgt $f_2 := \Delta_{C}^{\ctime_C(w)}([w])_{p(|w|)} \in F$.
    Wäre $f_2 \in F^-$, dann wäre auch $f_1 \in F^- = F \setminus F^+$, da auch $F^-$ $\delta$-abgeschlossen ist.
    Also folgt $f_2 \in F^+$. Also $w \in L_{C'}$.
    
    Insgesamt ergibt sich $L_C = L_{C'}$.
\end{proof}

\begin{remark}
    Das $t$ aus dem vorherigen Beweis ist im Allgemeinen tatsächlich auch nicht berechenbar - andernfalls
    wären semientscheidbare Sprachen stets entscheidbar, da sich Turingmaschinen auf Zellularautomaten reduzieren lassen und umgekehrt.
\end{remark}

Unter bestimmten Annahmen an die Funktion $t$ zur Zeitauswahl können allerdings $t$-haltende Automaten auch zu $F$-haltenden Automaten umgebaut werden. Die Funktion $t(n) := n - 1$ aus \cref{def:ZeitFunktoren} erfüllt diese Annahmen nicht, wohl aber die Funktion $t(n) := n$.

\begin{satz}[$F$-haltende Automaten sind unter Annahmen mindestens so mächtig wie $t$-haltende]
    Sei $C$ ein $t$-haltender Zellularautomat.
    Wenn es einen $F$-haltenden Zellularautomaten $X$ gibt
    mit $p_X = p_C$, $|\Sigma_X| = 1$ und $\cworsttime_X = t_C$,
    dann gibt es auch einen $F$-haltenden Zellularautomat $C'$ mit $L_{C'} = L_C$, $p_{C'} = p_P$ und $\cworsttime_{C'} = t_C$.
\end{satz}
\begin{proof}
    Führe $X$ und $C$ parallel aus, indem $\delta_{C'}$ durch $\delta_X$ und $\delta_C$ auf $Q_{C'} := Q_X \times Q_C$ definiert wird.
    
    Benutze $X$, um zu erkennen, wann eine Zelle in einen Final-Zustand gehen soll,
    indem $F_{C'} := \set{ (q_1, q_2) }{ (q_1, q_2) \in Q_X \times Q_C, \; q_1 \in F_X } $ gewählt wird.
    
    Benutze $C$, um zu erkennen, ob dieser Final-Zustand akzeptierend oder ablehnend sein soll,
    indem $F_{C'}^+ := \set{ (q_1, q_2) }{ (q_1, q_2) \in Q_X \times Q_C, \; q_1 \in F_X, q_2 \in F_C^+} $ gewählt wird.
    Der so definierte Automat $C'$ erfüllt die Behauptung.
\end{proof}

Folgende Proposition zeigt, dass die Annahmen aus dem vorherigen Satz notwendig sind.

\begin{proposition}[$t$-haltende Automaten sind mächtiger als $F$-haltende]
    Es gibt einen $t$-haltenden links-erkennenden Zellularautomaten mit
    $t(n) \in \finset{0, 1}$,
    welcher eine nicht-semientscheidbare Sprache erkennt.
    Diese Sprache kann offensichtlich nicht
    von einem $F$-haltenden Automaten erkannt werden,
    da sie sonst semientscheidbar wäre.
\end{proposition}
\begin{proof}
    Sei $L \subseteq \Sigma^*$ eine nicht-semientscheidbare Sprache und $\mathrm{cod}: \Nz \to \Sigma^*$ eine berechenbare surjektive Funktion.
    Dann ist $L' := \set{ w \in \Sigma^* }{ \mathrm{cod}(|w|) \in L }$ auch nicht-semientscheidbar.
    Setze $t(n) := 1_{cod(n) \in L}$ und $p(n) := 1$.
    Konstruiere $t$-haltenden Zellularautomat mit $\Sigma \cap F^+ = \emptyset$ und $\delta(\cdot, \cdot, \cdot) \in F^+$.
    Dieser Automat erkennt dann offensichtlich $L'$.
\end{proof}

\begin{comment}
    \begin{proposition}
        Sei $C$ ein $F$-haltender links-erkennender Zellularautomat und $n_0 \in \N$ so, dass $\cworsttime_C(n_0) < n_0$.
        Dann $\forall n \geq n_0: \cworsttime_C(n) = \cworsttime_C(n_0)$.
    \end{proposition}
    \begin{proof}
        Bei der Eingabe eines Wortes $w$ der Länge $n_0$ merkt $C$ nicht mehr, wenn $w$ verlängert wird.
        Ausführlicherer Beweis folgt.
    \end{proof}
\end{comment}

\section{Nützliche Eigenschaften}

Die folgenden zwei Sätze sind äußerst wichtig, um den Rand von Zellularautomaten in den Griff zu kriegen.
Üblicherweise wird schon in der Definition von Zellularautomaten ein toter Rand gefordert, was einige elegante Konstruktionen verhindert.
Es wird zunächst gezeigt, dass der Rand ohne Beschränkung der Allgemeinheit passiv und initial gewählt werden kann.

\begin{satz}[Wahl eines passiven und initialen Randes]
    Sei $C = (Q, \delta, \Sigma, \#, L, (F, F^+, p)) \in \ECA^\CAF$.
    Dann kann $C$ so zu $C'$ mit $L(C) = L(C')$ umgebaut werden, dass $\#_{C'}$ ein passiver und initialer Zustand ist.
    Gleichzeitig bleibt dabei die Links- \acs{bzw.} Rechts-Unabhängigkeit von $\delta_C$ erhalten.
\end{satz}
\begin{proof}
    Setze $Q_{C'} := Q^2 \cup \finset { \#_{C'} }$. Identifiziere $\Sigma$ mit $ \Sigma \times \finset{\#}$ und $\#_{C'}$ mit $\Spvek{\perp; \perp}$, $\perp \not\in Q$.

    Für $q \in Q \cup \finset{\perp}$ und $p \in Q$ definiere folgende abkürzende Notation $q^p \in Q$:
    \[
        q^p := 
        \begin{cases}
            q & \text{falls } q \neq \perp, \\
            p & \text{sonst.}
        \end{cases}
    \]
    
    Es sei $\delta_{C'}$ definiert durch:
    \[
       \delta'(\Spvek{a_1; a_2}, \Spvek{b_1; b_2}, \Spvek{c_1; c_2}) :=
       \begin{cases}
         \Spvek{  \delta(a_1 ^ {x}, b_1 ^ {x}, c_1 ^ {x}); \delta(x, x, x)  } 
         & \text{falls }
         \finset{x} = \finset{a_2, b_2, c_2} \setminus \finset{ \perp }, \\
         \#'
         & \text{sonst.}
       \end{cases}
    \]
    Falls $\delta$ links-unabhängig ist, streiche $a_2$ aus der Menge in obiger Definition.
    Falls $\delta$ rechts-unabhängig ist, streiche $c_2$.
    
    Offensichtlich ist nach dieser Konstruktion $\#_C'$ sowohl passiv als auch initial.
    
    Setze $t_r := t_l := t$.
    Falls $\delta$ links-unabhängig ist, setze jedoch $t_l := 0$.
    Falls $\delta$ rechts-unabhängig ist, setze stattdessen $t_r := 0$.
    
    Es reicht nun, für alle $t \in \Nz$ und $i \in \Z$ Folgendes zu zeigen:
    \[
        \Delta_{C'}^t([w])_i = \begin{cases}
            \Spvek{\Delta_C^t([w])_i; \Delta^t_{C}([\varepsilon])} &
                \text{falls } -t_r < i \leq t_l + |w|, \\
            \#' & \text{sonst.}
        \end{cases}
    \]
    
    Definiere $c^t_i := \Delta_{C'}^t([w])_{i-1}$.
    Die Behauptung gilt offensichtlich für $t = 0$.
    Die Behauptung gelte nun für ein $t \in \Nz$.
    Angenommen, $-(t+1)_r < i \leq (t+1)_l + |w|$.
    Dann folgt nach Induktionsvermutung und Definition von $\delta'$: \[
    \finset{ (c^t_{i-1})_2, (c^t_{i})_2, (c^t_{i+1})_2} \setminus \finset{ \perp } = \finset{ \Delta^t_{C}([\varepsilon])} =: \finset{x}
    \]
    
    Angenommen, $i - 1 \leq -t_r$. Dann folgt mit der Induktionsvermutung $((c^t_{i-1})_1)^x = \Delta^t_{C}([\varepsilon]) = \Delta^t_{C}([w])_{i-1}$.
    Ansonsten $((c^t_{i-1})_1)^x = \Delta^t_C([w])_{i-1}$.
    
    Angenommen, $i + 1 > t_l + |w|$. Dann $((c^t_{i+1})_1)^x = \Delta^t_{C}([\varepsilon]) = \Delta^t_{C}([w])_{i+1}$.
    Ansonsten $((c^t_{i+1})_1)^x = \Delta^t_C([w])_{i+1}$.
    
    Analog gilt für $i = -t_r$, $i = t + 1 + |w|$ und alle anderen $i$ im angenommenen Intervall $(c^t_{i})^x = \Delta^t_C([w])_i$.
    
    Folglich gilt $c^{t+1}_i = \Spvek{\Delta^{t+1}_C([w])_i; \Delta^{t+1}_{C}([\varepsilon])}$.
    
    Angenommen, $i \leq -(t+1)_r$. Dann folgt mit der Induktionsannahme $c^t_{i} = c^t_{i-1} = \#'$.
    Außerdem gilt $c^t_{i+1} = \#'$ oder $\delta$ ist rechts-unabhängig.
    Angenommen, $i > (t+1)_l + |w|$. Auch dann gilt nach Induktionsannahme $c^t_{i} = c^t_{i+1} = \#'$.
    Analog gilt dann $c^t_{i-1} = \#'$ oder $\delta$ ist links-unabhängig.
    
    In allen Fällen gilt dann $c^{t+1}_i = \#'$.
\end{proof}

Für Linearzeit-Zellularautomaten kann der Rand sogar ohne Beschränkung der Allgemeinheit tot gewählt werden.
Da ein toter Rand einen Speicherverbrauch nach sich zieht, der nur linear von der Wortlänge abhängt, folgt aus den Platzhierarchie-Sätzen für Turingmaschinen,
dass der Rand für allgemeine Zellularautomaten nicht tot gewählt werden kann:
Schließlich können sich Zellularautomaten und Turingmaschinen mit polynomiellen Mehraufwand für Zeit und Platz gegenseitig simulieren.

\begin{satz}[Wahl eines toten Randes bei Linearzeit-Zellularautomaten]
    \label{satzRauteTot}
    Sei $C \in \CAM{\CALeft, \CALT}$ ein Linearzeit-Zellularautomat.
    Dann kann $C$ ohne Veränderung der Laufzeit so zu $C'$ mit $L(C) = L(C')$ umgebaut werden, dass $\#_{C'}$ ein toter Zustand ist.
    Für eine passive Zustandsmenge $M \subseteq \Sigma_C$ mit $\#_C \in M$ ist
    die entsprechende Teilmenge von $\Sigma_{C'}$ auch passiv.
\end{satz}
\begin{proof}
    Da $C$ ein Linearzeit-Zellularautomat ist, gibt
    es ein $k \in \Nz$, sodass $\forall n \in \Nz: \cworsttime_C(n) \leq kn$.
    Setze $K := \finset{-k, ..., k}$ und  $Q' := Q_C^K \cup \finset{ \# }$.
    Die Elemente von $f \in Q_C^K$ können auf $\Z$ durch $f(n \in \Z \setminus K) := \#_C$ fortgesetzt werden.

    Für $i \in K$, $q \in Q'$ und $c \in Q_C$ definiere folgende abkürzende Notation:
    \[
        q_i^c :=
        \begin{rcases}
            \begin{dcases}
                c & \text{falls } q = \#, \\
                q(i) & \text{sonst.}
            \end{dcases}
        \end{rcases}
        \in Q_C
    \]

    Definiere $\delta'$ für $a, c \in Q'$ und $b \in Q^K_C$ durch
    
    \begin{alignat*}{2}
        & \delta'(a, \#, c) & := & \; \# \\
        & (\delta'(a, b, c))(i \in K) & := &
        \begin{cases}
            \delta(a_i^{b(i - 1)}, b(i), c_i^{b(i + 1)}) & \text{falls } i \text{ gerade ist,} \\
            \delta(c_i^{b(i - 1)}, b(i), a_i^{b(i + 1)}) & \text{sonst.}
        \end{cases}
    \end{alignat*}
    
    Bette $c \in \Sigma$ durch $\phi(c)$ wie folgt in $Q'$ ein:
    \[
        (\phi(c))(i \in K) :=
        \begin{cases}
            c & \text{ falls } i = 0, \\
            \#_C & \text { sonst.}
        \end{cases}
    \]
    Setze $F' := \set{ q \in Q_C^K }{ q(0) \in F }$
    und $F'^+ := \set{ q \in Q_C^K }{ q(0) \in F^+ }$.
    
    Es bleibt nun die Korrektheit dieser Konstruktion zu zeigen.
    
    Setze $f_{n \in \Nz}(i \in \Z) :=
    \begin{cases} 
        ((i - 1) \; \mathrm{mod} \; n) + 1  & \textrm{ falls } (i \; \mathrm{mod} \; 2n) \leq n, \\
        n - ((i - 1) \; \mathrm{mod} \; n)  & \textrm{ sonst.}
    \end{cases}$
    .
    \newline
    
    Folgende Tabelle zeigt einige Werte von $f_4$:
    \begin{center}
        \addvbuffer[8pt 8pt]{\begin{tabular}
           { | c | c | c | c | c | c | c | c | c | c | c | c | c  | c  | c |}\hline
             $i$ & -1 & 0 & 1 & 2 & 3 & 4 & 5 & 6 & 7 & 8 & 9 & 10 & 11 & 12 \\ \hline
          $f_4(i)$ & 2 & 1 & 1 & 2 & 3 & 4 & 4 & 3 & 2 & 1 & 1 & 2  & 3  & 4  \\ \hline
        \end{tabular}}
    \end{center}
    
    $f_n$ hat einen steigenden und einen fallenden Teil, die Funktion pendelt zwischen $1$ und $n$
    und es gilt stets $|f_n(i) - f_n(i+1)| \leq 1$.
    
    Für die Korrektheit reicht es nun zu zeigen, dass für $w \in \Sigma^*$, $t \in \finset{0, ..., k|w|}$ und $p \in \finset{-k|w|+t, ..., k|w|-t}$ gilt:
    \[
        \Delta_C^t([w])_p
            = (\Delta_{C'}^t([w])_{f_{|w|}(p)})(\lfloor \frac{p - 1}{|w|} \rfloor)
    \]
    Diese Aussage kann leicht induktiv gezeigt werden.
    Dabei müssen die Fälle betrachtet werden,
    dass $f_{|w|}(p+1)$ \acs{bzw.} $f_{|w|}(p+1)$
    kleiner, größer oder gleich $f_{|w|}(p)$ sein kann.
    Abhängig davon befindet sich die Zelle mit Position $p$ dann direkt neben dem Rand $\#_{C'}$ \acs{bzw.}
    entscheidet sich die Parität von $\lfloor \frac{p - 1}{|w|} \rfloor$.
    
    \begin{comment}
    $P_{t \in \Nz, n \in \N} := \finset{-kn+t, ..., kn-t}$.
    
        Sei $w \in \Sigma^*$.
        Beweis der Behauptung durch Induktion über $t$.

        Sei $t = 0$, $p \in \finset{1, ..., |w|}$.
        Dann $\Delta_{C'}^0([w])_p = w_p
        = \phi(w_p)(0)
        = (\Delta_{C'}^t([w])_{p})(0)
        = (\Delta_{C'}^t([w])_{f_{|w|}(p)})(\lfloor \frac{p - 1}{|w|} \rfloor)$.
        Sei nun $p \in P_{0,|w|} \setminus \finset{1, ..., |w|}$.
        Dann $\lfloor \frac{p - 1}{|w|} \rfloor \in K \setminus \finset{0}$ und
        $\Delta_{C'}^0([w])_p = \#_C =
        (\Delta_{C'}^t([w])_{f_{|w|}(p)})(\lfloor \frac{p - 1}{|w|} \rfloor)$.
    
    
        Sei $1 \leq t \leq k|w|$.
        Dann gilt das auch, ist aber mühselig. Am liebsten
        würde ich ja einen formalen, Computer-gestützten Beweis führen...
    \end{comment}
\end{proof}

Der folgende praktische Satz besagt, dass es auf endlich viele Ausnahmen in Sprachen nicht ankommt.
Insbesondere kann damit der Fall ignoriert werden, dass Sprachen das leere Wort enhalten können.
\begin{satz}
    \label{endlichVieleAusnahmen}
    Sei $C$ ein Zellularautomat,
    $L \subseteq \Sigma^*$ eine Sprache und $M \subseteq \Sigma^*$, $|M| \leq \infty$  eine endliche Menge von Wörtern, sodass $L_C \setminus M = L \setminus M$.
    Dann existiert ein Zellularautomat $C'$, der $L$ in der selben Zeit erkennt.
\end{satz}
\begin{proof}
    Offensichtlich sind dann $M_1 := L \cap M$ und $M_2 := (\Sigma^* \setminus L) \cap M$ auch endlich.
    $M_1$ und $M_2$ sind disjunkt und ihre Vereinigung ist genau $M$.
    $M_1$ enthält die Wörter, die $L$ hat, aber $L_C$ möglicherweise nicht, während $M_2$ die Wörter enthält, die $L$ nicht hat, $L_C$ aber möglicherweise schon.
    Da endliche Sprachen regulär sind und reguläre Sprachen von Echtzeit-Zellularautomaten erkannt werden können,
    existieren Zellularautomaten $C_1$ und $C_2$ mit $L_{C_1} = M_1$ und $L_{C_2} = M_2$.
    Akzeptiert weder $C_1$ noch $C_2$ ein Wort $w$, dann akzeptiert es $C$ genau dann, wenn $w \in L$.
    Werden also die Automaten $C$, $C_1$ und $C_2$ parallel ausgeführt, kann leicht ein Echtzeit-Zellularautomat $C'$ konstruiert werden mit $L_{C'} = L$.
\end{proof}
\chapter{Links-unabhängige Zellularautomaten}
\label{chap:LinksunabhAuto}

Zellen in links-unabhängigen Zellularautomaten hängen in ihrer Berechnung nicht von ihrem linken Nachbarn ab.
Informationen können sich also nur nach links verbreiten. Diese Einschränkung vereinfacht bestimmte Konstruktionen, was in \cref{chap:SpeedupKonstr} ausgenutzt wird.
In diesem Kapitel werden Zusammenhänge zwischen unbeschränkten und linksunabhängigen Zellularautomaten gezeigt.
Ähnliche Zusammenhänge werden in \cite{Choffrut1984} gezeigt.

Es ist allgemein bekannt,
dass $\CAL{\CART, \CALeft, \LOCA}$ eine strikte Teilmenge von $\CAL{\CART, \CALeft}$ ist und
dass $\CAL{\CALT, \CALeft, \LOCA}$ mit $\CAL{\CART, \CARight}$ zusammenfällt (siehe \cite{Kutrib2009}).

Der erste Satz zeigt, dass unbeschränkte Zellularautomaten zu einem linksunabhängigen Automaten
umgebaut werden können. Dabei verschiebt sich die Berechnung räumlich nach links und wird um den Faktor 2 langsamer.
\begin{satz}
    \label{zellautoZuLinksunabhaengig}
    Sei $C = (Q, \delta) \in \CA$ ein Zellularautomat (wie in \cref{fig:NormalZuRechtsUnabh}).
    Dann gibt es einen links-unabhängigen Zellularautomaten $C' = (Q' \supseteq Q, \delta')$ (wie in \cref{fig:NormalZuRechtsUnabh2}),
    sodass
    für $\forall c \in Q^{\Z}, \forall t \in \Nz, \forall i \in \Z$ gilt:
    \[
            \Delta_{C'}^{t}(c)_i =
            \begin{cases}
                \Delta_C^{\frac{t}{2}}(c)_{i+\frac{t}{2}} & \text{ falls } $t$ \text{ gerade, } \\
                (\Delta_C^{\frac{t - 1}{2}}(c)_{i+\frac{t - 1}{2}}, \Delta_C^{\frac{t-1}{2}}(c)_{i+\frac{t+1}{2}})  & \text{ sonst.}
            \end{cases}
    \]
        
    \begin{figure}[h!]
        \centering
        \includesvg[width=185pt]{images/NormalZuLinksUnabh_Norm}
        \caption{Teilweise Ausführung des Automaten $C$}
        \label{fig:NormalZuRechtsUnabh}
    \end{figure}
   \begin{figure}[h!]
        \centering
        \includesvg[width=185pt]{images/NormalZuLinksUnabh_Luh}
        \caption{Teilweise Ausführung des links-unabhängigen Automaten $C'$}
        \label{fig:NormalZuRechtsUnabh2}
    \end{figure}

\end{satz}
\begin{proof}
    Führe für diesen Beweis folgende Notation ein: ${c'}^{t}_i := \Delta^{t}_{C'}(c)_i$ und $c^t_i := \Delta^{t}_{C}(c)_i$.
    
    Setze $Q' := Q \cup Q^2$. Definiere $\delta'$ für $ b, c \in Q$ wie folgt:
    \[
        \delta'(\cdot, b, c) := (b, c) \in Q'
    \]
    und für $b, c \in Q \times Q$:
    \[
        \delta'(\cdot, (b_1, b_2), (c_1, c_2)) := \delta(b_1, b_2, c_2)
    \]
    Definiere $\delta(\cdot, b, c)'$ beliebig für alle anderen Fälle. Offensichtlich ist $\delta$ dann links-unabhängig.
    
    Sei $c \in Q^{\Z}$. Beweis der Behauptung durch Induktion über $t$. Für $t = 0$ trivial.
    
    Sei nun $t > 0, i \in \Z$.
    
    1. Fall: $t$ ist ungerade.
    Dann gilt nach Induktionsannahme
    \[
        \forall j \in \Z: {c'}^{t-1}_j = c^{\frac{t-1}{2}}_{j+\frac{t-1}{2}} \in Q
    \]
    
    Und damit
    \begin{align*}
        & {c'}^{t}_i \\
        &= \delta'({c'}^{t-1}_{i-1}, {c'}^{t-1}_i, {c'}^{t-1}_{i+1}) \\
        &= ({c'}^{t-1}_i, {c'}^{t-1}_{i+1})) \\
        &= (c^{\frac{t-1}{2}}_{i+\frac{t-1}{2}}, c^{\frac{t-1}{2}}_{i+\frac{t+1}{2}})
    \end{align*}
    
    2. Fall: $t$ ist gerade. Damit $t \geq 2$.
    Dann gilt nach Induktionsannahme
    \[
        \forall j \in \Z: {c'}^{t-1}_j
            = (c^{\frac{t-2}{2}}_{j+\frac{t-2}{2}}, c^{\frac{t-2}{2}}_{j+\frac{t}{2}}) \in Q \times Q
    \]
    
    Und damit
    \begin{align*}
        & {c'}^{t}_i \\
        &= \delta'({c'}^{t-1}_{i-1}, {c'}^{t-1}_i, {c'}^{t-1}_{i+1}) \\
        &= \delta'(q,  (c^{\frac{t-2}{2}}_{i+\frac{t-2}{2}}, c^{\frac{t-2}{2}}_{i+\frac{t}{2}}),
             (c^{\frac{t-2}{2}}_{i+1+\frac{t-2}{2}}, c^{\frac{t-2}{2}}_{i+1+\frac{t}{2}})) \\
        &= \delta(c^{\frac{t-2}{2}}_{i+\frac{t-2}{2}}, c^{\frac{t-2}{2}}_{i+\frac{t}{2}},
                c^{\frac{t-2}{2}}_{i+1+\frac{t}{2}}) \\
        &= c^{\frac{t}{2}}_{i+\frac{t}{2}}
    \end{align*}
    
\end{proof}

Der nächste Satz zeigt die andere Richtung: Links-unabhängige Zellularautomaten können
wieder zu einem Automaten ohne Einschränkung umgebaut werden. Das allein überrascht nicht, schließlich
sind links-unabhängige Automaten eine Spezialisierung der Zellularautomaten ohne Einschränkung.
Interessant ist jedoch, dass sich dabei die Berechnung nach rechts verschieben und um den Faktor 2 beschleunigen lässt.
\begin{satz}
    \label{linksunabhaengigZuZellauto}
    Sei $C = (Q, \delta) \in \CA$ ein links-unabhängiger Zellularautomat (wie in \cref{fig:LinksunabhZuNormal1} gezeigt).
    Dann gibt es einen Zellularautomaten $C' = (Q' \supseteq Q, \delta')$ (wie in \cref{fig:LinksunabhZuNormal2} gezeigt),
    sodass für $\forall c \in Q^{\Z}$, $\forall t \in \Nz$, $\forall i \in \Z$ gilt:
    \[
        \Delta^t_{C'}(c)_{i} = \Delta^{2t}_C(c)_{i-t}
    \]
    
    \begin{figure}[h!]
        \centering
        \includesvg[width=190pt]{images/LinksunabhZuNormal_Luh}
        \caption{Teilweise Ausführung des links-unabhängigen Automaten $C$}
        \label{fig:LinksunabhZuNormal1}
    \end{figure}
   \begin{figure}[h!]
        \centering
        \includesvg[width=190pt]{images/LinksunabhZuNormal_Norm}
        \caption{Teilweise Ausführung des Automaten $C'$}
        \label{fig:LinksunabhZuNormal2}
    \end{figure}

\end{satz}
\begin{proof}
    Führe für diesen Beweis folgende Notation ein: ${c'}^{t}_i := \Delta^{t}_{C'}(c)_i$ und $c^t_i := \Delta^{t}_{C}(c)_i$.
    
    Da $Q$ nicht leer ist, gibt es ein $q \in Q$.
    Setze $Q' := Q$ und für $a, b, c \in Q'$:
    \[
        \delta'(a, b, c) := \delta(q, \delta(q, a, b), \delta(q, b, c))
    \]
    
    Sei $c \in Q^{\Z}$. Beweis der Behauptung durch Induktion über $t$. Für $t = 0$ trivial. Sei $t > 0$, $i \in \Z$.

    Nach Induktionsannahme gilt $\forall j \in \Z: {c'}^{t-1}_{j+t-1} = c^{2t-2}_j$.
    
    Es folgt:
    \begin{align*}
        & {c'}^{t}_{j+t} \\
        &= \delta'({c'}^{t-1}_{j+t-1}, {c'}^{t-1}_{j+t}, {c'}^{t-1}_{j+t+1}) \\
        &= \delta'(c^{2t-2}_{j}, c^{2t-2}_{j+1}, c^{2t-2}_{j+2}) \\
        &= \delta(q, \delta(q, c^{2t-2}_{j}, c^{2t-2}_{j+1}), \delta(q, c^{2t-2}_{j+1}, c^{2t-2}_{j+2})) \\
        &= \delta(q, c^{2t-1}_j, c^{2t-1}_{j+1} ) \\
        &= c^{2t}_j 
    \end{align*}
\end{proof}

Mit beiden Sätzen zusammen kann ein Automat zunächst zu einem linksunabhängigen Automaten umgebaut,
dann transformiert und wieder zurück umgebaut werden. Dies wird in \cref{CAgfSpeedup} aus dem nächsten Kapitel ausgenutzt.
\chapter{Speedup-Konstruktionen}
\label{chap:SpeedupKonstr}

Bei der Fragestellung, wie Linearzeit-Zellularautomaten zu Echtzeit-Zellularautomaten verhalten,
ist es essentiell, zu untersuchen, inwiefern sich Zellularautomaten beschleunigen lassen.
Sogenannte Speedup-Sätze für Zellularautomaten sind schon einige bekannt (siehe \cite{Kutrib2009, MAZOYER199259}).
Um dem Formalismus dieser Arbeit gerecht zu werden, werden einige dieser Sätze in diesem Kapitel neu bewiesen \acs{bzw.} verallgemeinert.
In \cref{chap:AdvAuto} und \cref{chap:EingeschrAuto} werden die hier vorgestellten Speedup-Konstruktionen wieder aufgegriffen.

\section{Speedup um konstant viele Schritte}

Ein zentraler Speedup-Satz zeigt, dass Zellularautomaten um einen Schritt beschleunigt werden können,
wenn Annahmen über den Rand getroffen werden können. \cref{satzRauteTot} zeigt, dass wir diese Annahmen treffen dürfen.
Eine Besonderheit der hier vorgestellten Konstruktion ist, dass sie in einem gewissen Rahmen passive Zustände erhält.
Das wird in \cref{chap:EingeschrAuto} wieder aufgegriffen.

\begin{definition}[1-Schritt-Speedup-Konstruktion]
    Sei $(Q, \delta) \in \CA$ ein Zellularautomat.
    Definiere den Zellularautomaten $Sp(C) := (Q', \delta')$ mit $Q' := Q \times (Q^Q \cup \finset{ \perp })$.
    Identifiziere $Q$ mit $Q \times \finset{ \perp }$.
    Für ein Element $q \in Q'$ bezeichnen $q_q \in Q$ und $q_f \in Q^Q \cup \finset{ \perp }$ die erste \acs{bzw.} zweite Komponente, sodass gilt: $q = (q_q, q_f)$.
    Die Zustandsübergangsfunktion $\delta'$ sei wie folgt definiert:
    \[
        \delta'((a_q, a_f), (b_q, b_f), (c_q, c_f)) :=
        \begin{cases}
            (\delta(a_q, b_q, c_q), \perp) & \text{falls } c_f = \perp \\
            \phi_{c_f(b_q)}(\delta(a_q, b_q, c_q)) & \text{sonst} 
        \end{cases}
    \]
    
    Definiere weiter $\phi_c: Q \to Q'$ für ein $c \in Q$:
    \[
        \phi_c(b) := (b, \; Q \ni a \mapsto \delta(a, b, c) \in Q)
    \]
    
    \begin{figure}[h!]
        \begin{center}
        \includesvg[width=190pt]{images/KonstanterSpeedup}
        \end{center}
        \caption{1-Schritt Speedup}
        
        Für einen Zustand $q$ beschreiben die eingezeichneten Pfeile die Abbildung $q_f$:
        Der jeweils linke Zustand wird auf den unteren abgebildet, sofern der Zustand rechts
        zum Zeitpunkt $0$ bekannt ist.
        \label{fig:KonstanterSpeedup}
    \end{figure}
\end{definition}

\begin{satz}[Korrektheit der 1-Schritt-Speedup-Konstruktion]
    \label{satzSpeedupConstruction}
    Sei $C = (Q, \delta) \in \CA$ ein Zellularautomat, $\# \in Q$ ein ausgezeichneter Zustand
    und $c \in Q^{\Z}$ eine Konfiguration.
    Setze $C' := Sp(C)$.
    Sei $c' \in {Q_{C'}}^{\Z}$ eine Konfiguration mit $c'_j \in \finset{c_j, \phi_{\#}(c_j)}$ für alle $j \in \Z$. Sei $t \in \Nz$ und $i \in \Z$.
    
    \begin{enumerate}
        \item
            Falls $c'_{t+i} \in Q \times Q^Q$, dann $\Delta^t_{C'}(c')_i \in Q \times Q^Q$.
        \item
            Es gilt: $(\Delta^t_{C'}(c')_i)_q = \Delta^t_{C}(c)_i$.
        \item
            Falls $\Delta^t_{C'}(c')_i \in Q \times Q^Q$ und $c_{i+t+1} = \#$, dann gilt, wie in \cref{fig:KonstanterSpeedup} veranschaulicht:
            \[(\Delta^t_{C'}(c')_i)_f((\Delta^t_{C'}(c')_{i-1})_q) = \Delta^{t+1}_C(c)_i\]
    \end{enumerate}
\end{satz}
\begin{proof}
    Führe für diesen Beweis folgende Notation ein: $c'^{t}_i := \Delta^{t}_{C'}(c')_i$ und $c^t_i := \Delta^{t}_{C}(c)_i$.

    \begin{enumerate}
        \item
            Die Aussage gilt offensichtlich für $t = 0$. Angenommen, die Aussage gilt nun für ein $t \in \Nz$ und alle $i \in \Z$.
            Sei $c'_{t+1+i} \in Q \times Q^Q$. Dann gilt nach Definition von $\delta_{C'}$ und wegen $c'^{t}_{i+1} \in Q \times Q^Q$:
            \[
                c'^{t+1}_i = \delta_{C'}(c'^{t}_{i-1}, c'^{t}_{i}, c'^{t}_{i+1}) \in Q \times Q^Q
            \]
            Damit gilt die Aussage auch für $t + 1$.
            
        \item Gilt offensichtlich.
        
        \item
            Beweis durch Induktion über $t$.
            Für $t = 0$ gilt mit $c'^0_i \in Q \times Q^Q$ und $c_{i+1} = \#$:
            \begin{align*}
                  && (c'^0_i)_f((c'^0_{i-1})_q) \\
                = && (c'_i)_f((c'_{i-1})_q) \\
                = && (\phi_s(c_i))_f(c_{i-1}) \\
                = && \delta(c_{i-1}, c_i, c_{i+1}) = c^{1}_i
            \end{align*}
            
            Es gelte nun die Behauptung für ein $t \in \Nz$ und alle $i \in \Z$.
            Sei $c'^{t+1}_i \in Q \times Q^Q$ und $c_{i+(t+1)+1} = \#$.
            
            Nach Definition von $\delta_{C'}$ gilt $(c'^{t}_{i+1})_f \in Q^Q$.
            Wegen $c_{(i+1)+t+1} = \#$ gilt nach Induktionsvoraussetzung:
            \[
                (c'^t_{i+1})_f((c'^t_{i})_q) = c^{t+1}_{i+1}
            \]
            
            Es gilt:
            \begin{align*}
                  && (c'^{t+1}_i)_f = \delta_{C'}(c'^{t}_{i-1}, c'^{t}_{i}, c'^{t}_{i+1})_f \\
                = && \phi_{(c'^{t}_{i+1})_f((c'^{t}_{i})_q)}(\delta((c'^{t}_{i-1})_q, (c'^{t}_{i})_q, (c'^{t}_{i+1})_q))_f \\
                = && \phi_{c^{t+1}_{i+1}}(c^{t+1}_{i})_f \\
                = && Q \ni a \mapsto \delta(a, c^{t+1}_{i}, c^{t+1}_{i+1}) \in Q
            \end{align*}
            
            Und damit:
            \begin{align*}
                  && (c'^{t+1}_i)_f((c'^{t+1}_{i-1})_q) \\
                = && \delta(c^{t+1}_{i-1}, c^{t+1}_{i}, c^{t+1}_{i+1}) \\
                = && c^{t+2}_i
            \end{align*}
            
            
    \end{enumerate}
\end{proof}

Rekursiv lässt sich mithilfe des gezeigten Satzes ein Zellularautomat um konstant viele Schritte beschleunigen.
Wird \cref{satzRauteTot} ausgenutzt, lässt sich die folgende Aussage zeigen.

\begin{satz}[$k$-Schritt Speedup]
    \label{satzEchtzeitSpeedup}
    Es gilt:
    \[
        \CAL{\CART, \CALeft} =
            \CAL{\CAT, \CALeft, \mathrm{time}(\set{ n \mapsto \max \finset{0,  n + k - 1} }{ k \in \Nz })}
    \]
\end{satz}
\begin{proof}
    Es reicht zu zeigen, dass für alle $k \in \Nz$ gilt:
    \[
        \CAL{\CAT, \CALeft, \mathrm{time}(
                    n \mapsto \max \finset{0,  n + k - 1}
                )
            }
        \supseteq
        \CAL{\CAT, \CALeft, \mathrm{time}(
                    n \mapsto n + k
                )
            }
    \]
    Induktiv gilt dann die Behauptung.

    Sei $L \in \CAL{\CAT, \CALeft, \mathrm{time}(
                    n \mapsto n + k  )}$.
    Dann existiert ein $t$-haltender Zellularautomat $C$,
    sodass $\Delta_C^{|w|+k}([w])_1 \in F^+_C \Leftrightarrow w \in L$.
    Wegen \cref{satzRauteTot} kann angenommen werden, dass $\#_C$ ein toter Zustand ist.
    
    Setze $C' := Sp(C)$. Identifiziere $c \in \Sigma_C$ mit $\phi_{\#_C}(c)$ in $\Sigma_{C'}$ und setze $\#_{C'} := \phi_{\#_C}(\#_C)$.
    Definiere $F_{C'}^+$ wie folgt: \[
        F_{C'}^+ := \set{ q \in Q_C \times Q_C^{Q_C} }{ q_f({\#_C}) \in F_C^+ }
    \]
    
    Es kann angenommen werden, dass $|w| + k - 1 \geq 0$, da mit \cref{endlichVieleAusnahmen} das leere Wort ignoriert werden kann.
    Da $[w]_{|w|+k} \in Q_C \times Q_C^{Q_C}$, $[w]_{|w|+k+1} = \#_C$ und $\Delta^{|w|+k-1}_C([w])_0 = \#_C$, folgt mit \cref{satzSpeedupConstruction}
    $\Delta_{C'}^{|w|+k-1}([w])_1 \in F'^+_C
    \Leftrightarrow \Delta_C^{|w|+k}([w])_1 \in F^+_C
    \Leftrightarrow w \in L$.
    Also $L \in \CAL{\CAT, \CALeft, \mathrm{time}(
                    n \mapsto \max \finset{0,  n + k - 1}
                )}$.
\end{proof}


Mithilfe der Speedup-Sätze kann beispielsweise nun einfach gezeigt werden,
dass Wortlängen einer \CAM{\CART, \CALeft}-Sprache \acs{OBdA.} Vielfache einer beliebigen Zahl $k \in \N$ sind:
\begin{satz}
    \label{wrtRealtimeLengthIdeal}
    Sei $L \subseteq \Sigma^*$ eine Sprache, $\chr{x} \not\in \Sigma$, $k \in \N$.
    Definiere zu $L$:
    \[
        L(k) := \set{ w \chr{x}^{\min \set{i \in \Nz }{ (|w| + i) \in k\Nz }} } { w \in L }
    \]
    
    Es gilt: $L(k) \in \CAL{\CART, \CALeft}$ genau dann, wenn $L \in \CAL{\CART, \CALeft}$.
\end{satz}
\begin{proof}
    Angenommen $L(k) \in \CAL{\CART, \CALeft}$.
    Dann existiert ein Echtzeit-Automat $C'$
    mit $L_{C'} = L(k)$.
    Da bei links-erkennenden Echtzeit-Zellularautomaten
    das Akzeptanzverhalten unabhängig vom Verhalten des rechten
    Rands ist, kann der rechte Rand mit \chr{x} identifiziert werden (\acs{OBdA.} kann der rechte Rand vom linken unterschieden werden).
    Dieser Automat erkennt nun $L$
    in $t$ Schritten mit $t \in \finset{n-1, ..., n-1+k-1}$.
    Setze $C'_i$ auf den $i$-Schritt-Speedup-Automaten von $C'$
    und konstruiere den Automaten $C$, der die endlich vielen Automaten
    $C'_0$, $C'_1$ bis $C'_{k-1}$ parallel ausführt und genau
    dann akzeptiert, wenn einer von ihnen akzeptiert.
    Damit erkennt $C$ die Sprache $L$ in Echtzeit.
    
    Umgekehrt sei nun $L \in \CAL{\CART, \CALeft}$.
    Dann existiert ein Echtzeit-Automat $C$
    mit $L_C = L$.
    Konstruiere den Automat $C'$ wie folgt:
    $C'$ löscht in genau $k-1$ Schritten die
    Zellen am rechten Rand im Zustand \chr{x}.
    Sind es mehr als $k-1$ solcher Zellen, lehne das Wort ab, bei weniger
    wird gewartet, bis die $k-1$ Schritte vorbei sind.
    Anschließend wird auf dieser neuen Konfiguration
    nun der Automat $C$ ausgeführt.
    Parallel wird durch einen Automaten $X$ getestet,
    ob die Länge des Wortes ein Vielfaches von $m$ ist.
    Das Eingabewort soll nun genau dann von $C'$ akzeptiert werden,
    falls $C$ und $X$ akzeptieren.
    Dieser Automat $C'$ akzeptiert nun $L(k)$ in $n-1+k-1$
    vielen Schritten.
    Da $k$ fest ist, garantiert \cref{satzEchtzeitSpeedup}
    nun die Existenz eines $L(k)$-erkennenden Echtzeitautomaten.
\end{proof}


\section{Speedup von Echtzeit-Zellularautomaten}

Im vorherigen Abschnitt wurde gezeigt, dass Zellularautomaten,
die nur eine konstante Anzahl Schritte von Echtzeit entfernt sind,
auf Echtzeit beschleunigt werden können.
In diesem Kapitel wird gezeigt, dass sich Echtzeit-Zellularautomaten in gewisser Hinsicht auch beschleunigen lassen.
Offensichtlich ist die Zeiteinschränkung von Echtzeit-Zellularautomaten bereits minimal:
Wenn ein Wort $w$ in weniger als $|w|-1$ Schritten erkannt wird, kann das letzte Zeichen des Eingabewortes nicht zur Akzeptanz beitragen.
Allerdings lässt sich die Anzahl benötigter Schritte tatsächlich weiter verringern,
wenn dafür die Position der Zelle, die Akzeptanz oder Ablehnung signalisiert, in Richtung Mitte wandert.

Zunächst wird aber das folgende Lemma benötigt.

\begin{lemma}
    \label{linksunabhaengigSpeedup}
    Sei $C \in \CA$ ein links-unabhängiger Zellularautomat (wie in \cref{fig:LinUnabhSpeedup1}) und $2 \leq k \in \N$.
    Sei ferner $\# \in Q_C$ ein passiver und initialer Zustand.
    Dann gibt es einen links-unabhängigen Zellularautomaten
    $C' \in \CA$, sodass für $i \in \Z$ mit $i \leq 0$ und $t \in \Nz$
    und alle Konfigurationen $c$ mit $c_p = \#$ für $p \leq 0$ oder $p \geq p_0$ gilt:
    \[
        \Delta^{t}_{C'}(c)_i =
            w \in Q^k
            \text{ mit } w_j := \Delta^{ t + i - ki+k-j }_C(c)_{ ki-k+j }
    \]
    
    Für $k = 2$ erhält man, wie in \cref{fig:LinUnabhSpeedup2} gezeigt:
    \[
        \Delta^{t}_{C'}(c)_i = ( \Delta^{ t - i+1 }_C(c)_{ 2i-1 },
            \Delta^{ t - i }_C(c)_{ 2i })
    \]

    Für $k = 3$ ergibt sich:
    \[
        \Delta^{t}_{C'}(c)_i = ( \Delta^{t-2i+2}_C(c)_{3i-2},
        \Delta^{t-2i+1}_C(c)_{3i-1},
        \Delta^{t-2i}_C(c)_{3i}
        )
    \]
    
    \begin{figure}[!ht]
        \centering
        \includesvg[width=190pt]{images/LinUnabhSpeedup}
        \caption{Teilweise Ausführung des Automaten $C$}
        \label{fig:LinUnabhSpeedup1}
    \end{figure}
    \begin{figure}[!ht]
        \centering
        \includesvg[width=190pt]{images/LinUnabhSpeedup2}
        \caption{Teilweise Ausführung des Automaten $C'$ für $k = 2$}
        \label{fig:LinUnabhSpeedup2}
    \end{figure}
    
\end{lemma}
\begin{proof}
    Definiere $\delta^{w} \in Q^{|w-1|}$ für $w \in Q^*$ mit $|w| \geq 2$ wie folgt:
    \begin{align*}
        \delta^{w}_i := \begin{cases}
            \delta(\#, w_{|w|-1}, w_{|w|}) & \text{falls } i = |w| - 1 \\
            \delta(\#, w_{i}, \delta^{w}_{i+1}) & \text{sonst}
        \end{cases}
    \end{align*}

    Beispielsweise gilt $\delta^{q_1q_2q_3} = \delta(\#, q_1, \delta(\#, q_2, q_3))\delta(\#, q_2, q_3)$.
    
    Setze $Q' := Q_C \cup Q_C^k$. Identifiziere $\#$ mit $\#^k$. Definiere $\delta'$ wie folgt:
    \begin{alignat*}{2}
        & \delta'(\cdot, q_b \in Q_C \setminus \finset{\#}, & q_c \in Q_C) & := \delta(\#, q_b, q_c) \\
        & \delta'(\cdot, w_b \in Q_C^k, & q_c \in Q_C) & := \delta^{w_bq_c} \\
        & \delta'(\cdot, x_b \in Q', & w_c \in Q^k_C) & := \delta'(\#, x_b, (w_c)_1)
    \end{alignat*}
    
    Da $\#$ ein initialer Zustand in $C$ ist und auch in $\delta'$ durch die Identifikation mit $\#^k$ passiv bleibt, verhält sich $\delta'$ für Zellen mit positiver Position genauso wie $\delta$.
    
    Definiere nun die Hilfsfunktionen $\psi$ und $\phi$:
    \begin{alignat*}{2}
        & \psi(i, j) && := ki+j-k \\
        & \phi(t, i, j) && := t + i - \psi(i, j) \\
        & c^t_i && := \Delta^t_i(c)
    \end{alignat*}

    Für $\phi$ und $\psi$ gelten für $j \in \finset{1, ..., k}$ folgende triviale Fakten, die im Folgenden verwendet werden:
    \begin{alignat*}{2}
        & \psi(i, j) && \leq 0 \\
        & \phi(0, i, j) && \leq -\psi(i, j) \\
        & \psi(1, 1) && = 1 \\
        & \psi(i + 1, 1) && = \psi(i, k) + 1 \\
        & \psi(i, j) && = \psi(i, j - 1) + 1 \\
        & \phi(t, 1, 1) && = t \\
        & \phi(t, i+1, 1) && = \phi(t, i, k) \\
        & \phi(t + 1, i, j) && = \phi(t, i, j-1) \\
    \end{alignat*}
    Da im Beweis nur diese erwähnten Fakten über $\psi$ und $\phi$ verwendet werden, ist es nicht verwunderlich, dass diese Fakten $\psi$ und $\phi$ für $i \leq 0$ und $t \geq 0$ eindeutig bestimmen.

    Für die Wahl $C' := (Q', \delta')$ ist nun zu für alle $i \leq 0$ und $t \in \Nz$ zeigen: \[
        (\Delta^{t}_{C'}(c)_i)_j = c^{\phi(t, i, j)}_{\psi(i, j)}
    \]
    
    Beweis der Aussage über Induktion nach $t \in \Nz$.
    
    Sei $t = 0$. Da $\#$ passiv ist, gilt $c^{t'}_{i'} = \#$ für $0 \leq t' \leq -i'$ und $i \leq 0$.
    Wegen $\phi(0, i, j) \leq -\psi(i, j)$ und $\psi(i, j) \leq 0$ folgt
    $(\Delta^{0}_{C'}(c)_i)_j = \# = c^{\phi(0, i, j)}_{\psi(i, j)}$ für alle $j \in \finset{1, ..., k}$.
    
    Die Aussage gelte nun für ein $t \in \Nz$.
    
    Sei $x_a := \Delta^{t}_{C'}(c)_i$ und $x_b := \Delta^{t}_{C'}(c)_{i+1}$.
    Wegen $i \leq 0$ folgt $x_a \in Q^k_C$.
    
    
    Nach Induktionsvermutung gilt $(x_a)_j = c^{\phi(t, i, j)}_{\psi(i, j)}$.
    
    Wähle $q :=
    \begin{cases}
        x_b & \text{falls } x_b \in Q_C \\
        (x_b)_1 & \text{falls } x_b \in Q^k_C \\
    \end{cases}$.
    
    Für diese Wahl von $q$ gilt nun $q = c^{\phi(t, i + 1, 1)}_{\psi(i + 1, 1)}$:
    
    Angenommen, $x_b \in Q_C$. Nach Definition von $\delta'$ folgt $i = 0$
    und $q = x_b = c^t_1 = c^{\phi(t, 1, 1)}_{\psi(1, 1)} = c^{\phi(t, i + 1, 1)}_{\psi(i + 1, 1)}$.
    
    Andernfalls gilt $x_b \in Q^k_C$ und damit $i < 0$.
    Nach Induktionsvermutung gilt dann
    $q = (x_b)_1 = c^{\phi(t, i + 1, 1)}_{\psi(i + 1, 1)}$.
    
    Zeige $(\Delta^{t+1}_{C'}(c)_i)_j =  c^{\phi(t + 1, i, j)}_{\psi(i, j)}$ durch endliche, absteigende Induktion über $j$.
    
    Für $j = k$ gilt:
    \begin{align*}
        & (\Delta^{t+1}_{C'}(c)_i)_k \\
        & = \delta'(\#, x_a, x_b)_k \\
        & = \delta^{x_a q}_k \\
        & = \delta(\#, (x_a)_k, q) \\
        & = \delta(\#, c^{\phi(t, i, k)}_{\psi(i, k)}, c^{\phi(t, i + 1, 1)}_{\psi(i + 1, 1)}) \\
        & = \delta(\#, c^{\phi(t, i, k)}_{\psi(i, k)}, c^{\phi(t, i, k)}_{\psi(i, k) + 1}) \\
        & = c^{\phi(t + 1, i, k)}_{\psi(i, k)} \\
    \end{align*}
    
    Es gelte die Induktionsvermutung nun für $1 < j \leq k$. Dann gilt:
    \[
    \delta^{x_a q}_j = (\Delta^{t+1}_{C'}(c)_i)_j = c^{\phi(t + 1, i, j)}_{\psi(i, j)}
    \]
    
    Zeige nun die Vermutung für $j-1$:
    \begin{align*}
        & (\Delta^{t+1}_{C'}(c)_i)_{j-1} \\
        & = \delta'(\#, x_a, x_b)_{j-1} \\
        & = \delta^{x_a q}_{j-1} \\
        & = \delta(\#, (x_a)_{j-1}, \delta^{(x_a q)}_j) \\
        & = \delta(\#, c^{\phi(t, i, j-1)}_{\psi(i, j-1)}, c^{\phi(t + 1, i, j)}_{\psi(i, j)}) \\
        & = \delta(\#, c^{\phi(t, i, j-1)}_{\psi(i, j-1)}, c^{\phi(t, i, j-1)}_{\psi(i, j-1)+1}) \\
        & = c^{\phi(t + 1, i, j-1)}_{\psi(i, j - 1)} \\
    \end{align*}
    
    Damit ist auch der Induktionsschritt gezeigt und der Beweis vollendet.
\end{proof}

Damit lässt sich dann der folgende Satz zeigen.

\begin{satz}
    \label{CAgfSpeedup}
    Sei $C \in \CA$ ein Zellularautomat.
    Dann gibt es einen Zellularautomaten $C''' \in \CA$ und Funktionen $g_1$ und $g_2$, sodass für alle $p \in \N$ gilt:
    \begin{align*}
        g_1(\Delta_{C'''}^{2p-1}(c)_p) & = \Delta_C^{3p-2}(c)_1 \\
        g_2(\Delta_{C'''}^{2p}(c)_{p+1}) & = (\Delta_C^{3p-1}(c)_1, \Delta_C^{3p}(c)_1)
    \end{align*}
    
    Ferner existiert dann ein Automat $C_1$ und eine Funktion $f$, sodass für $i \geq 1$ gilt:
    \[
        f(\Delta_{C_1}^{2i+1}(c)_i) = (\Delta_C^{3i-3}(c)_1, \Delta_C^{3i-2}(c)_1, \Delta_C^{3i-1}(c)_1)
    \]
    
    \cref{fig:EchtzeitSpeedup} zeigt die Zustands-Abhängigkeiten der Automaten zueinander.
    Die Kreise stellen die Zustände der Ausführung des Automaten $C$ dar, die Sechsecke zeigen das Bild von $g_1$ bzw. $g_2$
    der Zustände von $C'''$ und die Fünfecke das Bild von $f$ der Zustände von $C_1$. Die eingezeichneten Pfeile
    zeigen, wodurch sich die Zustände ergeben.
    
    \begin{figure}[H]
        \centering
        \includesvg[width=175pt]{images/EchtzeitSpeedup}
        \caption{Visualisierung der Ausführung der Automaten $C$ (Zustände rund), $C'''$ (Zustände sechseckig) und $C_1$ (Zustände fünfeckig)}
        \label{fig:EchtzeitSpeedup}
    \end{figure}
\end{satz}
\begin{proof}
    Wähle $C'$ wie in \cref{zellautoZuLinksunabhaengig}, sodass
    für $\forall c \in Q^{\Z}, \forall t \in \Nz, \forall i \in \Z$ gilt:
    \[
        \Delta_{C'}^{t}(c)_i =
        \begin{cases}
            \Delta_C^{\frac{t}{2}}(c)_{i+\frac{t}{2}} & \text{ falls } $t$ \text{ gerade, } \\
            (\Delta_C^{\frac{t - 1}{2}}(c)_{i+\frac{t - 1}{2}}, \Delta_C^{\frac{t-1}{2}}(c)_{i+\frac{t+1}{2}})  & \text{ sonst.}
        \end{cases}
    \]
    
    Wähle $C''$ wie in \cref{linksunabhaengigSpeedup}, sodass für $i \in \Z$ mit $i \leq 0$ und $t \in \Nz$
    und alle Konfigurationen $c$ mit $c_p = \#$ für $p \leq 0$ oder $p \geq p_0$ gilt:
    \[
        \Delta^{t}_{C''}(c)_i = ( \Delta^{t-2i+2}_{C'}(c)_{3i-2},
        \Delta^{t-2i+1}_{C'}(c)_{3i-1},
        \Delta^{t-2i}_{C'}(c)_{3i}
        )
    \]
    
    Wähle $C'''$ wie in \cref{linksunabhaengigZuZellauto}, sodass für $\forall c \in Q^{\Z}$,
    $\forall t \in \Nz$, $\forall i \in \Z$ gilt:
    \[
        \Delta^t_{C'''}(c)_{i} = \Delta^{2t}_{C''}(c)_{i-t}
    \]
    
    Wähle $g_1(q \in Q''') := q_3$ und $g_2(q \in Q''') := ((q_2)_1, q_1)$.
    
    Es gilt dann:
    \begin{align*}
          & g_1(\Delta_{C'''}^{2p-1}(c)_p) \\
        = & (\Delta_{C''}^{4p-2}(c)_{1-p})_3 \\
        = & \Delta_{C'}^{6p-4}(c)_{3-3p} \\
        = & \Delta_{C}^{3p-2}(c)_1
    \end{align*}
    
    Außerdem gilt:
    \begin{align*}
          & g_2(\Delta_{C'''}^{2p}(c)_{p+1}) \\
        = & g_2(\Delta_{C''}^{4p}(c)_{1-p}) \\
        = & ((\Delta_{C'}^{6p-1}(c)_{2-3p})_1, \Delta_{C'}^{6p}(c)_{1-3p}) \\
        = & (\Delta_C^{3p-1}(c)_1, \Delta_C^{3p}(c)_1)
    \end{align*}

    Anhand von \cref{fig:EchtzeitSpeedup} ist nun ersichtlich, wie $C_1$ und $f$ konstruiert werden müssen.
\end{proof}

\section{Speedup um einen Faktor durch Komprimierung}

Wenn das Eingabewort komprimiert werden darf, lässt sich ein noch viel stärkerer Speedup-Satz zeigen.

\begin{definition}[Speedup-Konstruktion $S_k(C)$, $S_k(c)$, $\gamma_q$]
    \label{factorSpeedupConstruction}
    Es sei ein Zellularautomat $C := (Q, \delta) \in \CA$ und ein $k \in \N$ gegeben.
    Definiere $S_k(C) := (Q', \delta')$ mit $Q' := Q^k$.
    
    Definiere für $q \in Q$ und $l, r \in Q^*$ die Konfiguration $c(q, l, r) \in Q^{\Z}$:
    \[
        c(q, l, r)_i := \begin{cases}
            (lr)_{i + |l|} & \text{ falls } 1-|l| \leq i \leq |r| \\
            q & \text{ sonst}
        \end{cases}
    \]
    
    Wähle $q \in Q$ beliebig.
    Definiere $\delta' : {Q'}^3 \to Q'$ wie folgt:
    \[
        \delta'(a, b, c) := \Delta_C^k(c(q, a, bc))[1..k]
    \]
    
    Definiere die Konfigurationskomprimierung $S_k(c) \in Q'^\Z$ für eine Konfiguration $c \in Q^\Z$:
    \[
        S_k(c)_i := c_{k(i - 1) + 1}c_{k(i - 1) + 2}...c_{k(i - 1) + k} \in Q^k
    \]
    
    Definiere außerdem noch $\gamma_q : Q' \to Q$ mit
    $\gamma_q(a) := \Delta^{k-1}_C(c(q, \epsilon, a))_1$.
\end{definition}

\begin{satz}[Korrektheit der Speedup-Konstruktion]
    \label{factorSpeedupConstructionCorrectness}
    Es sei ein Zellularautomat $C := (Q, \delta) \in \CA$, ein $k \in \N$ und eine Konfiguration $c: \Z \to Q$ gegeben.
    Für beliebige $t \in \Nz, i \in \Z$ und $j \in \finset{1, ..., k}$ gilt dann:
    \[
        (\Delta^t_{S_k(C)}(S_k(c))_i)_j = \Delta^{kt}_C(c)_{k(i-1)+j}
    \]
    Ferner gilt für $q := (\Delta^t_{S_k(C)}(S_k(c))_{i-1})_k$:
    \[
        \gamma_{q}( \Delta^t_{S_k(C)}(S_k(c))_{i} )
        = \Delta^{kt+k-1}_C(c)_{k(i-1)+1}
    \]
\end{satz}
\begin{proof}
    Die Aussage kann durch einfaches Nachrechnen gezeigt werden, auf das an dieser Stelle verzichtet wird.
\end{proof}

Die Konstruktion aus \cref{factorSpeedupConstruction} kann beispielsweise verwendet werden, um folgenden Satz zu beweisen.

\begin{theorem}
    \label{linSpeedup}
    $\CAL{\CALT, \CALeft} = \CAL{\CAT, \CALeft, \mathrm{time}(\finset{ n \mapsto 2n }) }$
\end{theorem}
\begin{proof}
    Mithilfe der in \cref{chap:ErweiterteNakamuraKonstr} und \cref{factorSpeedupConstruction} vorgestellten Konstruktionen kann das Theorem leicht bewiesen werden.
    Weitere Beweise finden sich in \cite{MAZOYER199259} und \cite{IBARRA1988225}
\end{proof}


\chapter{Erweiterte Nakamura-Konstruktion zur asynchronen Synchronisation}
\label{chap:ErweiterteNakamuraKonstr}

\section{Definition}

\begin{definition}[Erweiterte Nakamura-Konstruktion $A(M, f, C, P)$]
    \label{erweiterteNakamuraKonstruktion}
    Es seien zwei Zellularautomaten $M, C \in \CA$, eine Funktion $f: Q_M \to \mathrm{Op}$ mit $\mathrm{Op} := \set{
        \mathrm{reset}, \mathrm{set}(q), \mathrm{step} }{ q \in Q_C }$ und eine Menge $P \subseteq Q_C$ von passiven, initialen Zuständen gegeben.
    Der Automat $M$ gibt Steuerbefehle über die Funktion $f$ und der Automat $C$ ist der asynchron simulierte Automat.
    Mithilfe der Menge $P$ lässt sich in bestimmten Fällen
    die benötigte Zeit zum Simulieren verbessern.
    
    Setze $N := \finset{ 0, ..., 5 }$, $Q' := (Q_C^3 \times N \times \Nz) \cup \finset{ \perp }$ und $Q := Q' \times Q_M$.
    Die Komponenten eines Elements $q \in Q_C^3$ seien symbolisch mit $q_a$, $q_{a-1}$ und $q_0$ bezeichnet: $q = (q_a, q_{a-1}, q_0)$.
    $q_a$ meint dabei den aktuellen Zustand des simulierten Automaten, $q_{a-1}$ den Zustand des vorherigen Schrittes (sofern dieser existiert) und $q_0$ den Zustand der simulierten Zelle in der Anfangskonfiguration.
    Diese Notation lässt sich auf die erste Komponente von $Q'$ und dann $Q$ übertragen. Die zweite Komponente eines Elements $q \in Q' \setminus \finset{ \perp }$ wird mit $q_t$ und die dritte mit $q_{t'}$ bezeichnet. $q_t$ speichert \enquote{grob}, wieviele Schritte im simulierten Automaten für diese Zelle schon berechnet wurden. $q_{t'}$ speichert diese Zahl exakt und dient nur zur Beweisführung. Da die Komponente $q_{t'}$
    nach Konstruktion nicht relevant für das sichtbare Verhalten des Automaten sein wird, existiert ein äquivalenter Automat, der ohne diese Komponente auskommt. Dessen Zustandsmenge ist dann endlich.
    Für $q \in Q$ bezeichne mit $q_q$ die erste Komponente und mit $q_m$ die zweite.
    
    Definiere zunächst $\delta': {Q'}^3 \to Q'$ für $a, b, c \in Q'$ für den Fall $\perp \in \finset{ a, b, c }$:
    \[
        \delta'(a, b, c) := 
        \begin{cases}
            \perp & \text{falls } b = \perp \\
            ((b_0, b_0, b_0), 0, 0) & \text{sonst}
        \end{cases}
    \]
    Dem Zustand $\perp \in Q'$ kommt damit die Bedeutung zu, nichts über den Zustand der Zelle im simulierten Automaten zu wissen.
    Die Nachbarn einer Zelle in diesem Zustand können dann offensichtlich keinen Schritt berechnen.
    
    Definiere für $t \in N$ die Funktion $s(t)$ zur Bestimmung eines Nachfolgers in $N$ (siehe \cref{wirkungVonS}):
    \[
        s(t) :=
        \begin{cases}
            t + 1 & \text{falls } t < 5 \\
            3 & \text{falls } t = 5
        \end{cases}
    \]
    
    \begin{figure}[h]
        \centering
        \includesvg{Nakamura_s.svg}
        \caption{Wirkung von s}
        \label{wirkungVonS}
    \end{figure}
    
    
    Definiere $q^t$ zum Zugriff auf den möglicherweise in $q$ gespeicherten Zustand der Zelle zum Zeitpunkt $t'$ mit $s(t') = t$
    für $q \in Q'$ und $t \in N$:
    \[
        q^{t} :=
        \begin{cases}
            q_{0} & \text{falls } q_0 \in P \text{ und } q_t = 0 \\
            q_{a-1} & \text{falls } q_0 \not\in P \text{ und } q_t = t \\
            q_a & \text{falls } q_0 \not\in P \text{ und } s(q_t) = t \\
            \perp & \text{sonst}
        \end{cases}
    \]
    
    Falls $q_0$ von Anfang an tot ist,
    beinhaltet $q$ Zustandsinformationen
    der simulierten Zelle für alle Zeitpunkte,
    da tote Zustände in jedem Schritt gleich bleiben.
    
    Definiere nun $\delta'$ für $a, b, c \in Q'$ falls $\perp \not\in \finset{ a, b, c}$:
    \[
        \delta'(a, b, c) := 
        \begin{cases}
            \Spvek{(b_0, b_0, b_0), 0, 0}
            & \text{ falls } b_0 \in P \text{ und } b_t = 0 \\
            \Spvek{(\delta(a^t, b^t, c^t), b^t, b_0), t, b_{t'} + 1}
            & \text{ sonst, falls } \perp \not\in \finset{ a^t, c^t } \text{ für } t := s(b_t) \\
            \Spvek{(\delta(a^t, b^t, c^t), b^t, b_0), t, b_{t'}}
            & \text{ sonst, falls } \perp \not\in \finset{a^t, c^t} \text{ für } t := b_t \\
            \Spvek{(\delta(a_0, b_0, c_0), b_0, b_0), 1, 1}
            & \text{ sonst}
        \end{cases}
    \]
    
    Der erste Fall hält tote Zustände in der Simulation fest, da sie nach Definition konstant sind. Dies ermöglicht nach Definition von $q^t$ eine schnellere Simulation, da nicht auf Zellen in solch einem Zustand gewartet werden muss. Aus Komplexitätsgründen werden nur solche toten Zustände berücksichtigt, die von $M$ mit $\mathrm{set(q)}$ schon direkt gesetzt wurden, was die Bedingung $b_t = 0$ erklärt.
    
    Im zweiten Fall wird der nächste Schritt des simulierten Automaten für die zu aktualisierende Zelle berechnet.
    Dies ist offensichtlich nur möglich, wenn
    die Nachbarzellen Informationen zum aktuellen Schritt gespeichert haben.
    
    Andernfalls, wenn die Nachbarzellen nur Informationen zum letzten Schritt gespeichert haben, greift Fall drei und zu aktualisierende Zelle berechnet den aktuellen Schritt neu.
    Dies ist wichtig, falls sich Zellen zurücksetzen.
    
    Im vierten Fall setzt die Zelle die Simulation
    lokal zurück und berechnet wieder den ersten Schritt der Simulation.
    
    
    Abschließend ist $\delta$ für $a, b, c \in Q$ nun wie folgt definiert:
    \[
        \delta(\Spvek{a_q, a_m}, \Spvek{b_q, b_m}, \Spvek{c_q, c_m}) := 
        \begin{cases}
            \text{Setze } c := \delta_M(a_m, b_m, c_m) \\
            \Spvek{((q, q, q), 0, 0), c} & \text{falls } f(c) = \mathrm{set}(q) \\
            \Spvek{\perp, c} & \text{falls } f(c) = \mathrm{reset} \\
            \Spvek{\delta'(a_q, b_q, c_q), c} & \text{falls } f(c) = \mathrm{step}
        \end{cases}
    \]
    
    Damit kann der Automat $M$ die Simulation kontrollieren
    und diese für einzelne Zellen auf einen neuen Anfangszustand zurücksetzen. Die Konstruktion stellt sicher,
    dass sich benachbarte Zellen einer solchen zurückgesetzten Zelle
    korrekt verhalten.
    
    Definiere nun $A(M, f, C, P) := (Q, \delta)$.
    
    Wenn nun eine Konfiguration $c_M : \Z \to Q_M$ von $M$
    gegeben ist, kann sie als Konfiguration $c: \Z \to Q$ von $A(M, f, C, P)$
    aufgefasst werden:
    \[
        c_i :=
        \begin{cases}
            (((q, q, q), 0, 0), (c_M)_i) & \text{falls } f((c_M)_i) = \mathrm{set}(q) \\
            (\perp, (c_M)_i) & \text{ sonst}
        \end{cases}
    \]
\end{definition}

\begin{definition}[Sichtbare Startzustandszeit $st_o(t, i)$ und Startkonfiguration $c(t, i)$, sichtbare aktiven Zellen $A(t, i)$]
    Es seien $M, C \in \CA$, $f$ und $P$ wie in \cref{erweiterteNakamuraKonstruktion} und eine Konfiguration $c_M: \Z \to Q_M$ gegeben.
    Seien $t \in \Nz$, $i \in \Z$ und $o \in \mathrm{Op}$.
    Definiere die sichtbare Startzustandszeit $\mathrm{st}_{M, C, f, P, o}(t, i) := \mathrm{st}_o(t, i)$:
    \[
        \mathrm{st}_o(t, i) :=
        \begin{cases}
            t' &
                \text{falls } f(\Delta_M^{t'}(c_M)_i) = o
                \text{ für } t' := \max \set{ k \in \finset{0, ..., t} }{ f(\Delta_M^{k}(c_M)_i) \neq \mathrm{step} }
            \\
            \infty & \text{sonst}
        \end{cases}
    \]
    
    Definiere die von $(t, i)$ aus sichtbare Startkonfiguration $c_{M, f, \#}(t, i)_j := c(t, i)$
    für $\# \in Q_C$, $t \in \Nz$ und $p \in \Z$ wie folgt:
    \[
        c(t, i)_j :=
        \begin{cases}
            q & 
                \text{falls } \mathrm{st}_{\mathrm{set}(q)}(t - |i-j|, j) \neq \infty
             \\
            \# & \text{sonst}
        \end{cases}
    \]
    
    Definiere die von $(t, i)$ aus sichtbare tote rechte Zelle $\mathrm{dR}(t, i)$ und linke Zelle $\mathrm{dL}(t, i)$:
    \[
        \mathrm{dR}(t, i) := \min (\set{ j \in \Z}{j \geq i \land c(t, i)_j \in P} \cup \finset{ \infty })
    \]
    und
    \[
        \mathrm{dL}(t, i) := \max (\set{ j \in \Z}{j \leq i \land c(t, i)_j \in P} \cup \finset{ -\infty })
    \]
    
    Definiere damit die von $(t, i)$ aus sichtbaren aktiven Zellen $A(t, i)$:
    \[
        A(t, i) := \finset{ \mathrm{dL}(t, i), ..., \mathrm{dR}(t, i)}
    \]
\end{definition}

\section{Korrektheit und Eigenschaften}

\begin{satz}[Korrektheit der erweiterten Nakamura-Konstruktion]
    Es seien $M, C \in \CA$, $f$ und $P$ wie in \cref{erweiterteNakamuraKonstruktion} und eine Konfiguration $c_M: \Z \to Q_M$ gegeben.
    Sei $c$ die Auffassung von $c_M$ als Konfiguration von $A := A(M, f, C, P)$.
    
    Es gelten folgende Aussagen für $t \in \Nz$, $i \in \Z$:
    
    \begin{enumerate}
        \item
            $A$ führt $M$ korrekt aus:
            \[
                (\Delta_A^t(c)_i)_m = \Delta_M^t(c)_i
            \]
        \item
            Sei $c \in \finset{-1, 1}$ mit $\perp \not \in \finset{a := (\Delta_A^t(c)_i)_q, b := (\Delta_A^t(c)_{i+c})_q}$.
            Benachbarte Zellen mit ähnlicher \enquote{groben} Zeit haben ähnliche \enquote{exakte} Zeit:
            \[
                s(a_{t}) = s(b_{t}) \Rightarrow a_{t'} = b_{t'} \text{ und }  a_{t} = s(b_{t}) \Rightarrow a_{t'} = b_{t'} + 1
            \]
            Für $t \in \finset{0, 1, 2}$ gilt offensichtlich:
            \[
                a_{t} = t \Rightarrow a_{t'} = t
            \]
        \item
            Sei $q := (\Delta_A^t(c)_i)_q$ mit $q \neq \perp$.
            Dann reflektiert $q_a$ aus Sicht von $(t, i)$ den korrekt simulierten Zustand:
            \[
                q_a = \Delta_C^{q_{t'}}(c_{M, f, \#}(t, i))_i
            \]
            Analog reflektiert $q_{a-1}$ den vorherigen Zustand, falls $q_t > 0$:
            \[
                q_{a-1} = \Delta_C^{q_{t' - 1}}(c_{M, f, \#}(t, i))_i
            \]
            Für $q_t = 0$ gilt $q_{a-1} = q_a$.
    \end{enumerate}
    
\end{satz}

\begin{satz}[Zeitbedarf der erweiterten Nakamura-Konstruktion in einfachen Fällen]
    \label{timeNakamuraConstruction}
    Es seien $M, C \in \CA$, $f$ und $P$ wie in \cref{erweiterteNakamuraKonstruktion} und eine Konfiguration $c_M: \Z \to Q_M$ gegeben.
    Sei außerdem $c$ die Auffassung von $c_M$ als Konfiguration von $A := A(M, f, C, P)$.
    
    Für $t \in \Nz$, $i \in \Z$ und $q := (\Delta_{A}^t(c)_i)_q$ setze:
    \[
        \mathrm{tc}(t, i) := \max \set{ \hat{t} \in \Nz}
            {
                \forall k \in \finset{-\hat{t}, ..., \hat{t} } \cap A(t, i):
                \mathrm{st}_{\mathrm{set}}(t-|k|, i+k) \leq t - \hat{t}
            }
    \]
    \begin{enumerate}
        \item
            Für $q \neq \perp$ gilt $q_{t'} \leq \mathrm{tc}(t, i)$.
        \item
            Für $q \neq \perp$ gilt sogar  $q_{t'} = \mathrm{tc}(t, i)$, falls für alle $k \in \finset{-\hat{t}, ..., \hat{t} } \cap A(t, i)$ gilt:
            \[
                t_1(t, k) > t_2(t, k)
                \; \land \;
                st_{\mathrm{reset}}(t_1(t, k) - 1, i + k) \leq t_2(t, k)
            \]
            Wobei
            \[
                t_1(t, k) := \min \finset{ t_2(t, k - 1), t_2(t, k + 1) }
            \]
            und
            \[
                t_2(t, k) := st_{\mathrm{set}}(t - |k|, i + k)
            \]
        \item
            Sei $c': \Z \to Q_C$ eine Konfiguration von $C$ und $n$ so, dass $c'_i \neq \# \Leftrightarrow i \in \finset{ 1, ..., n }$.
            Angenommen, $P \subseteq \finset{ \# } $ und für $i \in \Z$ und $t \in \Nz$ gelte:
            \[
                f(\Delta_M^t(c_M)_i) =
                \begin{cases}
                    \mathrm{reset} & \text{falls } t = 0 \text{ und } i \in \finset{1, .., n} \\
                    \mathrm{set}(c'_i) & \text{falls } t = 0 \text{ und } i \not\in \finset{1, .., n} \\
                    \mathrm{set}(c'_i) & \text{falls } i \in \finset{1, ..., n} \text{ und } t = 2 * i + 1 \\
                    \mathrm{step} & \text{sonst}
                \end{cases}
            \]
            
            Dann gilt für $3 \leq t \leq 3n$:
            \[
                (\Delta^{t}_A(c)_1)_a = \Delta^{\lfloor \frac{t}{3} \rfloor - 1}_C(c')
            \]
     \end{enumerate}
\end{satz}
\chapter{Advice-Zellularautomaten}
\label{chap:AdvAuto}

Dieses Kapitel beschäftigt sich damit, inwiefern die Mächtigkeit bestimmter Klassen von Zellularautomaten verändert wird,
wenn der Automat zusätzlich zum Eingabewort ein vom Eingabewort abhängiges Unterstützungswort (im Folgenden Advice genannt) erhält.
Insbesondere wird gezeigt, dass bestimmte solcher Klassen unter Zuhilfenahme ausgewählter Advices abgeschlossen sind.
Solche Advices heißen dann verträglich bezüglich dieser Klasse.
Diese Abschlusseigenschaft ist nützlich, um die Konstruktion einiger Realzeit-Zellularautomaten zu vereinfachen.

\section{Definition}

\begin{definition}
    Ein $\Sigma$-$\Gamma$-Advice (Hinweis) $\mathcal{A}$ ist eine Abbildung
        $\mathcal{A}: \Sigma^* \to \Pot(\Gamma^*)$ mit $\forall w \in \Sigma^*: \mathcal{A}(w) \subseteq \Sigma^{|w|}$.
    Wenn $\forall w \in \Sigma^*: |\mathcal{A}(w)| = 1$, kann $\mathcal{A}$ auch
    als Abbildung $\mathcal{A} : \Sigma^* \to \Gamma^*$ aufgefasst werden.
\end{definition}

\begin{definition}
    Definiere injektive Funktion $\comp: (\Sigma \times \Gamma)^* \to (\Sigma^* \times \Gamma^*)$
    um ein Wort über zwei Alphabete in zwei Wörter über je ein Alphabet zu zerlegen:
    \[
        \forall v \in \Sigma^*, w \in \Gamma^{|v|}: \comp(\Spvek{v_1; w_1}...\Spvek{v_{|v|}; w_{|v|}}) := \Spvek{v; w}
    \]
    
    Definiere $\comb: \comp((\Sigma \times \Gamma)^*) \to (\Sigma \times \Gamma)^*$,
    um zwei Wörter über je ein Alphabet zu einem Wort über zwei Alphabete zusammenzufügen:
    \[
        \comb(w) := \comp^{-1}(w)
    \]
    
\end{definition}

\begin{exmp}
    \[
        \comp(\Spvek{\chr{1}; \chr{a}}\Spvek{\chr{2}; \chr{b}}\Spvek{\chr{3}; \chr{c}}) = \Spvek{\chr{123}; \chr{abc}}
    \]
    
    \[
        \comb(\Spvek{\chr{123}; \chr{abc}}) = \Spvek{\chr{1}; \chr{a}}\Spvek{\chr{2}; \chr{b}}\Spvek{\chr{3}; \chr{c}}
    \]        
\end{exmp}

\begin{definition}
    Es sei ein Advice $\mathcal{A}$ und ein Zellularautomat $C \in \ECA$ gegeben. Mit Unterstützung des Advices $\mathcal{A}$ kann $C$ nun folgende Sprache erkennen:
    \[
         L_{\mathcal{A}, C} := \set{ w \in \Sigma^*}{\exists v \in \mathcal{A}(w): \comb(\Spvek{w; v}) \in L(C)}
    \]
    
    Für $M(\mathcal{A}) := \set{ \comb(\Spvek{ w; v}) }
        { w \in \Sigma^* \land v \in \mathcal{A}(w) }$
    kann $L_{\mathcal{A}, C}$ auch wie folgt aufgefasst werden:
    \[
        L_{\mathcal{A}, C} = \comp(
            L(C) \cap M(\mathcal{A})
        )_1
    \]
    
    Folgender $\ECA$-Funktor stellt einer Klasse von Zellularautomaten ein gegebenes Advice zur Verfügung:
    \[
        \orakel(\mathcal{A})(M \subseteq \ECA) := \set{ (Q_C, \delta_C, \Sigma, \#_C, L_{\mathcal{A}, C}, (S_C, C))} { C \in M, \; \Sigma_C = \Sigma \times \Gamma }
    \]
\end{definition}

\section{Ergebnisse}

\begin{satz}
    Seien $\mathcal{A}_1$ und $\mathcal{A}_2$ zwei Advices.
    Definiere $(\mathcal{A}_1 \cup \mathcal{A}_2)(w) := \mathcal{A}_1(w) \cup \mathcal{A}_2(w)$ und 
    $(\mathcal{A}_1 \cap \mathcal{A}_2)(w) := \mathcal{A}_1(w) \cap \mathcal{A}_2(w)$.
    
    Sind $\mathcal{A}_1$ und $\mathcal{A}_2$ $\CAM{\CART, \CALeft}$-verträglich, so sind
    auch $\mathcal{A}_1 \cup \mathcal{A}_2$ und $\mathcal{A}_1 \cap \mathcal{A}_2$
    $\CAM{\CART, \CALeft}$-verträglich.
\end{satz}
\begin{proof}
    Es werden entsprechend zwei Ausführungen simuliert - je eine auf einem 
    
\end{proof}

\begin{satz}
    Sei $(\Sigma \cup \finset{\Box}) ^2 \subseteq \Gamma, \Box \not\in \Sigma$.
    Definiere den \enquote{Komprimierungs}-Advice $\mathcal{A}_K$:
    \[
        \mathcal{A}_K(w) := 
                  \Spvek{w_1; w_2} \Spvek{w_3; w_4} ... \Spvek{w_{|w-1|}; w_{|w|}}
                        \; \Spvek{\Box; \Box}^\frac{|w|}{2}
    \]
    Falls $|w|$ ungerade, setze $w := w\Box$.
    Unter dem Advice $\mathcal{A}_K$ fällt Realzeit mit Linearzeit zusammen:
    \[
        \CAL{\CART, \CALeft, \orakel({\mathcal{A}})} = \CAL{\CALT, \CALeft}
    \]
\end{satz}
\begin{proof}
    Die Inklusion \enquote{$\subseteq$} ist klar.
    Sei $L \in \CAL{\CALT, \CALeft}$.
    Nach \cref{linSpeedup} existiert ein Zellularautomat $C$, der $L$ in $2n$ Schritten erkennt.
    Der Automat $C' := S_2(C)$ aus \cref{factorSpeedupConstruction}
    erkennt $L$ dann in $n$ Schritten auf der komprimierten Konfiguration $S_2([w])$.
    Da die komprimierte Konfiguration durch das Advice bereitgestellt wird, kann $L$ mithilfe des Advices in Echtzeit erkannt werden.
\end{proof}


\begin{comment}
    \begin{satz}
        Sei $(\Sigma \cup \finset{\Box}) ^2 \subseteq \Gamma, \Box \not\in \Sigma$.
        
        \begin{enumerate}
            \item  $\mathcal{A}_1(w) := \Gamma^{|w|}$
        \end{enumerate}
        
        Es gilt jeweils:
        \[
            \CAL{\CART, \orakel({\mathcal{A}_i})} = \CAL{\CALT, \orakel({\mathcal{A}_i})}
        \]
        
    \end{satz}

    \begin{definition}
        Seien $\mathcal{A}_1$ und $\mathcal{A}_2$ zwei Orakel, $M \subseteq \ECA$ (falls nicht angegeben, $M := \ECA^\CART$).
        \[
            \mathcal{A}_1 \leq_M \mathcal{A}_2 \; :\Leftrightarrow \; \mathcal{L}(M^{\orakel(\mathcal{A}_1)}) \subseteq \mathcal{L}(M^{\orakel(\mathcal{A}_2)})
        \]        
    \end{definition}
    
    \begin{lemma}
        Es gibt kein maximales Orakel.
    \end{lemma}
    
\end{comment}

\begin{satz}
    Sei $\mathcal{A}$ ein Advice mit $\CAL{\CART, \CALeft, \orakel({\mathcal{A}})} \neq \CAL{\CALT, \CALeft, \orakel({\mathcal{A}})}$.
    Dann $\CAL{\CART, \CALeft} \neq \CAL{\CALT, \CALeft}$.
\end{satz}
\begin{proof}
    Angenommen $\CAL{\CART, \CALeft, \orakel({\mathcal{A}})} \neq \CAL{\CALT, \CALeft, \orakel({\mathcal{A}})}$.
    
    Wegen $\CAL{\CART, \CALeft, \orakel({\mathcal{A}})} \subseteq \CAL{\CALT, \CALeft, \orakel({\mathcal{A}})}$
    gibt es laut Annahme ein $L \in \CAL{\CALT, \CALeft, \orakel({\mathcal{A}})} \setminus \CAL{\CART, \CALeft, \orakel({\mathcal{A}})}$.
    Dann gibt es einen Automaten $C \in \CAM{\CALT, \CALeft}$ mit
    
    \[
        L = L_{\mathcal{A}, C} = \set{ w \in \Sigma^*}{\exists v \in \mathcal{A}(w): \comb(\Spvek{w; v}) \in L(C)}
    \]
    
    Angenommen, es gäbe einen Automaten $C' \in \CAM{\CART, \CALeft}$ mit $L_{C'} = L_C$.
    Dann $L_{\mathcal{A}, C} = \set{ w \in \Sigma^*}{\exists v \in \mathcal{A}(w): \comb(\Spvek{w; v}) \in L(C')} = L_{\mathcal{A}, C'}$ und damit
    $L \in \CAL{\CART, \CALeft, \orakel({\mathcal{A}})}$.
    Widerspruch zur Annahme! Also $L(C) \in \CAL{\CALT, \CALeft} \setminus \CAL{\CART, \CALeft}$.
\end{proof}


\section{\texorpdfstring{$\CAM{\CART, \CALeft}$}{CA\^RT-L}-verträgliche Advices}

\begin{definition}[$\mathcal{A}$ ist $M$-verträglich]
    Sei $M \subseteq \ECA$ eine Menge von Zellularautomaten und $\mathcal{A}$ ein Advice.
    $\mathcal{A}$ heißt $M$-verträglich, falls gilt:
    \[
        \mathcal{L}(M) = \mathcal{L}(M^\mathcal{A})
    \]
\end{definition}

\begin{satz}
    \label{lemmaIgnoriereAdvice}
    Sei ein Advice $\mathcal{A}$ und eine Menge 
    $F \subseteq {\Nz}^{\Nz}$ von Funktionen gegeben.
    Dann gilt: $\mathcal{L}(\CAM{ \mathrm{time}(F), \CALeft }) \subseteq  \mathcal{L}(\CAM{ \mathrm{time}(F), \CALeft, \orakel(\mathcal{A})})$.
\end{satz}
\begin{proof}
    Der Advice kann ignoriert werden, indem $\Sigma \times \Gamma$ auf $\Sigma$ reduziert wird.
\end{proof}







\begin{definition}
    Seien $k, n \in \Nz$ zwei Zahlen. Die Funktion $\mathrm{ones}$ gibt ein Wort der Länge $n$ zurück, sodass die $k$ ersten Zeichen Einsen sind.
    \[
        \mathrm{ones}(k, n) := (\chr{1}^k \chr{0}^n)[1..n]
    \]
\end{definition}

\begin{exmp}
    \[
        \mathrm{ones}(2, 5) = \chr{11000}
    \]
    \[
        \mathrm{ones}(6, 5) = \chr{11111}
    \]        
\end{exmp}

\begin{definition}
    Sei $\Gamma$ ein endliches Alphabet.
    Ein $F$-haltender Zellularautomat $C \in \ECA^\CAF$ berechnet eine Funktion $f: \Sigma^* \to \Gamma^*$, wenn
    $\Gamma \subseteq Q_C$, $\#_C \not\in \Gamma$ und 
    \[
        \forall w \in \Sigma^*: \Delta^{\ctime_C(w)}_{C}([w]) = [f(w)]
    \]
    gilt.
\end{definition}

\begin{satz}
    \label{lemmaEinfachesOrakel}
    Wenn $\mathcal{A}: \Sigma^* \to \Gamma^*$ eine Funktion ist mit $\forall w \in \Sigma^*: |f(w)| = |w|$,
    die von einem $F$-haltenden Zellularautomaten $B$ mit
    $\ctime_C(\cdot) = k$ und $k \in \Nz$ berechnet wird,
    dann ist $\mathcal{A}$ $\CAM{\CART, \CALeft}$-verträglich.
\end{satz}
\begin{proof}
    Die Inklusion \enquote{$\subseteq$} der $\CAM{\CART, \CALeft}$-Verträglichkeit folgt durch ~\cref{lemmaIgnoriereAdvice}.
    
    Sei nun $L \in \CAL{\CART, \CALeft, \orakel({\mathcal{A}})}$, $\bar{C} \in \CAM{\CART, \CALeft, \orakel({\mathcal{A}})}$ sodass $L_{\bar{C}} = L$.
    Sei $C := C_{\bar{C}}$ der zugrundeliegende $\Sigma \times \Gamma$ erkennende Zellularautomat,
    der mithilfe des Advices $\mathcal{A}$ die Sprache $L$ in Realzeit erkennt.
    
    Es gibt einen Zellularautomaten $I$, der in $k$ Schritten die Identität berechnet. Folglich gibt es einen Zellularautomaten $A$, der die Funktion $f: \Sigma^* \to (\Sigma \times \Gamma)^*$ mit
    $f(w) = \comb(\Spvek{w; f(w)})$ in $k$ Schritten berechnet, in dem $I$ und $B$ für $k$ Schritte parallel ausgeführt werden.
    
    Konstruiere den Automaten $C'$ durch Hintereinanderausführung der Automaten $A$ und $C$.
    Für ein Wort $w \in \Sigma^*$ braucht der Automat $C'$ $k + |w|$ viele Schritte und es gilt $L_{C'} = L$.
    
    Nach \cref{satzRealzeitSpeedup} gibt es nun einen Automaten $C'' \in \CAM{\CART, \CALeft}$
    mit $L_{C''} = L_{C'}$.
    Also $L \in \CAL{\CART, \CALeft}$.
\end{proof}

\begin{corollary}
    Sei $\Gamma = \B$, $k \in \Nz$.
    Folgende Advices sind $\CAM{\CART, \CALeft}$-verträglich:
    \begin{enumerate}
        \item $\mathcal{A}_1(w) := \mathrm{ones}(k, |w|)$ 
        \item $\mathcal{A}_2(w) := \mathrm{ones}(k, |w|)^{\mathrm{Rev}}$
    \end{enumerate}
\end{corollary}
\begin{proof}
    Die Funktionen $\mathcal{A}_i$ können offensichtlich
    von einem $F$-haltenden Zellularautomat in $k$ Schritten berechnet werden.
    Mit ~\cref{lemmaEinfachesOrakel} folgt die Behauptung.
\end{proof}

\begin{comment}
        $Q' := (Q_C \times \finset{ 0, \dots, c })$. Identifiziere $\Sigma' := \Sigma_C$ mit $(\finset{\chr{0}} \times \Sigma_C) \times \finset{0} $. Setze $\#' := (\#_C, 0)$.
        
        $\delta'(\Spvek{\#_C; k_a}, \Spvek{\Spvek{ \gamma_b; c_b}; k_b}, c) := \Spvek{ \Spvek{\chr{1}; c_b}; k_b + 1}$
        
        $\delta'(\Spvek{\Spvek{\gamma_a; c_a}; k_a}, \Spvek{\Spvek{\gamma_b; c_b}; k_b}, c)
            := \Spvek{ \Spvek{ \max \finset{\gamma_a, \gamma_b}; c_b}; k_b + 1}$
        
        $\delta'(a, \Spvek{q_b; k_b}, c) := \Spvek{q_b; k_b + 1}$
\end{comment}



\begin{comment}
    \begin{satz}
        Für $\mathcal{A}(w) := a_1...a_{|w|} \in \B^*$ mit \[
           a_i :=
           \begin{cases}
             \chr{1} & \text{falls } \exists j \in \Nz: i = 2^j \\
             \chr{0} & \text{sonst}
           \end{cases}
        \]
        gilt $\CAL{\CART, \orakel({\mathcal{A}})} = \CAL{\CART}$.
    \end{satz}
    
    
    
    \begin{definition}
        Setze $C_{exp} := (Q, \delta)$ mit $Q := \B \times (\B \cup \finset{ \perp })$.
        $Q$ setzt sich aus zwei Komponenten zusammen.
        Die erste Komponente eines Zustands zu einem Zeitpunkt $t$ an der Position $i$ signalisiert, ob $i+t$ eine Zweierpotenz sein kann.
        In der Abbildung wird dies durch die Farbe blau dargestellt.
        Die zweite Komponente bestimmt den Zustand des Filters. In der Abbildung kennzeichnet Grün einen geöffneten Filterzustand (1), Rot einen geschlossenen (0)
        und keine Markierung einen nicht initialisierten Filter ($\perp$).
        
        Die Zustandsübergangsfunktion $\delta$ ist dann wie folgt definiert:
        \[
            \delta(\cdot, (e, f), (e_R, f_R)) :=
            \begin{cases}
                (e_R, \perp) & \text{ falls } f = \perp \text{ und } f_R \neq 0 \\
                (e_R, 1) & \text{ falls } f = \perp \text{ und } f_R = 0 \\
                (0, f) & \text{ falls } f \neq \perp \text{ und } e_R = 0 \\
                (f, 1 - f) & \text{ falls } f \neq \perp \text{ und } e_R = 1
            \end{cases}
        \]
        
        Für eine Start-Konfiguration
        ${c_{exp}}_i = \begin{cases} (0, \perp) & \text{ falls } i < 1  \\ (1, 1) & \text{ falls } i = 1 \\ (1, \perp) & \text{ falls } i > 1 \end{cases}$
        ergibt sich folgendes Zustands-Zeit-Diagramm:
        
        \includegraphics[scale=.5]{exp1}
        
        Offensichtlich ist $\delta$ links-unabhängig.
        
        Nach Anwendung von \cref{linksunabhaengigSpeedup} ergibt sich folgendes Zustands-Zeit-Diagramm:
        \includegraphics[scale=.5]{exp2}
    \end{definition}
    
    
    \begin{lemma}
        \label{lemmaCExpDetail}
        Setze $c^t_i := \Delta^t_{C_{exp}}(c_{exp})_i$.
        Es bezeichne $(c^t_i)_e$ die erste Komponente 
        und $(c^t_i)_f$ die zweite Komponente des Tupels $c^t_i$.
        Es gilt:
        \begin{itemize}
            \item[1)] $(c^t_i)_e = 1$, falls $i \geq 2$.
            \item[2)] $(c^t_i)_e = 1$, falls $i+t \in 2^{2-i}\N$.
            \item[3)] $(c^t_i)_e = 1$, falls $i+t \in \set{2^k}{0 \leq k \leq 1-i}$.
            \item[4)] $(c^t_i)_e = 0$, falls $i < 2$ und $i+t \not\in 2^{2-i}\N \cup \set{2^k}{0 \leq k \leq 1-i}$.
            \item[5)] $(c^t_i)_f = \perp$, falls $i \geq 2$.
            \item[6)] $(c^t_i)_f = \perp$, falls $i + t < 2^{1-i}$.
            \item[7)] $(c^t_i)_f = \lfloor \frac{i+t}{2^{1-i}} \rfloor \; \mathrm{mod} \; 2$, falls $i < 2$ und $i + t \geq 2^{1-i}$.
        \end{itemize}
        
        
        \[
            (c^t_i)_e = \begin{cases}
                1 & \text{falls } 
            i \geq 2 \text{ oder } i+t \in (2^{2-i})\N \text{ oder } i+t \in \set{2^k}{0 \leq k \leq 1-i} \\
                0 & \text{sonst}
            \end{cases}
        \]
        und
        \[
            (c^t_i)_f = \begin{cases}
                \perp & \text{ falls } i \geq 2 \text{ oder } i + t < 2^{1-i} \\
                \lfloor \frac{i+t}{2^{1-i}} \rfloor \; \mathrm{mod} \; 2 & \text{ sonst }
            \end{cases}
        \]
    \end{lemma}
    \begin{proof}
        Beweis aller Aussagen durch simultane Induktion über $t$.
        Einsetzen zeigt, dass alle Aussagen für $t = 0$ gelten.
        
        Es gelten nun alle Aussagen für ein $t \geq 0$.
        
        \begin{itemize}
            \item[1), 5)]
                Sei $i \geq 2$.
                
                Wegen 1) und 5) gilt $(c^t_{i+1})_e = 1$,
                $(c^t_{i})_f = \perp$ und $(c^t_{i+1})_f = \perp \neq 0$.
                Damit $c^{t+1}_i = \delta_{C_{exp}}(?, (?, \perp), (1, \perp)) = (1, \perp)$.
            
            \item[2)]
                Sei $i+(t+1) = 2^{2-i}k$ für ein $k \in \N$. OBdA ist $i < 2$.
                
                Wegen $i+t \geq 2^{2-i} - 1 \geq 2^{1-i}$ folgt mit 7)
                \begin{align*}
                (c^t_i)_f
                & = \lfloor \frac{i+t}{2^{1-i}} \rfloor \; \mathrm{mod} \; 2 \\
                & = \lfloor \frac{2^{2-i}k - 1}{2^{1-i}} \rfloor \; \mathrm{mod} \; 2
                = \lfloor 2k - \frac{1}{2^{1-i}} \rfloor \; \mathrm{mod} \; 2 \\
                & = 2k-1 \; \mathrm{mod} \; 2 = 1
                \end{align*}
                
                Weiter gilt wegen 2)
                $(c^t_{i+1})_e = 1$, da $(i + 1) + t \in 2^{2-i}\N$.
                Also gilt $(c^{t+1}_i)_e = \delta_{C_{exp}}(?, (?, 1), (1, ?)) = 1$.
                
            \item[3)]
                Sei nun $i + (t + 1) \in \set{2^k}{0 \leq k \leq 1-i}$. OBdA ist $i < 2$.
                
                Wegen 6) gilt $(c^t_i)_f = \perp$, da $i + t < i + t + 1 \leq 2^{1 - i}$.
                Außerdem gilt wegen 3)
                $(c^t_{i+1})_e = 1$, da $i + t + 1 \in \set{2^k}{0 \leq k \leq 1-(i+1)} \; \cup \; \finset{2^{2-(i+1)}}$.
                Damit folgt $(c^{t+1}_i)_e = \delta_{C_{exp}}(?, (?, \perp), (1, ?))_e = 1$.
                
            \item[4)]
                Sei $i < 2$, $i + (t+1) \not\in 2^{2-i}\N \cup \set{2^k}{0 \leq k \leq 1-i}$.
                
                Angenommen, $(c^{t+1}_i)_e = 1$.
                Nach Definition von $\delta_{C_{exp}}$ folgt
                $(c^t_{i+1})_e = 1$ und $(c^t_i)_f \neq 0$.
                Wegen 4) gilt $i+1 \geq 2$ oder $(i+1)+t \in 2^{2-(i+1)}\N \; \cup \; \set{2^k}{0 \leq k \leq 1-(i+1)}$.
                Folglich $i = 1$ oder $i+t+1 \in (2^{2-i}\N) + 2^{1-i}$.
                Letzteres gilt auch im Fall $i = 1$.
                Damit gibt es ein $k \in \N$, sodass $i+t = 2^{2-i}k + 2^{1-i} - 1$.
                Da insbesondere $i + t \geq 2^{2-i} + 2^{1-i} - 1 > 2^{1-i}$ gilt, ergibt sich mit 7) folgendes:
                \begin{align*}
                    (c^t_i)_f
                    & = \lfloor \frac{i+t}{2^{1-i}} \rfloor \; \mathrm{mod} \; 2 \\
                    & = \lfloor \frac{2^{2-i}k + 2^{1-i} - 1}{2^{1-i}} \rfloor \; \mathrm{mod} \; 2
                    = \lfloor 2k + 1 - \frac{1}{2^{1-i}} \rfloor \; \mathrm{mod} \; 2 \\
                    & = 2k \; \mathrm{mod} \; 2
                    = 0
                \end{align*}
                Was der Annahme widerspricht. Damit gilt $(c^{t+1}_i)_e = 0$.
            
            \item[6)]
                Sei nun $i + (t+1) < 2^{1 - i}$.
                Mit 6) folgt wegen $i + t < 2^{1 - i}$, dass $(c^t_i)_f = \perp$.
                
                Angenommen, $(i+1)+t < 2^{1-(i+1)}$. Dann gilt wegen 6) $(c^t_{i+1})_f = \perp \neq 0$.
                Andernfalls, $(i+1)+t \geq 2^{1-(i+1)}$. Dann $2^{1-(i+1)} \leq (i+1)+t < 2^{1-i}$.
                Daraus folgt $\lfloor \frac{(i+1)+t}{2^{1-(i+1)}} \rfloor = 1$.
                Auch hier gilt mit 7) dann $(c^t_{i+1})_f = 1 \neq 0$.
                
                Es folgt $(c^{t+1}_i)_f = \delta_{C_{exp}}(?, (?, \perp), (?, ? \neq 0))_f = \perp$.
            
            \item[7)]
                Sei nun $i < 2$ und $i + (t + 1) \geq 2^{1-i}$.
                \begin{itemize}
                    \item[Fall 1)] $(c^t_i)_f = \perp$.
                        Wegen 7) gilt $i + t + 1 = 2^{1-i}$.
                        Dann $(c^t_{i+1})_f = \lfloor \frac{(i+1)+t}{2^{1-(i+1)}} \rfloor \; \mathrm{mod} \; 2 = 0$,
                        also $(c^{t+1}_i)_f = \delta_{C_{exp}}(?, (?, \perp), (?, 0))_f = 1 = \lfloor \frac{i+(t+1)}{2^{1-i}} \rfloor \; \mathrm{mod} \; 2$.
                    \item[Fall 2)] $(c^t_i)_f \neq \perp$.
                        \begin{itemize}
                            \item[Fall 2.1)] $(c^t_{i+1})_e = 0$.
                                Wegen 4) gilt dann $(i + 1) + t \not\in (2^{2-(i+1)})\N$.
                                Nach Definition von $\delta_{C_{exp}}$ gilt $(c^{t+1}_i)_f = (c^t_i)_f$.
                                
                                Angenommen $
                                (c^{t+1}_i)_f
                                = (c^t_i)_f
                                = \lfloor \frac{i+t}{2^{1-i}} \rfloor \; \mathrm{mod} \; 2
                                \neq \lfloor \frac{i+(t+1)}{2^{1-i}} \rfloor \; \mathrm{mod} \; 2
                                $.
                                Dann ergibt sich durch $i+(t+1) \in 2^{1-i}\N$ ein Widerspruch.
                            
                            \item[Fall 2.2)] $(c^t_{i+1})_e = 1$.
                                Wegen 4) ist dann $(i+1)+t \in 2^{2-(i+1)}\N$.
                                Nach Definition gilt $(c^{t+1}_i)_f = 1 - (c^t_i)_f$.
                                
                                Es bleibt also zu zeigen, dass
                                $\lfloor \frac{i+(t+1)}{2^{1-i}} \rfloor \; \mathrm{mod} \; 2
                                \neq \lfloor \frac{i+t}{2^{1-i}} \rfloor \; \mathrm{mod} \; 2$.
                                
                                Diese Ungleichung gilt wegen $i+1+t \in 2^{1-i}\N$.
                        
                        \end{itemize}
                \end{itemize}
                
                
                
            
        \end{itemize}
    \end{proof}
    \begin{lemma}
        \label{lemmaCExpProp2}
        Für $i < 2$ und $t < 3*2^{2-i} - i$ gilt: 
        \[
            (\Delta_{C_{exp}}^t(c_{exp})_i)_e = 1 \Leftrightarrow \exists k \in \Nz: i + t = 2^k
        \]
    \end{lemma}
    \begin{proof}
        Folgt direkt aus \cref{lemmaCExpDetail}.
    \end{proof}
    
    
    \begin{lemma}
        \label{lemmaNumberTheoryInequality}
        Sei $t, i \in \Nz$ mit $i \geq 1$ und $t \geq 2i - 2$.
        Dann $i - 2 < t < 3 * 2 ^ {2+t-i}-i$.
    \end{lemma}
    \begin{proof}
        Die erste Ungleichung ist trivial.
        Es gilt $1 \leq i \leq \frac{t}{2} + 1$.
        Wenn die zweite Ungleichung nicht gilt, gilt sie insbesondere für solche $i$ nicht, 
        die für ein festes $t$ maximal gewählt werden und damit für $i = \frac{t}{2} + 1$.
        Es bleibt zu zeigen, dass $\frac{3}{2}t + 1 < 6 * 2^{\frac{t}{2}}$.
        Diese Ungleichung gilt offensichtlich für $t = 0$ und $t = 1$ und kann
        leicht induktiv mit Schrittweite 2 auf alle natürlichen Zahlen fortgesetzt werden.
    \end{proof}
    
    \begin{satz}
        Es gibt einen Zellularautomaten $C$,
        sodass für $i \in \Z$ und $t \in \Nz$ mit $i - 2 < t < 3*2^{2-i+t}-i$ gilt:
        \[
            (\Delta^{t}_{C}(c_{exp})_i)_e = 1 \Leftrightarrow \exists k \in \Nz: i + t = 2^k
        \]
        Wegen \cref{lemmaNumberTheoryInequality} gilt die Aussage insbesondere,
        falls $i \geq 1$ und $t \geq 2i - 2$.
    \end{satz}
    \begin{proof}
        Nach \cref{linksunabhaengigSpeedup} gibt es einen Zellularautomaten $C$,
        sodass $\Delta^t_{C}(c_{exp})_i = \Delta^{2t}_{C_{exp}}(c_{exp})_{i-t}$.
        Aus den Voraussetzungen folgt, dass $i-t < 2$ und $2t < 3 * 2^{2-(i-t)}-(i-t)$.
        Es gilt damit mit \cref{lemmaCExpProp2}:
        \[
            i + t = (i-t) + 2t = 2^k
            \Leftrightarrow(\Delta^{2t}_{C_{exp}}(c_{exp})_{i-t})_e = 1
            \Leftrightarrow (\Delta^t_{C}(c_{exp})_{i})_e = 1
        \]        
        
    \end{proof}

\end{comment}

\begin{definition}[Präfixstabil]
    Ein Advice $\mathcal{A}$ heißt präfixstabil, wenn für alle $w, s \in \Sigma^*$ gilt:
    \[
        \mathcal{A}(w) =
        \set{ v[1, ..., |w|] } { v \in \mathcal{A}(ws) }
    \]
\end{definition}

\begin{lemma}
    \label{wrtRealtimeLengthIdealAdvice}
    
    Sei $\mathcal{A}$ ein präfixstabiles Advice, $k \in \N$ und $L \in \CAL{\CART, \CALeft, \orakel({\mathcal{A}})}$.
    Dann gilt für $L(k)$ aus \cref{wrtRealtimeLengthIdeal}:
    \[
        L(k) \in \CAL{\CART, \CALeft, \orakel({\mathcal{A}})}
    \]
    
    Sei $\mathcal{M}_k$ die Menge alle Sprachen, deren Wortlängen Vielfache von k sind.
    Es gilt dann folgende Implikation:
    \begin{align*}
        \CAL{\CART, \CALeft, \orakel({\mathcal{A}})} \cap \mathcal{M}_k  & = \CAL{\CART, \CALeft} \cap \mathcal{M}_k \\
        \Rightarrow
        \CAL{\CART, \CALeft, \orakel({\mathcal{A}})} & = \CAL{\CART, \CALeft}
    \end{align*}
\end{lemma}
\begin{proof}
    Analog zum Beweis von \cref{wrtRealtimeLengthIdeal}:
    Da $\mathcal{A}$ präfixstabil ist, können Zeichen am Ende des Wortes gelöscht werden, ohne dass sich
    das Advice auf dem übrig gebliebenen Teil des Wortes ändert.
    
    Ist nun $L \in \CAL{\CART, \CALeft, \orakel({\mathcal{A}})}$,
    gilt $L(k) \in \CAL{\CART, \CALeft, \orakel({\mathcal{A}})} \cap \mathcal{M}_k$
    und damit auch
    $L(k) \in \CAL{\CART, \CALeft} \cap \mathcal{M}_k$.
    Mit \cref{wrtRealtimeLengthIdeal} folgt dann $L \in \CAL{\CART, \CALeft}$.
\end{proof}


\begin{theorem}
    \label{thmAdviceMain}
    Sei $H = (Q, \delta) \in \CA$ ein Zellularautomat mit $\Sigma \subseteq Q$.
    Der folgende Advice $\mathcal{A}_H$ ist $\CAM{\CART, \CALeft}$-verträglich:
    \[
        \mathcal{A}_H(w) := v \in Q^{|w|} \text{ mit } v_i := \Delta_A^{i-1}([w])_1 \\
    \]
\end{theorem}
\begin{proof}
    Die Inklusion \enquote{$\subseteq$} der $\CAM{\CART, \CALeft}$-Verträglichkeit folgt durch ~\cref{lemmaIgnoriereAdvice}.
    Sei nun $L \in \CAL{\CART, \CALeft, \orakel({\mathcal{A}_H})}$.
    Da $\mathcal{A}_H$ präfixstabil ist, kann wegen \cref{wrtRealtimeLengthIdealAdvice} und \cref{endlichVieleAusnahmen}
    \ac{OBdA.} angenommen werden,
    dass die Wortlängen von $L$ alle Vielfache von $3$ sind und dass $L$ nicht das leere Wort enthält.

    Es existiert also ein Echtzeit-Zellularautomat $C$ mit $L_{\mathcal{A}, C} = L$. Nach \cref{satzRauteTot} kann der Rand von $C$ tot gewählt werden.
    Sei $w \in \Sigma^*$ mit $|w| \in 3\N$, $c := [w]$, $v := \mathcal{A}_H(w)$ und $c' := [\comb(\Spvek{ w; v})]$.
    Nach \cref{CAgfSpeedup} gibt es einen Automaten $C_1$ und eine Funktion $f_1$, sodass für $i \in \N$ gilt:
    \[
        f_1(\Delta_{C_2}^{2i+1}(c)_i) = (\Delta_C^{3i-3}(c)_1, \Delta_C^{3i-2}(c)_1, \Delta_C^{3i-1}(c)_1)
    \]
    
    Ferner lässt sich leicht ein Automat $K$ konstruieren und eine Funktion $f_2$ angeben, sodass für $i \geq 1$ gilt:
    \[
        f_2(\Delta_{K}^{2i+1}(c)_i) = (c_{3i-2}, c_{3i-1}, c_{3i})
    \]
    
    Indem $C_1$ und $K$ gleichzeitig ausgeführt werden, lässt sich ein Automat $M$ konstruieren und eine Funktion $f$ angeben,
    sodass für $n := \frac{|w|}{3}$ gilt:
    \[
        f(\Delta_M^t(c)_i) =
        \begin{cases}
            \mathrm{reset} & \text{falls } t = 0 \text{ und } i \in \finset{1, .., n} \\
            \mathrm{set}(\#\#\#) & \text{falls } t = 0 \text{ und } i \not\in \finset{1, .., n} \\
            \mathrm{set}(s_i)
                & \text{falls } i \in \finset{1, ..., n} \text{ und } t = 2 * i + 1 \\
            \mathrm{step} & \text{sonst}
        \end{cases}
    \]
    Wobei $s_i$ für $i \in \Z$ wie folgt definiert ist:
    \[
        s_i := \begin{cases}
                    \Spvek{ c_{3i-2} ; \Delta_C^{3i-3}(c)_1 }
                        \Spvek{ c_{3i-1} ; \Delta_C^{3i-2}(c)_1} 
                        \Spvek{ c_{3i} ; \Delta_C^{3i-1}(c)_1}
                        \in (\Sigma \times Q_H)^3
                        & \text{falls } i \in \finset{ 1, ..., n } \\
                    \#\#\# \in Q_C & \text{sonst} 
                \end{cases}
    \]
    
    Damit gilt aber gerade $s = S_3(c')$ wobei $S_3(c')$ die komprimierte $c'$-Konfiguration aus \cref{factorSpeedupConstruction} ist.
    Setze $S := S_3(C)$ auf die Speedup-Konstuktion, die drei Schritte in einem simuliert. Setze nun $A := A(M, f, S, \finset{\#})$.
    Dann gilt nach \cref{timeNakamuraConstruction}:
    \[
        (\Delta^{ |w| }_A(c)_1)_a = \Delta_S^{n - 1}(s)
    \]
    Da $\#$ in $C$ und damit auch $\#\#\#$ in $S$ tot ist, gilt damit nach \cref{factorSpeedupConstructionCorrectness}:
    \[
        \gamma_\#((\Delta^{ |w| }_A(c)_1)_a) = \gamma_\#(\Delta_S^{n-1}(s)) = \Delta_C^{|w| - 1}(c')
    \]
    
    Mit entsprechender Wahl der akzeptierenden Zustände und einem konstanten Speedup von einem Schritt
    folgt damit $L \in \CAL{\CART, \CALeft}$.
\end{proof}

\begin{corollary}
    Sei $L \in \CAL{\CART, \CALeft}$.
    Der folgende Advice $\mathcal{A}_L$ ist $\CAM{\CART, \CALeft}$-verträglich:
    \[
        \mathcal{A}_L(w) := v \in \B^{|w|} \text{ mit } v_i :=
        \begin{cases}
            1 & \text{ falls } w_{[1...i]} \in L \\
            0 & \text{ sonst }
        \end{cases}
    \]
\end{corollary}
\begin{proof}
    Nach Wahl von $L$ gibt es einen Zellularautomaten $C \in \CAM{\CART, \CALeft}$
    mit $L_C = L$. Damit gilt für ein $w \in \Sigma_C^*$ und alle $t \in \finset{0, ..., |w|-1}$:
    \[
        \Delta_C^{t}([w])_1 \in
        \begin{cases}
            F_C^+ & \text{falls } w_{[1...t+1]} \in L \\
            Q \setminus F_C^+ & \text {sonst}
        \end{cases}
    \]
    Indem Zustände aus $F_C^+$ als $1$ und Zustände aus $Q \setminus F_C^+$ als $0$ aufgefasst werden, 
    folgt mit \cref{thmAdviceMain} die Behauptung.
\end{proof}


\begin{satz}
    Der folgende Advice $\mathcal{A}$ ist $\CAM{\CART, \CALeft}$-verträglich:
    \[
        \mathcal{A}(w) := \mathrm{ones}(2^{\lfloor \log_2{|w|} \rfloor - 1}, |w|) \in \B^*
    \]
\end{satz}
\begin{proof}
    
\end{proof}

\chapter{Eingeschränkte Zellularautomaten}
\label{chap:EingeschrAuto}

In diesem Kapitel wird die Auswirkung der Forderung nach passiven Zellen im Eingabewort
bei Echtzeit-Zellularautomaten untersucht.

\section{Definition}

Zunächst wird definiert, inwiefern Echtzeit-Zellularautomaten eingeschränkt werden.
Dazu werden Advices aus dem vorherigen Kapitel verwendet und ein entsprechender Einschränkungs-Funktor definiert.

\begin{definition}
    Zu einer gegebenen Funktion $d: \Nz \to \N$ kann ein $\Sigma$-$\B$-Advice $d^*$ wie folgt konstruiert werden:
    \[
        d^*(w) := \mathrm{ones}(d(|w|), |w|)^{Rev}
    \]        

    Dies definiert den $\CARestr$-$\ECA$-Funktor zum Einschränken von Zellularautomaten:
    \begin{multline*}
        \CARestr(d)(M \subseteq \ECA) := \\
            \set{ C \in \orakel(d^*)(M) }{
                (\Sigma_C \times \finset{ \texttt{0} }) \union \finset{ \#_C } \text{ ist eine passive Zustandsmenge} }
    \end{multline*}
\end{definition}

Es mag auf den ersten Blick seltsam sein, dass Advices, die Zellularautomaten \acs{ggf.} mächtiger machen,
dazu verwendet werden, Zellularautomaten einzuschränken.
Tatsächlich werden die Sprachklassen durch den $\CARestr$-Funktor nicht nur verkleinert,
wie \cref{einschraenkungUnentscheidbar} zeigt.
Mit sinnvollen Wahlen der Einschränkungsfunktion $d$ ergeben sich aber auch sinnvolle Einschränkungen.

\section{Ergebnisse}

Zunächst wird bemerkt, dass der $k$-Schritt Speedup für Zellularautomaten aus \cref{satzEchtzeitSpeedup}
auch für eingeschränkte Zellularautomaten gilt.

\begin{satz}[$k$-Schritt Speedup für eingeschränkte Zellularautomaten]
    \label{satzEingeschraenktEchtzeitSpeedup}
    Sei $d$ eine Funktion $d: \Nz \to \N$. Es gilt:
    \[
        \CAL{\CART, \CALeft, \CARestr(d)} = \CAL{\CAT, \CALeft, \mathrm{time}(\set{ n \mapsto \max \finset{0,  n + k - 1} }{ k \in \Nz }), \CARestr(d)}
    \]
\end{satz}
\begin{proof}
    Siehe Beweis von \cref{satzEchtzeitSpeedup}:
    \cref{satzRauteTot} erhält passive \#-enthaltende Mengen und 
    in der Konstruktion des Automaten $C'$ in \cref{satzEchtzeitSpeedup}
    kann $\Sigma_C \times \finset{\chr{0}}$ mit sich selbst und
    $q \in \Sigma_C \times \finset{\chr{1}}$ mit $\phi_{\#}(q)$ identifiziert werden, ohne dass der Beweis seine Gültigkeit verliert.
    Die Zelle am rechten Rand ist schließlich immer Element von $q \in \Sigma_C \times \finset{\chr{1}}$.
    Passive \#-enthaltende Mengen bleiben also insgesamt erhalten, sodass der konstruierte Automat
    den Bedingungen eingeschränkter Zellularautomaten entspricht.
\end{proof}

\subsection{Relation \texorpdfstring{$\sim_{k,d,L}$}{sim\_kdL}}

Es hat sich als beweistechnisch sinnvoll herausgestellt, eine Relation $\sim_{k,d,L}$ als Verallgemeinerung zur Nerode-Äquivalenzrelation zu betrachten.

\begin{definition}
    Seien $k \in \Nz$, $d$ eine Funktion $d: \Nz \to \N$ und $L \subseteq \Sigma^*$.
    Angelehnt an die Nerode-Äquivalenzrelation wird auf $\Sigma^*$ die Relation $\sim_{k,d,L}$ definiert.
    Seien dazu $v_1, v_2 \in \Sigma^*$.
    \[
        v_1 \sim_{k,d,L} v_2 \; :\Leftrightarrow \; \forall w \in \Sigma^*: d(|wv_1|) = d(|wv_2|) = k \Rightarrow (wv_1 \in L \Leftrightarrow wv_2 \in L)
    \]
    
    $\sim_{k,d,L}$ ist allerdings im Allgemeinen keine Äquivalenzrelation!
\end{definition}

Eng mit der eingeführten Relation $\sim_{k,d,L}$ ist die Funktion $f_{C,k}$ verwandt, wie \cref{satzFEqualsImpliesEquiv} zeigt.
\begin{definition}
    Sei $C \in \CAM{\CART, \CALeft, \CARestr(d)}$ für ein $d: \Nz \to \N$. Definiere die Funktion $f_{C, k}: \Sigma^* \to Q_C^*$
    wie in \cref{fig:RestrAutomata_fCk} gezeigt:
    \[
        f_{C, k}(v) := w[1..\min(k+1, |v|)] \; \text{ mit } \; w := \Delta_C^{\max(0, |v|-k-1)}(  \joinw(\Spvek{ v; \mathrm{ones}(k, |v|)^{Rev} })  )
    \]
    
    \begin{figure}[h!]
        \begin{center}
        \includesvg{fCk}
        \end{center}
        \caption{Berechnung von $f_{C,k}(v)$ für $k = 2$}
        \label{fig:RestrAutomata_fCk}
        Dunkelgraue Zustände sind passiv. Hellgraue Zustände sind für das Akzeptanzverhalten relevant.
    \end{figure}
    
\end{definition}

\begin{satz}
    \label{satzFEqualsImpliesEquiv}
    Sei $C \in \CAM{\CART, \CALeft, \CARestr(d)}$ für ein $d: \Nz \to \N$. Es gilt für alle $k \in \Nz$ und $v_1, v_2 \in \Sigma^*$:
    \[
        f_{C, k}(v_1) = f_{C, k}(v_2) \Rightarrow v_1 \sim_{k,d,L} v_2
    \]
\end{satz}
\begin{proof}
    Sei $k \in \Nz$. Seien $v_1, v_2 \in \Sigma^*$ so, dass $f_{C, k}(v_1) = f_{C, k}(v_2)$.
    Sei $w \in \Sigma^*$ mit $d(|wv_1|) = d(|wv_2|) = k$.
    Es bleibt zu zeigen, dass $wv_1 \in L$ genau dann gilt, wenn $wv_2 \in L$.
    
    Angenommen, $|v_1| < k + 1$. Dann 
    \[
        |f_{C, k}(v_2)| = |f_{C, k}(v_1)| = |v_1| < k + 1
    \]
    und damit $|f_{C, k}(v_2)| = \min(k+1, |v_2|) = |v_2|$, also $|v_1| = |v_2|$.
    Wegen $\max(0, |v_{i \in \finset{1,2}}|-k-1) = 0$ folgt $v_1 = v_2$, also gilt offensichtlich $v_1 \sim_{k,d,L} v_2$.
    
    Seien nun \acs{OBdA.} $|v_1|, |v_2| > k$.
    Setze $u_{i \in \finset{1, 2}} := \joinw(\Spvek{ wv_i; d^*(wv_i) })$
    und $c^t_i := \Delta_C^{t}( u_i )$.
    
    Es gilt wegen der Forderung nach passiven Zuständen (rote Markierung in \cref{fig:RestrAutomata_fCk_Equiv}, graue Zellen sind passiv): 
    \[(c_1^{|v_1|-k-1})_{[1..|w|]} = \joinw(\Spvek{ w; \chr{0}^{|w|} }) = (c_2^{|v_2|-k-1})_{[1..|w|]}\]
    
    Und es gilt (blaue Markierung in \cref{fig:RestrAutomata_fCk_Equiv}): 
    \[(c_1^{|v_1|-k-1})_{[|w|+1..|w|+k+1]} = |f_{C, k}(v_1)| = |f_{C, k}(v_2)| = (c_2^{|v_2|-k-1})_{[|w|+1..|w|+k+1]}\]
    
    
    Also gilt zusammen (grüne Markierung in \cref{fig:RestrAutomata_fCk_Equiv}): 
    \[(c_1^{|v_1|-k-1})_{[1..|w|+k+1]} = (c_2^{|v_2|-k-1})_{[1..|w|+k+1]}\]
    
    \begin{figure}[h!]
        \centering
        \includesvg[width=380pt]{fCkImpliesEquiv}
        \caption{$f_{C,k}(v_1) = f_{C,k}(v_2) \Rightarrow v_1 \sim_{k,d,L} v_2$}
        \label{fig:RestrAutomata_fCk_Equiv}
    \end{figure}
    
    Und damit ist auch der Zustand $f$ aus \cref{fig:RestrAutomata_fCk_Equiv} in beiden Fällen gleich,
    es folgt also schließlich $wv_1 \in L \; \Leftrightarrow \; wv_2 \in L$.
\end{proof}

Damit ergibt sich dann folgende praktische Aussage.

\begin{corollary}
    \label{satzanzahleqivclass}
    Sei $C \in \CAM{\CART, \CALeft, \CARestr(d)}$ für ein $d: \Nz \to \N$, $k \in \Nz$ und
    $M \subseteq \Sigma^*$ eine Menge paarweiser nicht-ähnlicher Wörter bezüglich $\sim_{k,d,L}$. Es gilt:
    \[
        |M| \leq |f_{C, k}(\Sigma^*)| \leq |Q_C|^{k + 2}
    \]
\end{corollary}
\begin{proof}
    Angenommen, $|M| > |f_{C, k}(\Sigma^*)|$. Wegen des Schubfachprinzips gibt es $m_1, m_2 \in M$ mit $f_{C,k}(m_1) = f_{C,k}(m_2)$ und wegen \cref{satzFEqualsImpliesEquiv} folgt mit $m_1 \sim_{k,d,L} m_2$ ein Widerspruch zur Wahl von $M$.
    
    Die zweite Ungleichung folgt aus $|f_{C,k}(\cdot)| \leq k + 1$.
\end{proof}

\subsection{Konstante Einschränkung}

Nicht verwunderlich, kann ein Zellularautomat, der nur endlich viele Zellen zur Berechnung verwenden darf,
auch nur reguläre Sprachen erkennen. Dies ist eine direkte Folgerung aus \cref{satzanzahleqivclass}.

\begin{satz}
    $\forall c \in \N, d(\cdot) := c: \CAL{\CART, \CALeft, \CARestr(d)} = \textrm{REG}$
\end{satz}
\begin{proof}
    Da $\sim_{c,d,L}$ für eine Sprache $L$ gerade mit der Nerode-Äquivalenzrelation übereinstimmt,
    folgt aus \cref{satzanzahleqivclass}, dass es nur endlich viele Nerode-Äquivalenzklassen gibt und damit die Regularität von $L$.
    
    Umgekehrt ist die rechteste Zelle der Eingabe nie passiv.
    Zum Erkennen einer regulären Sprache $L$ kann dann der endliche Automat, der $L^{Rev}$ erkennt, benutzt werden,
    um einen entsprechenden Zellularautomaten zu konstruieren, der dann trotz Einschränkung $L$ erkennt.
\end{proof}

Sind es zwar weiterhin nur endliche viele Zellen, die die Berechnung benutzen darf,
\acs{z.B.} eine oder zwei, hängt die genaue Anzahl
aber wieder frei von der Länge des Eingabewortes ab,
so lassen sich sogar unentscheidbare Sprachen erkennen.

\begin{satz}
    \label{einschraenkungUnentscheidbar}
    $\exists d: \Nz \to \N$ mit $d(\cdot) \leq 2$, sodass $\CAL{\CART, \CALeft, \CARestr(d)}$ eine unentscheidbare Sprache enthält.
\end{satz}
\begin{proof}
    Es gibt einen Zellularautomaten, der das Eingabewort genau dann akzeptiert, wenn vom Advice genau zwei Zellen mit einer $\chr{1}$ markiert wurden.
    Die Funktion $d$ kann so abhängig von der Länge des Wortes Informationen über unentscheidbare Sprachen an den Zellularautomaten weitergeben.
\end{proof}

\subsection{Vergleich verschiedener Einschränkungen}

Nicht verwunderlich, schränkt es einen Echtzeit-Zellularautomat kaum ein, wenn zur Berechnung nur endlich viele Zellen nicht verwendet werden dürfen.
\begin{satz}
    Sei $c \in \Nz$. Dann gilt für $d(n) := n - c$:
    \[
        \CAL{\CART, \CALeft, \CARestr(d)} = \CAL{\CART, \CALeft}
    \]
\end{satz}
\begin{proof}
    \cref{bestimmteEinfacheAdvices} zeigt die Inklusion \enquote{$\subseteq$}.
    Für die andere Inklusion kann die asynchrone Simulations-Konstruktion aus \cref{chap:ErweiterteNakamuraKonstr}
    verwendet werden, um dann mit einem entsprechenden Speedup-Faktor (siehe \cref{factorSpeedupConstruction})
    die eigentliche Berechnung ausführen, sobald nach $c$ Schritten keine Zelle im Eingabewort mehr passiv ist.
\end{proof}

Das wohl interessanteste Ergebnis dieses Kapitels ist das folgende: Je weniger Zellen dem Automat für die Berechnung zur
Verfügung stehen (asymptotisch in Abhängigkeit zur Wortlänge gesehen), desto weniger Sprachen kann er erkennen.

\begin{satz}
    \label{d1WenigerAlsd2}
    Seien $d_1, d_2: \Nz \to \N$ zwei Funktionen, sodass $d_1(n) \leq \frac{n}{2}$ und $\liminf\limits_{n \rightarrow \infty} \frac{d_2(n)}{d_1(n)} = 0$.
    Dann gibt es ein $L \in \CAL{\CART, \CALeft, \CARestr(d_1)} \setminus \CAL{\CART, \CALeft, \CARestr(d_2)}$.
\end{satz}
\begin{proof}
    Wähle $L := \set{ w^{Rev}vw }{ v, w \in \B^*, \; |w| = d_1(|wvw|) }$.
    
    Es gilt $L \in \CAL{\CART, \CALeft, \CARestr(d_1)}$, da die nicht-passiven Zellen genau das letzte Vorkommen von $w$ markieren.
    Die aktiven Zeichen wandern dann nach links und versuchen, den Anfang von $w^{Rev}$ zu finden.
    Bereits gefunden Anfänge werden dann Schritt für Schritt verlängert. Anfänge, die nicht fortgesetzt werden können, werden verworfen.

    Es gilt aber auch $L \not\in \CAL{\CART, \CALeft, \CARestr(d_2)}$:
    Sei $n \in \Nz$ und $k := d_1(n)$. Wegen $d_1(n) \leq \frac{n}{2}$ gibt es ein $v \in \B^*$, sodass $n = |v| + 2k$.
    Seien $w_1, w_2 \in \B^k$ zwei verschiedene Wörter der Länge $k$.
    Dann gilt $w_1^{Rev}vw_1 \in L$, aber $w_2^{Rev}vw_1 \not\in L$, also $w_1 \not\sim_{d_2(n),d,L} w_2$.
    Damit ist $\B^k$ eine Menge paarweiser nicht-ähnlicher Wörter und $|\B^k| = 2^k$.
    
    Angenommen, es gibt ein $C \in \CAM{\CART, \CALeft, \CARestr(d_2)}$ mit $L_C = L$.
    
    Nach \cref{satzanzahleqivclass} gilt:
    \[
        \forall n \in \Nz:  2^{d_1(n)} = |\B^k| \leq {|Q_C|}^{d_2(n)+2} = 2^{\log_2(|Q_C|)(d_2(n)+2)}
    \]
    
    Also gilt:
    \[
        1 \leq \log_2(|Q_C|) ( \frac{ d_2(n) }{ d_1(n) } + \frac{ 2 }{ d_1(n) })
    \]
    Laut Annahme gibt es nun eine monoton steigende Folge $a_n$, sodass $\frac{d_2(a_n)}{d_1(a_n)}$ beliebig klein wird.
    Da dafür $d_1(a_n)$ beliebig groß wird, folgt ein Widerspruch zur Annahme.
\end{proof}

\begin{remark}
    \cref{d1WenigerAlsd2} besagt allerdings nicht, dass die Sprachklassen für langsamer wachsende Einschränkungen Teilmengen sind:
    \cref{einschraenkungUnentscheidbar} zeigt ja gerade, dass selbst Automaten mit fast konstanter Einschränkung
    unentscheidbare Sprachen entscheiden können, die offensichtlich nicht von Automaten mit Einschränkung $d(n) = \frac{n}{2}$ erkannt werden können.
\end{remark}



\begin{comment}

\begin{satz}
    $\forall d \in O(n): \CAL{\CALT, \CALeft, \CARestr(d)} = \CAL{\CALT, \CALeft, \orakel(d^*)}$
\end{satz}



\begin{satz}
    $d(n) := \lceil n / 2 \rceil: \CAL{\CART, \CALeft \CARestr(d)}  \supseteq \CAL{\CART, \CALeft}$
\end{satz}
\end{comment}

\begin{comment}
\begin{satz}
    $\forall c \in \N, d(n) := \lceil n / c \rceil: \CAL{\CART, \CALeft \CARestr(d)}  \supseteq \CAL{\CART, \CALeft}$
\end{satz}
\begin{proof}
    Bisher nur für $c = 2$.
\end{proof}

\begin{satz}
    $\forall c \in \N, d(n) := n - 2^{\lfloor \log_2{n} \rfloor - 1}: \CAL{\CART, \CALeft, \CARestr(d)} = \CAL{\CART, \CALeft}$
\end{satz}
\begin{proof}
    Kompliziert.
\end{proof}
\end{comment}

\chapter{Zusammenfassung}

Wie erwartet, konnte die usprüngliche Fragestellung, ob Echtzeit mit Linearzeit zusammenfällt,
nicht weiter geklärt werden - das Problem bleibt offen.
Allerdings hat \ref{chap:AdvAuto} gezeigt, dass Echtzeit-Zellularautomaten erstaunlich mächtig sind
und mit \ref{thmAdviceMain} ein Werkzeug zur Verfügung gestellt, mit dem Echtzeit-Zellularautomaten
in gewisser Hinsicht hintereinandergeschaltet werden können.
Linearzeit-Zellularautomaten sind offensichtlich abgeschlossen unter Komposition von Linearzeit-Berechnungen -
\ref{thmAdviceMain} liefert einen Ersatz für Echtzeit-Zellularautomaten.





% \include{chapters/einleitung}  % Einleitung
% \include{chapters/grundlagen}  % Grundlagen
% \include{chapters/analyse}     % Analyse
% \include{chapters/entwurf}     % Entwurf
% \include{chapters/implemen}    % Implementierung
% \include{chapters/eval}        % Evaluierung
% \include{chapters/zusammenf}   % Zusammenfassung und Ausblick

%% ++++++++++++++++++++++++++++++++++++++++++
%% Anhang
%% ++++++++++++++++++++++++++++++++++++++++++

\appendix
%\include{anhang_a}
%\include{anhang_b}

%% ++++++++++++++++++++++++++++++++++++++++++
%% Literatur
%% ++++++++++++++++++++++++++++++++++++++++++
%  mit dem Befehl \nocite werden auch nicht 
%  zitierte Referenzen abgedruckt
\cleardoublepage
\phantomsection
\addcontentsline{toc}{chapter}{\bibname}
%%
\nocite{*} % nur angeben, wenn auch nicht im Text zitierte Quellen 
           % erscheinen sollen
\bibliographystyle{template/itmabbrv} % mit abgekürzten Vornamen der Autoren
%\bibliographystyle{gerplain} % abbrvnat unsrtnat
% spezielle Zitierstile: Labels mit vier Buchstaben und Jahreszahl
%\bibliographystyle{itmalpha}  % ausgeschriebene Vornamen der Autoren
\bibliography{thesis}
%% ++++++++++++++++++++++++++++++++++++++++++
%% Index
%% ++++++++++++++++++++++++++++++++++++++++++
\ifnotdraft{
\cleardoublepage
\phantomsection
\printindex            % Index, Stichwortverzeichnis
}
\end{document}
%% end of file
